
\section{Differential Graded Algebra}
\label{sec:algebra}

In this section $\k$ will be any commutative ring. We will recap some of the basic definitions of commutative algebra in a graded setting. By \emph{linear}, \emph{module}, \emph{tensor product}, etc \dots we always mean $\k$-linear, $\k$-module, tensor product over $\k$, etc \dots.

\subsection{Graded algebra}

\begin{definition}
	A \emph{graded module} $M$ is a family of modules $\{M_n\}_{n\in\Z}$. An element $x \in M_n$ is called a \emph{homogeneous element} and said to be of \emph{degree $\deg{x} = n$}. We will often identify $M = \bigoplus_{n \in \Z} M_n$.
\end{definition}

For an ordinary module $M$ we can consider the graded module $M[0]$ \emph{concentrated in degree $0$} defined by setting $M[0]_0 = M$ and $M[0]_n = 0$ for $i \neq 0$. If clear from the context we will denote this graded module by $M$. In particular $\k$ is a graded module concentrated in degree $0$.

\begin{definition}
	A linear map $f: M \to N$ between graded modules is \emph{graded of degree $p$} if it respects the grading and raises the degree by $p$, i.e.
	$$ \restr{f}{M_n} : M_n \to N_{n+p}. $$
\end{definition}

\begin{definition}
	The graded maps $f: M \to N$ between graded modules can be arranged in a graded module by defining:
	$$ \Hom_{gr}(M, N)_n = \{ f: M \to N \I f \text{ is graded of degree } n \}. $$
\end{definition}

Note that not all linear maps can be decomposed into a sum of graded maps, so that $\Hom_{gr}(M, N) \subset \Hom(M, N)$ may be proper for some $M$ and $N$.

Recall that the tensor product of modules distributes over direct sums. This defines a natural grading on the ordinary tensor product.

\begin{definition}
	The graded tensor product is defined as:
	$$ (M \tensor N)_n = \bigoplus_{i + j = n} M_i \tensor N_j. $$
	The tensor product extends to graded maps. Let $f: A \to B$ and $g:X \to Y$ be two graded maps, then their tensor product $f \tensor g: A \tensor B \to X \tensor Y$ is defined as:
	$$ (f \tensor g)(a \tensor x) = (-1)^{\deg{a}\deg{g}} \cdot f(a) \tensor g(x). $$
\end{definition}

The sign is due to \emph{Koszuls sign convention}: whenever two elements next to each other are swapped (in this case $g$ and $a$) a minus sign appears if both elements are of odd degree. More formally we can define a swap map
$$ \tau : A \tensor B \to B \tensor A : a \tensor b \mapsto (-1)^{\deg{a}\deg{b}} b \tensor a. $$

The graded modules together with graded maps of degree $0$ form the category $\grMod{\k}$ of graded modules. From now on we will simply refer to maps instead of graded maps. Together with the tensor product and the ground ring, $(\grMod{\k}, \tensor, \k)$ is a symmetric monoidal category (with the symmetry given by $\tau$). This now dictates the definition of a graded algebra.

\begin{definition}
	A \emph{graded algebra} consists of a graded module $A$ together with two maps of degree $0$:
	$$ \mu: A \tensor A \to A \quad\text{ and }\quad \eta: k \to A $$
	such that $\mu$ is associative and $\eta$ is a unit for $\mu$.

	A map between two graded algebra will be called a \emph{graded algebra map} if the map is compatible with the multiplication and unit. Such a map is necessarily of degree $0$.
\end{definition}

Again these objects and maps form a category, denoted as $\grAlg{\k}$. We will denote multiplication by a dot or juxtaposition, instead of explicitly mentioning $\mu$.

\begin{definition}
	A graded algebra $A$ is \emph{commutative} if for all $x, y \in A$
	$$ xy = (-1)^{\deg{x}\deg{y}} yx. $$
\end{definition}


\subsection{Differential graded algebra}

\begin{definition}
	A \emph{differential graded module} $(M, d)$ is a graded module $M$ together with a map $d: M \to M$ of degree $-1$, called a \emph{differential}, such that $dd = 0$. A map $f: M \to N$ is a \emph{chain map} if it is compatible with the differential, i.e. $d_N f = f d_M$.
\end{definition}

A differential graded module $(M, d)$ with $M_i = 0$ for all $i < 0$ is a \emph{chain complex}. A differential graded module $(M, d)$ with $M_i = 0$ for all $i > 0$ is a \emph{cochain complex}. It will be convenient to define $M^i = M_{-i}$ in the latter case, so that $M = \bigoplus_{n \in \N} M^i$ and $d$ is a map of \emph{upper degree} $+1$.

\begin{definition}
	Let $(M, d_M)$ and $(N, d_N)$ be two differential graded modules, their tensor product $M \tensor N$ is a differential graded module with the differential given by:
	$$ d_{M \tensor N} = d_M \tensor \id_N + \id_M \tensor d_N. $$
\end{definition}

\todo{Prove that this is in fact a differential?}

Finally we come to the definition of a differential graded algebra. This will be a graded algebra with a differential. Of course we want this to be compatible with the algebra structure, or stated differently: we want $\mu$ and $\eta$ to be chain maps.

\begin{definition}
	A \emph{differential graded algebra (DGA)} is a graded algebra $A$ together with an differential $d$ such that in addition the \emph{Leibniz rule} holds:
	$$ d(xy) = d(x) y + (-1)^{\deg{x}} x d(y) \quad\text{ for all } x, y \in A. $$
\end{definition}

\todo{Define the notion of derivation?}

It is not hard to see that this definition precisely defines the monoidal objects in the category of differential graded modules. The category of DGAs will be denoted by $\DGA_\k$, the category of commutative DGAs (CDGAs) will be denoted by $\CDGA_\k$. If no confusion can arise, the ground ring $\k$ will be suppressed in this notation.

Let $M$ be a DGA, just as before $M$ is called a \emph{chain algebras} if $M_i = 0$ for $i < 0$. Similarly if $M^i = 0$ for all $i < 0$, then $M$ is a \emph{cochain algebra}.

\todo{The notation $\CDGA$ seem to refer to cochain algebras in literature and not arbitrary CDGAs.}

\subsection{Homology}

Whenever we have a differential graded module we have $d \circ d = 0$, or put in other words: the image of $d$ is a submodule of the kernel of $d$. The quotient of the two graded modules will be of interest.

\begin{definition}
	Given a differential graded modules $(M, d)$ we define the \emph{homology} of $M$ as: $H(M, d) = \ker(d) / \im(d)$.
	It is naturally graded as follows:
	$$ H(M, d)_i = H_i(M, d) = \ker(\restr{d}{M_i}) / d(M_{i+1}). $$
	If $d$ has degree $+1$ we define the \emph{cohomology} as:
	$$ H(M, d)^i = H^i(M, d) = \ker(\restr{d}{M^i}) / d(M^{i-1}). $$
\end{definition}

For differential graded algebras we can consider the (co)homology by forgetting the multiplicative structure. However this multiplication will actually pass to (co)homology:

\begin{lemma}
	Let $(A, d)$ be a differential graded algebra. The kernel $\ker(d)$ is a subalgebra of $A$ and the image $d(A)$ is an ideal, so that the quotient
	$$ H(A) = \ker(d) / \im(d) $$
	is a graded algebra.
\end{lemma}
\begin{proof}
	\todo{Maybe just state this?}
\end{proof}

\TODO{Discuss:
\titem The Künneth theorem (especially in the case of fields)
\titem The tensor algebra $T : Ch^\ast(\Q) \to \DGA_\Q$ and free cdga $\Lambda : Ch^\ast(\Q) \to \CDGA_\Q$
\titem Coalgebras and Hopfalgebras?
\titem Define reduced/connected differential graded things
\titem Singular (co)homology as a quick example?
}