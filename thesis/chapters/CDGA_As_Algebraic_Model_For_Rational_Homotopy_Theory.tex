
\Chapter{Homotopy Theory For cdga's}{HomotopyTheoryCDGA}

Recall that a cdga $A$ is a commutative differential graded algebra, meaning that
\begin{itemize}
	\item it has a grading: $A = \bigoplus_{n\in\N} A^n$,
	\item it has a differential: $d: A \to A$ with $d^2 = 0$,
	\item it has an associative and unital multiplication: $\mu: A \tensor A \to A$ and
	\item it is commutative: $x y = (-1)^{\deg{x}\cdot\deg{y}} y x$.
\end{itemize}
And all of the above structure is compatible with each other (e.g. the differential is a derivation of degree $1$, the maps are graded, \dots). We have a left adjoint $\Lambda$ to the forgetful functor $U$ which assigns the free graded commutative algebras $\Lambda V$ to a graded module $V$. This extends to an adjunction (also called $\Lambda$ and $U$) between commutative differential graded algebras and differential graded modules.

In homological algebra we are especially interested in \emph{quasi isomorphisms}, i.e. maps $f: A \to B$ inducing an isomorphism on cohomology: $H(f): HA \iso HB$. This notions makes sense for any object with a differential.

We furthermore have the following categorical properties of cdga's:
\begin{itemize}
	\item The finite coproduct in $\CDGA_\k$ is the (graded) tensor product.
	\item The finite product in $\CDGA_\k$ is the cartesian product (with pointwise operations).
	\item The equalizer (resp. coequalizer) of $f$ and $g$ is given by the kernel (resp. cokernel) of $f - g$. Together with the (co)products this defines pullbacks and pushouts.
	\item $\k$ and $0$ are the initial and final object.
\end{itemize}

In this chapter the ring $\k$ is assumed to be a field of characteristic zero. In particular the modules are vector spaces.

\todo{augmentations?}

\section{Cochain models for the $n$-disk and $n$-sphere}

We will first define some basic cochain complexes which model the $n$-disk and $n$-sphere. $D(n)$ is the cochain complex generated by one element $b \in D(n)^n$ and its differential $c = d(b) \in D(n)^{n+1}$. On the other hand we define $S(n)$ to be the cochain complex generated by one element $a \in S(n)^n$ with trivial differential (i.e. $d a = 0$). In other words:
$$ D(n) = ... \to 0 \to \k \to \k \to 0 \to ... $$
$$ S(n) = ... \to 0 \to \k \to 0 \to 0 \to ... $$

Note that $D(n)$ is acyclic for all $n$, or put in different words: $j_n : 0 \to D(n)$ induces an isomorphism in cohomology. The sphere $S(n)$ has exactly one non-trivial cohomology group \linebreak $H^n(S(n)) = \k \cdot [a]$. There is an injective function $i_n : S(n+1) \to D(n)$, sending $a$ to $c$. The maps $j_n$ and $i_n$ play the following important role in the model structure of cochain complexes, where weak equivalences are quasi isomorphisms, fibrations are degreewise surjective and cofibrations are degreewise injective for positive degrees \cite[Example 1.6]{goerss2}.

\begin{claim}
	The set $I = \{i_n : S(n+1) \to D(n) \I n \in \N\}$ generates all cofibrations and the set $J = \{j_n : 0 \to D(n) \I n \in \N\}$ generates all trivial cofibrations.
\end{claim}

As we do not directly need this claim, we omit the proof. However, in the next section we will prove a similar result for cdga's in detail.

$S(n)$ plays a another special role: maps from $S(n)$ to some cochain complex $X$ correspond directly to elements in the kernel of $\restr{d}{X^n}$. Any such map is null-homotopic precisely when the corresponding elements in the kernel is a coboundary. So there is a natural isomorphism: $\Hom(S(n), X) / {\simeq} \iso H^n(X)$.

By using the free cdga functor we can turn these cochain complexes into cdga's $\Lambda D(n)$ and $\Lambda S(n) $. So $\Lambda D(n)$ consists of linear combinations of $b^k$ and $c b^k$ when $n$ is even, and it consists of linear combinations of $c^k b$ and $c^k$ when $n$ is odd. In both cases we can compute the differentials using the Leibniz rule:

\[\xymatrix @R=0cm{
	d(b^k) = k \cdot c b^{k-1}   &   d(c^k b) = c^{k+1}\\
	d(c b^k) = 0   &   d(c^k) = 0
	}
\]

Those cocycles are in fact coboundaries (using that $\k$ is a field of characteristic $0$):
$$ c b^k = \frac{1}{k} d(b^{k+1}) $$
$$ c^k = d(b c^{k-1}) $$

There are no additional cocycles in $\Lambda D(n)$ besides the constants and $c$. So we conclude that $\Lambda D(n)$ is acyclic as an augmented algebra. In other words $\Lambda(j_n): \k \to \Lambda D(n)$ is a quasi isomorphism.

The situation for $\Lambda S(n)$ is easier as it has only one generator (as algebra). For even $n$ this means it is given by polynomials in $a$. For odd $n$ it is an exterior algebra, meaning $a^2 = 0$. Again the sets $\Lambda(I) = \{ \Lambda(i_n) : \Lambda S(n+1) \to \Lambda D(n) \I n \in \N\}$ and $\Lambda(J) = \{ \Lambda(j_n) : \k \to \Lambda D(n) \I n \in \N\}$ play an important role.

\begin{theorem}
	The sets $\Lambda(I)$ and $\Lambda(J)$ generate a model structure on $\CDGA_\k$ where:
	\begin{itemize}
		\item weak equivalences are quasi isomorphisms,
		\item fibrations are (degree wise) surjective maps and
		\item cofibrations are maps with the left lifting property against trivial fibrations.
	\end{itemize}
\end{theorem}

We will prove this theorem in the next section. Note that the functors $\Lambda$ and $U$ thus form a Quillen pair with this model structure.

\subsection{Why we need $\Char{\k} = 0$ for algebras}
The above Quillen pair $(\Lambda, U)$ fails to be a Quillen pair if \linebreak $\Char{\k} = p \neq 0$. We will show this by proving that the maps $\Lambda(j_n)$ are not weak equivalences for even $n$. Consider $b^p \in \Lambda D(n)$, then by the Leibniz rule:
$$ d(b^p) = p \cdot c b^{p-1} = 0. $$
So $b^p$ is a cocycle. Now assume $b^p = d x$ for some $x$ of degree $p n - 1$, then $x$ contains a factor $c$ for degree reasons. By the calculations above we see that any element containing $c$ has a trivial differential or has a factor $c$ in its differential, contradicting $b^p =  d x$. So this cocycle is not a coboundary and $\Lambda D(n)$ is not acyclic.


\section{The Quillen model structure on \titleCDGA}

\section{Model structure on \texorpdfstring{$\CDGA_\k$}{CDGA}}
\label{sec:model-of-cdga}

\TODO{First discuss the model structure on (co)chain complexes. Then discuss that we want the adjunction $(\Lambda, U)$ to be a Quillen pair. Then state that (co)chain complexes are cofib. generated, so we can cofib. generate CDGAs.}

In this section we will define a model structure on CDGAs over a field $\k$ of characteristic zero\todo{Can $\k$ be a c. ring here?}, where the weak equivalences are quasi isomorphisms and fibrations are surjective maps. The cofibrations are defined to be the maps with a left lifting property with respect to trivial fibrations.

\begin{proposition}
	There is a model structure on $\CDGA_\k$ where $f: A \to B$ is
	\begin{itemize}
		\item a \emph{weak equivalence} if $f$ is a quasi isomorphism,
		\item a \emph{fibration} if $f$ is an surjective and
		\item a \emph{cofibration} if $f$ has the LLP w.r.t. trivial fibrations
	\end{itemize}
\end{proposition}

We will prove the different axioms in the following lemmas. First observe that the classes as defined above are indeed closed under multiplication and contain all isomorphisms.

Note that with these classes, every cdga is a fibrant object.

\begin{lemma}
	[MC1] The category has all finite limits and colimits.
\end{lemma}
\begin{proof}
	As discussed earlier \todo{really discuss this somewhere} products are given by direct sums and equalizers are kernels. Furthermore the coproducts are tensor products and coequalizers are quotients.
\end{proof}

\begin{lemma}
	[MC2] The \emph{2-out-of-3} property for quasi isomorphisms.
\end{lemma}
\begin{proof}
	Let $f$ and $g$ be two maps such that two out of $f$, $g$ and $fg$ are weak equivalences. This means that two out of $H(f)$, $H(g)$ and $H(f)H(g)$ are isomorphisms. The \emph{2-out-of-3} property holds for isomorphisms, proving the statement.
\end{proof}

\begin{lemma}
	[MC3] All three classes are closed under retracts
\end{lemma}
\begin{proof}
	\todo{Make some diagrams and write it out}
\end{proof}

Next we will prove the factorization property [MC5]. We will do this by Quillen's small object argument. When proved, we get an easy way to prove the missing lifting property of [MC4]. For the Quillen's small object argument we use classes of generating cofibrations.

\begin{definition}
	Define the following objects and sets of maps:
	\begin{itemize}
		\item $S(n)$ is the CDGA generated by one element $a$ of degree $n$ such that $da = 0$.
		\item $T(n)$ is the CDGA generated by two element $b$ and $c$ of degree $n$ and $n+1$ respectively, such that $db = c$ (and necessarily $dc = 0$).
		\item $I = \{ i_n: \k \to T(n) \I n \in \N \}$ is the set of units of $T(n)$.
		\item $J = \{ j_n: S(n+1) \to T(n) \I n \in \N \}$ is the set of inclusions $j_n$ defined by $j_n(a) = b$.
	\end{itemize}
\end{definition}

\begin{lemma}
	The maps $i_n$ are trivial cofibrations and the maps $j_n$ are cofibrations.
\end{lemma}
\begin{proof}
	Since $H(T(n)) = \k$ \todo{Note that this only hold when characteristic = 0} we see that indeed $H(i_n)$ is an isomorphism. For the lifting property of $i_n$ and $j_n$ simply use surjectivity of the fibrations. \todo{give a bit more detail}
\end{proof}

\begin{lemma}
	The class of (trivial) cofibrations is saturated.
\end{lemma}
\begin{proof}
	\todo{prove this}
\end{proof}

As a consequence of the above two lemmas, the class generated by $I$ is contained in the class of trivial cofibrations. Similarly the class generated by $J$ is contained in the class of cofibrations. We also have a similar lemma about (trivial) fibrations.

\begin{lemma}
	If $p: X \to Y$ has the RLP w.r.t. $I$ then $p$ is a fibration.
\end{lemma}
\begin{proof}
	Easy\todo{Define a lift}.
\end{proof}

\begin{lemma}
	If $p: X \to Y$ has the RLP w.r.t. $J$ then $p$ is a trivial fibration.
\end{lemma}
\begin{proof}
	As $p$ has the RLP w.r.t. $J$, it also has the RLP w.r.t. $I$. From the previous lemma it follows that $p$ is a fibration. To show that $p$ is a weak equivalence ... \todo{write out}
\end{proof}

We can use Quillen's small object argument with these sets. The argument directly proves the following lemma. Together with the above lemmas this translates to the required factorization.

\begin{lemma}
	A map $f: A \to X$ can be factorized as $f = pi$ where $i$ is in the class generated by $I$ and $p$ has the RLP w.r.t. $I$.
\end{lemma}
\begin{proof}
	Quillen's small object argument. \todo{small = finitely generated?}
\end{proof}

\begin{corollary}
	[MC5a] A map $f: A \to X$ can be factorized as $f = pi$ where $i$ is a trivial cofibration and $p$ a fibration.
\end{corollary}

The previous factorization can also be described explicitly as seen in \cite{bousfield}. Let $f: A \to X$ be a map, define $E = A \tensor \bigtensor_{x \in X}T(\deg{x})$. Then $f$ factors as:
$$ A \tot{i} E \tot{p} X, $$
where $i$ is the obvious inclusion $i(a) = a \tensor 1$ and $p$ maps (products of) generators $a \tensor b_x$ with $b_x \in T(\deg{x})$ to $f(a) \cdot x \in X$.

\begin{lemma}
	A map $f: A \to X$ can be factorized as $f = pi$ where $i$ is in the class generated by $J$ and $p$ has the RLP w.r.t. $J$.
\end{lemma}
\begin{proof}
	Quillen's small object argument.
\end{proof}

\begin{corollary}
	[MC5b] A map $f: A \to X$ can be factorized as $f = pi$ where $i$ is a cofibration and $p$ a trivial fibration.
\end{corollary}


\subsection{Homotopy relation on \texorpdfstring{$\CDGA_\k$}{CDGA}}
Although the abstract theory of model categories gives us tools to construct a homotopy relation (\DefinitionRef{homotopy}), it is useful to have a concrete notion of homotopic maps.

Consider the free cdga on one generator $\Lambda(t, dt)$, this can be thought of as the (dual) unit interval. Indeed there is an isomorphism $\Lambda(t, dt) \iso \Apl_1$ and so we have maps for the two endpoint: $d_0, d_1: \Lambda(t, dt) \to \k \iso \Apl_0$. Given a cdga $X$ we will consider $d_0, d_1: \Lambda(t, dt) \tensor X \to \k \tensor X \iso X$.

\Definition{cdga_homotopy}{
	We call $f, g: A \to X$ homotopic ($f \simeq g$) if there is a map
	$$ h: A \to \Lambda(t, dt) \tensor X, $$
	such that $d_0 h = g$ and $d_1 h = f$.
}

In terms of model categories, such a homotopy is a right homotopy and the object $\Lambda(t, dt) \tensor X$ is a path object for $X$. We can easily see that it is a very good path object, first note that $\Lambda(t, dt) \tensor X \to X \oplus X$ is surjective (for $(x, y) \in X \oplus X$ take $t \tensor x + 1 \tensor y$), secondly $\Apl_0 \to \Apl_1$ is a cofibration and so is $X \to \Lambda(t, dt) \tensor X$.

Clearly we have that $f \simeq g$ implies $f \simeq^r g$ (see \DefinitionRef{right_homotopy}), however the converse need not be true.

\Lemma{cdga_homotopy}{
	If $A$ is a cofibrant cdga and $f \simeq^r g: A \to X$, then $f \simeq g$ in the above sense.
}
\Proof{
	Because $A$ is cofibrant, there is a very good homotopy $H$. Consider a lifting problem to construct a map $Path_X \to \Lambda(t, dt) \tensor X$.
}

\Corollary{cdga_homotopy_eqrel}{
	For cofibrant $A$, $\simeq$ defines a equivalence relation.
}
\Definition{cdga_homotopy_classes}{
	For cofibrant $A$ define the set of equivalence classes as:
	$$ [A, X] = \Hom_{\CDGA_\k}(A, X) / \simeq. $$
}

The results from model categories immediately imply the following results.
\Corollary{cdga_homotopy_properties}{
	Let $A$ be cofibrant.
	\begin{itemize}
		\item Let $i: A \to B$ be a trivial cofibration, then the induced map $i^\ast: [B, X] \to [A, X]$ is a bijection.
		\item Let $p: X \to Y$ be a trivial fibration, then the induced map $p_\ast: [A, X] \to [A, Y]$ is a bijection.
		\item Let $A$ and $X$ both be cofibrant, then $f: A \we X$ is a weak equivalence if and only if $f$ is a strong homotopy equivalence. Moreover, the two induced maps are bijections:
		$$ f_\ast: [Z, A] \tot{\iso} [Z, X], $$
		$$ f^\ast: [X, Z] \tot{\iso} [A, X]. $$
	\end{itemize}
}

\Lemma{cdga_homotopy_homology}{
	Let $f, g: A \to X$ be two homotopic maps, then $H(f) = H(g): HA \to HX$.
}
\Proof{
	We only need to consider $H(d_0)$ and $H(d_1)$.
}

\section{Homotopy relations on \titleCDGA}

Although the abstract theory of model categories gives us tools to construct a homotopy relation (\DefinitionRef{homotopy}), it is useful to have a concrete notion of homotopic maps.

Consider the free cdga on one generator $\Lambda(t, dt)$, where $\deg{t} = 0$, this can be thought of as the (dual) unit interval with endpoints $1$ and $t$. Notice that this cdga is isomorphic to $\Lambda(D(0))$ as defined in the previous section. We define two \emph{endpoint maps} as follows:
$$ d_0, d_1 : \Lambda(t, dt) \to \k $$
$$ d_0(t) = 1, \qquad d_1(t) = 0, $$
this extends linearly and multiplicatively. Note that it follows that we have $d_0(1-t) = 0$ and $d_1(1-t) = 1$. These two functions extend to tensor products as $d_0, d_1: \Lambda(t, dt) \tensor X \to \k \tensor X \tot{\iso} X$.

\Definition{cdga_homotopy}{
	We call $f, g: A \to X$ homotopic ($f \simeq g$) if there is a map
	$$ h: A \to \Lambda(t, dt) \tensor X, $$
	such that $d_0 h = g$ and $d_1 h = f$.
}

In terms of model categories, such a homotopy is a right homotopy and the object $\Lambda(t, dt) \tensor X$ is a path object for $X$. We can easily see that it is a very good path object\todo{Refereer}. First note that $\Lambda(t, dt) \tensor X \tot{(d_0, d_1)} X \oplus X$ is surjective (for $(x, y) \in X \oplus X$ take $t \tensor x + (1-t) \tensor y$). Secondly we note that $\Lambda(t, dt) = \Lambda(D(0))$ and hence $\k \to \Lambda(t, dt)$ is a cofibration, by \LemmaRef{model-cats-coproducts} we have that $X \to \Lambda(t, dt) \tensor X$ is a (necessarily trivial) cofibration.

Clearly we have that $f \simeq g$ implies $f \simeq^r g$ (see \DefinitionRef{right_homotopy}), however the converse need not be true.

\Lemma{cdga_homotopy}{
	If $A$ is a cofibrant cdga and $f \simeq^r g: A \to X$, then $f \simeq g$ in the above sense.
}
\Proof{
	Because $A$ is cofibrant, there is a very good homotopy $H$. Consider a lifting problem to construct a map $Path_X \to \Lambda(t, dt) \tensor X$.
}

\Corollary{cdga_homotopy_eqrel}{
	For cofibrant $A$, $\simeq$ defines a equivalence relation.
}
\Definition{cdga_homotopy_classes}{
	For cofibrant $A$ define the set of equivalence classes as:
	$$ [A, X] = \Hom_{\CDGA_\k}(A, X) / \simeq. $$
}

The results from model categories immediately imply the following results. Here we use Lemma \ref{lem:left_homotopy_properties}, \ref{lem:right_homotopy_properties} and \ref{lem:weak_strong_homotopy}.
\Corollary{cdga_homotopy_properties}{
	Let $A$ be cofibrant.
	\begin{itemize}
		\item Let $i: A \to B$ be a trivial cofibration, then the induced map $i^\ast: [B, X] \to [A, X]$ is a bijection.
		\item Let $p: X \to Y$ be a trivial fibration, then the induced map $p_\ast: [A, X] \to [A, Y]$ is a bijection.
		\item Let $A$ and $X$ both be cofibrant, then $f: A \we X$ is a weak equivalence if and only if $f$ is a strong homotopy equivalence. Moreover, the two induced maps are bijections:
		\begin{align*}
			f_\ast: [Z, A] &\tot{\iso} [Z, X], \\
			f^\ast: [X, Z] &\tot{\iso} [A, X].
		\end{align*}
	\end{itemize}
}
\Remark{cdga-weak-eq-bijection}{
	By \RemarkRef{cdga-mc5a-left-inverse} we can generalize the second item to arbitrary weak equivalences: If $A$ is cofibrant and $f : X \to Y$ a weak equivalence, then the induced map $f_\ast : [A, X] \to [A, Y]$ is a bijection, as seen from the following diagram:
	\[ \xymatrix{
		& [A, X \tensor C] \ar[dl]_{\overline{\phi}_\ast}^\iso \ar[dr]^{\psi_\ast}_\iso & \\
		[A, X] \ar[rr]^{f_\ast} & & [A, Y]
	}\]
}

\Lemma{cdga-homotopy-homology}{
	Let $f, g: A \to X$ be two homotopic maps, then $H(f) = H(g): HA \to HX$.
}
\Proof{
	Let $h$ be the homotopy such that $f = d_1 h$ and $g = d_0 h$. By the Künneth theorem we get the following commuting square for $i = 0, 1$:
	\[ \xymatrix{
		H(\Lambda(t, dt)) \tensor H(A) \ar[r]^-{d_i \tensor \id} \ar[d]^-{\iso} & \k \tensor H(A) \ar[d]^-{\iso} \\
		H(\Lambda(t, dt) \tensor A) \ar[r]^-{d_i} & H(\k \tensor A)
	} \]
	Now we know that $H(d_0) = H(d_1) : H(\Lambda(t, dt)) \to \k$ as $\Lambda(t, dt)$ is acyclic and the induced map sends $1$ to $1$. So the two bottom maps in the diagram are equal as well. Now we conclude $H(f) = H(d_1)H(h) = H(d_0)H(h) = H(g)$.
}


\section{Homotopy theory of augmented cdga's}

Recall that an augmented cdga is a cdga $A$ with an algebra map $A \tot{\counit} \k$ (this implies that $\counit \unit = \id$). This is precisely the dual notion of a pointed space. We will use the general fact that if $\cat{C}$ is a model category, then the over (resp. under) category $\cat{C} / A$ (resp. $A / \cat{C}$) for any object $A$ admit an induced model structure. In particular, the category of augmented cdga's (with augmentation preserving maps) has a model structure with the fibrations, cofibrations and weak equilavences as above.

Although the model structure is completely induced, it might still be fruitful to discuss the right notion of a homotopy for augmented cdga's. Consider the following pullback of cdga's:
\[ \xymatrix{
	\Lambda(t, dt) \overline{\tensor} A \ar[r] \xypb \ar[d] & \Lambda(t, dt) \tensor A \ar[d] \\
	\k \ar[r] & \k \tensor \Lambda(t, dt)
}\]
The pullback is the subspace of elements $x \tensor a$ in $\Lambda(t, dt) \tensor A$ such that $\counit(a) \cdot x \in \k$. Note that this construction is dual to a construction on topological spaces: in order to define a homotopy which is constant on the point $x_0$, we define the homotopy to be a map from a quotient ${X \times I} / {x_0 \times I}$.
\Definition{homotopy-augmented}{
	Two maps $f, g: A \to X$ between augmented cdga's are said to be \emph{homotopic} if there is a map
	$$h : A \to \Lambda(t, dt) \overline{\tensor} X$$
	such that $d_0 h = g$ and $d_1 h = f$.
}

In the next section homotopy groups of augmented cdga's will be defined. In order to define this we first need another tool.
\Definition{indecomposables}{
		Define the \Def{augmentation ideal} of $A$ as $\overline{A} = \ker \counit$. Define the \Def{cochain complex of indecomposables} of $A$ as $QA = \overline{A} / \overline{A} \cdot \overline{A}$.
} 

The first observation one should make is that $Q$ is a functor from algebras to modules (or differential algebras to differential modules) which is particularly nice for free algebras, as we have that $Q \Lambda V = V$ for any (differential) module $V$.

\todo{tensor}


\section{Homotopy groups of cdga's}

As the eventual goal is to compare the homotopy theory of spaces with the homotopy theory of cdga's, it is natural to investigate an analogue of homotopy groups in the category of cdga's. In topology we can only define homotopy groups on pointed spaces, dually we will consider augmented cdga's in this section. Recall that an augmented cdga is a cdga $A$ with an algebra map $A \tot{\counit} \k$ such that $\counit \unit = \id$.

\Definition{cdga-homotopy-groups}{
	Define the \Def{augmentation ideal} of $A$ as $\overline{A} = \ker \counit$. Define the \Def{cochain complex of indecomposables} of $A$ as $QA = \overline{A} / \overline{A} \cdot \overline{A}$.

	Now define the \Def{homotopy groups of a cdga} $A$ as
	$$ \pi^i(A) = H^i(QA). $$
}

This construction is functorial and, as the following lemma shows, homotopy invariant.

\Lemma{cdga-homotopic-maps-equal-pin}{
	Let $f: A \to B$ be a map of augmented cdga's. Then there is an functorial induced map on the homotopy groups. Moreover if $g: A \to B$ is homotopic to $f$, then the induced maps are equal:
	$$ f_\ast = g_\ast : \pi_\ast(A) \to \pi_\ast(B). $$
}
\Proof{
	Let $\phi: A \to B$ be a map of algebras. Then clearly we get an induced map $\overline{A} \to \overline{B}$ as $\phi$ preserves the augmentation. By composition we get a map $\phi': \overline{A} \to Q(B)$ for which we have $\phi'(xy) = \phi'(x)\phi'(y) = 0$. So it induces a map $Q(\phi): Q(A) \to Q(B)$. By functoriality of taking homology we get $f_\ast : \pi^n(A) \to \pi^n(B)$.

	Now if $f$ and $g$ are homotopic, then there is a homotopy $h: A \to \Lambda(t, dt) \tensor B$. By the Künneth theorem we have:
	$$ {d_0}_\ast = {d_1}_\ast : H(\Lambda(t, dt) \tensor Q(B)) \to H(Q(B)). $$
	This means that $f_\ast = {d_1}_\ast h_\ast = {d_0}_\ast h_\ast = g_\ast$. \todo{detail}
}

Consider the augmented cdga $V(n) = S(n) \oplus \k$, with trivial multiplication and where the term $\k$ is used for the unit and augmentation. This augmented cdga can be thought of as a specific model of the sphere. In particular the homotopy groups can be expressed as follows.

\Lemma{cdga-dual-homotopy-groups}{
	There is a natural bijection for any augmented cdga $A$
	$$ [A, V(n)] \tot{\iso} \Hom_\k(\pi^n(A), \k). $$
}
\Proof{
	Note that $Q(V(n))$ in degree $n$ is just $\k$ and $0$ in the other degrees, so its homotopy groups consists of a single $\k$ in degree $n$. This establishes the map:
	$$ \Phi: \Hom_\CDGA(A, V(n)) \to \Hom_\k(\pi^n(A), \k). $$

	Now by \LemmaRef{cdga-homotopic-maps-equal-pin} we get a map from the set of homotopy classes $[A, V(n)]$ instead of just maps. \todo{injective, surjective}
}

From now on the dual of a vector space will be denoted as $V^\ast = \Hom_\k(V, \k)$. So the above lemma states that there is a bijection $[A, V(n)] \iso \pi^n(A)^\ast$.

\todo{long exact sequence}



\Chapter{Polynomial Forms}{Adjunction}
\label{sec:cdga-of-polynomials}

\section{CDGA of Polynomials}

\section{CDGA of Polynomials}

\newcommand{\Apl}[0]{{A_{PL}}}

We will now give a cdga model for the $n$-simplex $\Delta^n$. This then allows for simplicial methods. In the following definition one should be reminded of the topological $n$-simplex defined as convex span.

\begin{definition}
	For all $n \in \N$ define the following cdga:
	$$ (\Apl)_n = \frac{\Lambda(x_0, \ldots, x_n, dx_0, \ldots, dx_n)}{(\sum_{i=0}^n x_i - 1, \sum_{i=0}^n dx_i)} $$
	So it is the free cdga with $n+1$ generators and their differentials such that $\sum_{i=0}^n x_i = 1$ and in order to be well behaved $\sum_{i=0}^n dx_i = 0$.
\end{definition}

Note that the inclusion $\Lambda(x_1, \ldots, x_n, dx_1, \ldots, dx_n) \to \Apl_n$ is an isomorphism of cdga's. So $\Apl_n$ is free and (algebra) maps from it are determined by their images on $x_i$ for $i = 1, \ldots, n$ (also note that this determines the images for $dx_i$). This fact will be used throughout.

These cdga's will assemble into a simplicial cdga when we define the face and degeneracy maps as follows ($j = 1, \ldots, n$):

$$ d_i(x_j) = \begin{cases}
	x_{j-1}, &\text{ if } i < j \\
	0,       &\text{ if } i = j \\
	x_j,     &\text{ if } i > j
\end{cases} \qquad d_i : \Apl_n \to \Apl_{n-1} $$
$$ s_i(x_j) = \begin{cases}
	x_{j+1},       &\text{ if } i < j \\
	x_j + x_{j+1}, &\text{ if } i = j \\
	x_j,           &\text{ if } i > j	
\end{cases} \qquad s_i : \Apl_n \to \Apl_{n+1} $$

One can check that $\Apl \in \simplicial{\CDGA_\k}$. We will denote the subspace of homogeneous elements of degree $k$ as $\Apl^k \in \simplicial{\Mod{\k}}$, this is indeed a simplicial $\k$-module as the maps $d_i$ and $s_i$ are graded maps of degree $0$.

\begin{lemma}
	$\Apl^k$ is contractible.
\end{lemma}
\begin{proof}
	We will prove this by defining an extra degeneracy $s: \Apl_n \to \Apl_{n+1}$. Define for $i = 1, \ldots, n$:
	\begin{align*}
		s(1) &= (1-x_0)^2 \\
		s(x_i) &= (1-x_0) \cdot x_{i+1}
	\end{align*}
	Extend on the differentials and multiplicatively on $\Apl_n$. As $s(1) \neq 1$ this map is not an algebra map, however it well-defined as a map of cochain complexes. In particular when restricted to degree $k$ we get a linear map:
	$$ s: \Apl^k_n \to \Apl^k_{n+1}. $$
	Proving the necessary properties of an extra degeneracy is fairly easy. For $n \geq 1$ we get (on generators):
	\begin{align*}
		d_0 s(1)   &= d_0 (1 - x_0)^2 = (1 - 0) \cdot (1 - 0) = 1 \\
		d_0 s(x_i) &= d_0((1-x_0)x_{i+1}) = d_0(1-x_0) \cdot x_i  \\
		           &= (1-0) \cdot x_i = x_i
	\end{align*}
	So $d_0 s = \id$.
	\begin{align*}
		d_{i+1} s(1) &= d_{i+1} (1 - x_0)^2 = d_{i+1} (\sum_{j=1}^n x_j)^2 \\
		             &= (\sum_{j=1}^{n-1} x_j)^2 = (1-x_0)^2 = s d_i(1)    \\
		d_{i+1} s(x_j) &= d_{i+1}(1-x_0) d_{i+1}(x_j) = (1-x_0) d_i(x_{j+1}) = s d_i (x_j)
	\end{align*}
	So $d_{i+1} s = s d_i$. Similarly $s_{i+1} s = s s_i$. And finally for $n=0$ we have $d_1 s = 0$.

	So we have an extra degeneracy $s: \Apl^k \to \Apl^k$, and hence (see for example \cite{goerss}) we have that $\Apl^k$ is contractible. As a consequence $\Apl \to \ast$ is a weak equivalence.
\end{proof}

\begin{lemma}
	$\Apl_n^k$ is a Kan complex.
\end{lemma}
\begin{proof}
	By the simple fact that $\Apl_n^k$ is a simplicial group, it is a Kan complex \cite{goerss}.
\end{proof}

\begin{corollary}
	$\Apl^k \to \ast$ is a trivial fibration in the standard model structure on $\sSet$.
\end{corollary}





\section{Polynomial Forms on a Space}
\label{sec:polynomial-forms}

\subsection{Polynomial Forms}

There is a general way to construct functors from $\sSet$ whenever we have some simplicial object. In our case we have the simplicial cdga $\Apl$ (which is nothing more than a functor $\opCat{\DELTA} \to \CDGA$) and we want to extend to a contravariant functor $\sSet \to \CDGA_\k$. This will be done via Kan extensions.

Given a category $\cat{C}$ and a functor $F: \DELTA \to \cat{C}$, then define the following on objects:
\begin{align*}
	F_!(X)      &= \colim_{\Delta[n] \to X} F[n] &\quad X \in \sSet \\
	F^\ast(C)_n &= \Hom_{\cat{C}}(F[n], Y)       &\quad C \in \cat{C}
\end{align*}
A simplicial map $X \to Y$ induces a map of the diagrams of which we take colimits. Applying $F$ on these diagrams, make it clear that $F_!$ is functorial.




\chapter{Minimal models}
\label{sec:minimal-models}

In this section we will discuss the so called minimal models. These are cdga's with the property that a quasi isomorphism between them is an actual isomorphism.

\begin{definition}
	A cdga $(A, d)$ is a \emph{Sullivan algebra} if
	\begin{itemize}
		\item $A = \Lambda V$ is free as a commutative graded algebra, and
		\item $V$ has a filtration
		$$ 0 = V(-1) \subset V(0) \subset V(1) \subset \cdots \subset \bigcup_{k \in \N} V(k) = V, $$
		such that $d(V(k)) \subset \Lambda V(k-1)$.
	\end{itemize}

	An cdga $(A, d)$ is a \emph{minimal (Sullivan) algebra} if in addition
	\begin{itemize}
		\item $d$ is decomposable, i.e. $\im(d) \subset \Lambda^{\geq 2}V$.
	\end{itemize}
\end{definition}

\begin{definition}
	Let $(A, d)$ be any cdga. A \emph{(minimal) Sullivan model} is a (minimal) Sullivan algebra $(M, d)$ with a weak equivalence:
	$$ (M, d) \we (A, d). $$
\end{definition}

The requirement that there exists a filtration can be replaced by a stronger statement.

\begin{lemma}
	Let $(A, d)$ be a cdga which is $1$-reduced, quasi-free and with a decomposable differential. Then $(A, d)$ is a minimal algebra.
\end{lemma}
\begin{proof}
	Let $V$ generate $A$. Take $V(n) = \bigoplus_{k=0}^n V^k$ (note that $V^0 = V^1 = 0$). Since $d$ is decomposable we see that for $v \in V^n$: $d(v) = x \cdot y$ for some $x, y \in A$. Assuming $dv$ to be non-zero we can compute the degrees:
	$$ \deg{x} + \deg{y} = \deg{xy} = \deg{dv} = \deg{v} + 1 = n + 1. $$
	As $A$ is $1$-reduced we have $\deg{x}, \deg{y} \geq 2$ and so by the above $\deg{x}, \deg{y} \leq n-1$. Conclude that $d(V(k)) \subset \Lambda(V(n-1))$.
\end{proof}


\section{Existence}

\begin{theorem}
	Let $(A, d)$ be an $0$-connected cdga, then it has a Sullivan model $(\Lambda V, d)$. Furthermore if $(A, d)$ is $r$-connected, $V^i = 0$ for all $i \leq r$.
\end{theorem}
\begin{proof}
	Start by setting $V(0) = H^{\geq 1}(A)$ and $d = 0$. This extends to a morphism $m_0 : (\Lambda V(0), 0) \to (A, d)$.

	Note that the freeness introduces products such that the map $H(m_0) : H(\Lambda V(0)) \to H(A)$ is not an isomorphism. We will ``kill'' these defects inductively.

	Suppose $V(k)$ and $m_k$ are constructed. Consider the defect $\ker H(m_k)$ and let $\{[z_\alpha]\}_{\alpha \in A}$ be a basis for it. Define $V_{k+1} = \bigoplus_{\alpha \in A} \k \cdot v_\alpha$ with the degrees $\deg{v_\alpha} = \deg{z_\alpha}-1$.

	Now extend the differential by defining $d(v_\alpha) = z_\alpha$. This step kills the defect, but also introduces new defects which will be killed later. Notice that $z_\alpha$ is a cocycle and hence $d^2 v_\alpha = 0$.

	Since $[z_\alpha]$ is in the kernel of $H(m_k)$ we see that $m_k z_\alpha = d a_\alpha$ for some $a_\alpha$. Extend $m_k$ to $m_{k+1}$ by defining $m_{k+1}(v_\alpha) = a_\alpha$. Notice that $m_{k+1} d v_\alpha = m_{k+1} z_\alpha = d a_\alpha = d m_{k+1} v_\alpha$.

	Now take $V(k+1) = V(k) \oplus V_{k+1}$. We already proved that $d$ is indeed a differential, and that $m_{k+1}$ is indeed a chain map.

	Complete the construction by taking the union: $V = \bigcup_k V(k)$. Clearly $H(m)$ is surjective, this was establsihed in the first step. Now if $H(m)[z] = 0$, then we know $z \in \Lambda V(k)$ for some stage $k$ and hence by construction is was killed, i.e. $[z] = 0$. So we see that $m$ is a quasi isomorphism and by construction $(\Lambda V, d)$ is a sullivan algebra.

	\todo{minimality for $1$-connected}
\end{proof}


\section{Uniqueness}

Before we state the uniqueness theorem we need some more properties of minimal models.

\begin{lemma}
	Sullivan algebras are cofibrant.
\end{lemma}
\begin{proof}
	Consider the following lifting problem, where $\Lambda V$ is a Sullivan algebra.

	\cimage[scale=0.5]{Sullivan_Lifting}

	By the left adjointness of $\Lambda$ we only have to specify a map $\phi: V \to X$ such that $p \circ \phi = g$. We will do this by induction.
	\begin{itemize}
		\item Suppose $\{v_\alpha\}$ is a basis for $V(0)$. Define $V(0) \to X$ by choosing preimages $x_\alpha$ such that $p(x_\alpha) = g(v_\alpha)$ ($p$ is surjective). Define $\phi(v_\alpha) = x_\alpha$.
		\item Suppose $\phi$ has been defined on $V(n)$. Write $V(n+1) = V(n) \oplus V'$ and let $\{v_\alpha\}$ be a basis for $V'$. Then $dv_\alpha \in \Lambda V(n)$, hence $\phi(dv_\alpha)$ is defined and
		$$ d \phi d v_\alpha = \phi d^2 v_\alpha = 0 $$
		$$ p \phi d v_\alpha = g d v_\alpha = d g v_\alpha. $$
		Now $\phi d v_\alpha$ is a cycle and $p \phi d v_\alpha$ is a boundary of $g v_\alpha$. By the following lemma there is a $x_\alpha \in X$ such that $d x_\alpha = \phi d v_\alpha$ and $p x_\alpha = g v_\alpha$. The former property proves that $\phi$ is a chain map, the latter proves the needed commutativity $p \circ \phi = g$.
	\end{itemize}
\end{proof}

\begin{lemma}
	Let $p: X \to Y$ be a trivial fibration, $x \in X$ a cycle, $p(x) \in Y$ a boundary of $y' \in Y$. Then there is a $x' \in X$ such that
	$$ dx' = x \quad\text{ and }\quad px' = y'. $$
\end{lemma}
\begin{proof}
	We have $p^\ast [x] = [px] = 0$, since $p^\ast$ is injective we have $x = d \overline{x}$ for some $\overline{x} \in X$. Now $p \overline{x} = y' + db$ for some $b \in Y$. Choose $a \in X$ with $p a = b$, then define $x' = \overline{x} - da$. Now check the requirements: $p x' = p \overline{x} - p a = y'$ and $d x' = d \overline{x} - d d a = d \overline{x} = x$.
\end{proof}

\begin{lemma}
	Let $f: X \we Y$ be a weak equivalence between cdga's and $M$ a minimal model for $X$. Then $f$ induces an bijection:
	$$ f_\ast: [M, X] \tot{\iso} [M, Y]. $$
\end{lemma}
\begin{proof}
	If $f$ is surjective this follows from the fact that $M$ is cofibrant and $f$ being a trivial fibration (see \cite[lemma 4.9]{dwyer}).

	In general we can construct a cdga $Z$ and trivial fibrations $X \to Z$ and $Y \to Z$ inducing bijections:
	$$ [M, X] \tot{\iso} [M, Z] \toti{\iso} [M, Y], $$
	compatible with $f_\ast$. \cite[Proposition 12.9]{felix}.
\end{proof}

\begin{lemma}
	Let $\phi: (M, d) \we (M', d')$ be a weak equivalence between minimal algebras. Then $\phi$ is an isomorphism.
\end{lemma}
\begin{proof}
	Let $M$ and $M'$ be generated by $V$ and $V'$. Then $\phi$ induces a weak equivalence on the linear part $\phi_0: V \we V'$ \cite[Theorem 1.5.2]{loday}. Since the differentials are decomposable, their linear part vanishes. So we see that $\phi_0: (V, 0) \tot{\iso} (V', 0)$ is an isomorphism.
	Conclude that $\phi = \Lambda \phi_0$ is an isomorphism.
\end{proof}

\begin{theorem}
	Let $m: (M, d) \we (A, d)$ and $m': (M', d') \we (A, d)$ be two minimal models. Then there is an isomorphism $\phi (M, d) \tot{\iso} (M', d')$ such that $m' \circ \phi \eq m$.
\end{theorem}
\begin{proof}
	By the previous lemmas we have $[M', M] \iso [M', A]$. By going from right to left we get a map $\phi: M' \to M$ such that $m' \circ \phi \eq m$. On homology we get $H(m') \circ H(\phi) = H(m)$, proving that (2-out-of-3) $\phi$ is a weak equivalence. The previous lemma states that $\phi$ is then an isomorphism.
\end{proof}


\section{The minimal model of the sphere}
We know from singular cohomology that the cohomology ring of a $n$-sphere is $\Z[X] / (X^2)$. This allows us to construct a minimal model for $S^n$.
\Definition{minimal-model-sphere}{
	Define $A(n)$ to be the cdga defined as
	$$ A(n) = \begin{cases}
		\Lambda(e) \quad \deg{e} = n \quad de = 0 \qquad &\text{ if $n$ is odd } \\
		\Lambda(e, f) \quad \deg{e} = n, \deg{f} = 2n-1 \quad df = e^2 \qquad &\text{ if $n$ is even }
	\end{cases}. $$
}

\todo{prove $HA(n) \tot{\iso} HA(S^n)$}


\chapter{The main equivalence}

In this section we aim to prove that the homotopy theory of rational spaces is the same as the homotopy theory of cdga's over $\Q$. Before we prove the equivalence, we will show that $A$ and $K$ form a Quillen pair. This already provides an adjunction between the homotopy categories. Besides the equivalence of the homotopy categories we will also investigate homotopy groups on a cdga directly. The homotopy groups of a space will be dual to the homotopy groups of the associated cdga.

We will prove that $A$ preserves cofibrations and trivial cofibrations. We only have to check this fact for the generating (trivial) cofibrations in $\sSet$. Note that the contravariance of $A$ means that a (trivial) cofibrations should be sent to a (trivial) fibration.

\begin{lemma}
	$A(i) : A(\Delta[n]) \to A(\del \Delta[n])$ is surjective.
\end{lemma}
\begin{proof}
	Let $\phi \in A(\del \Delta[n])$ be an element of degree $k$, hence it is a map $\del \Delta[n] \to \Apl^k$. We want to extend this to the whole simplex. By the fact that $\Apl^k$ is Kan and contractible we can find a lift $\overline{\phi}$ in the following diagram showing the surjectivity.
	\begin{displaymath}
		\xymatrix {
		\del \Delta[n] \ar[r]^\phi \arcof[d]^i & \Apl^k \\
		\Delta[n] \ar@{-->}[ru]_{\overline{\phi}}
		}
	\end{displaymath}
\end{proof}

\begin{lemma}
	$A(j) : A(\Delta[n]) \to A(\Lambda^k_n)$ is surjective and a quasi isomorphism.
\end{lemma}
\begin{proof}
	As above we get surjectivity from the Kan condition. To prove that $A(j)$ is a quasi isomorphism we pass to the singular cochain complex and use that $C^\ast(j) : C^\ast(\Delta[n]) \we C^\ast(\Lambda^n_k)$ is a quasi isomorphism. Consider the following diagram and conclude that $A(j)$ is surjective and a quasi isomorphism.
	\begin{displaymath}
		\xymatrix {
		A(\Delta[n]) \ar[r]^{A(j)} \arwe[d]^\oint & A(\Lambda^k_n) \arwe[d]^\oint \\
		C^\ast(\Delta[n]) \ar[r]^{C^\ast(j)} & C^\ast(\Lambda^k_n)
		}
	\end{displaymath}
\end{proof}

Since $A$ is a left adjoint, it preserves all colimits and by functoriality it preserves retracts. From this we can conclude the following corollary.

\begin{corollary}
	$A$ preserves all cofibrations and all trivial cofibrations and hence is a left Quillen functor.
\end{corollary}

\begin{corollary}
	$A$ and $K$ induce an adjunction on the homotopy categories:
	$$ LA : \Ho(\sSet) \leftadj \opCat{\Ho(\CDGA)} : RK. $$
\end{corollary}

The induced adjunction in the previous corollary is given by $LA(X) = A(X)$ for $X \in \sSet$ (note that every simplicial set is already cofibrant) and $RK(Y) = K(Y^{cof})$ for $Y \in \CDGA$. By the use of minimal models, and in particular the functor $M$. We get the following adjunction between $1$-connected objects:

\Corollary{minimal-model-adjunction}{
	There is an adjunction:
	$$ M : \Ho(\sSet_1) \leftadj \opCat{\Ho(\text{Minimal models}^1)} : RK, $$
	where $M$ is given by $M(X) = M(A(X))$ and $RK$ is given by $RK(Y) = K(Y)$ (because minimal models are always cofibrant).
}


\section{Homotopy groups of cdga's}
We are after an equivalence of homotopy categories, so it is natural to ask what the homotopy groups of $K(A)$ are for a cdga $A$. In order to do so, we will define homotopy groups of cdga's directly and compare the two notions.

Recall that an augmented cdga is a cdga $A$ with an algebra map $A \tot{\counit} \k$ such that $\counit \unit = \id$.

\Definition{cdga-homotopy-groups}{
	Define the \Def{augmentation ideal} of $A$ as $\overline{A} = \ker \counit$. Define the \Def{cochain complex of indecomposables} of $A$ as $QA = \overline{A} / \overline{A} \cdot \overline{A}$.

	Now define the \Def{homotopy groups of a cdga} $A$ as
	$$ \pi^i(A) = H^i(QA). $$
}

Note that for a free cdga $\Lambda C$ there is a natural augmentation and the chain complexes of indecomposables $Q \Lambda C$ is naturally isomorphic to $C$. Consider the augmented cdga $V(n) = D(n) \oplus \k$, with trivial multiplication and where the term $\k$ is used for the unit and augmentation. There is a weak equivalence $A(n) \to V(n)$ (recall \DefinitionRef{minimal-model-sphere}).

\Lemma{cdga-dual-homotopy-groups}{
	Let $A$ be an augmented cdga, then
	$$ [A, V(n)] \tot{\iso} \Hom_\k(\pi^n(A), \k). $$
}
\todo{prove}

We will denote the dual of a vector space as $V^\ast = \Hom_\k(V, \k)$.

\Theorem{cdga-dual-homotopy-groups}{
	Let $X$ be a cofibrant augmented cdga, then
	$$ \pi_n(KX) \iso \pi^n(X)^\ast. $$
}
\Proof{
	First note that $KX$ is a Kan complex (because it is a simplicial group). Using the homotopy adjunction and the lemma above we get:
	\begin{align*}
		\pi_n(KX) &= [S^n, KX]      \\
		          &\iso [X, A(S^n)] \\
		          &\iso [X, A(n)]   \\
		          &\iso [X, V(n)]   \\
		          &\iso \pi^n(X)^\ast
	\end{align*}
	\todo{Prove all isomorphisms.}
	\todo{Group structure?}
}

We get a particularly nice result for minimal cdga's, because the functor $Q$ is the left inverse of the functor $\Lambda$ and the differential is decomposable.

\Corollary{minimal-cdga-homotopy-groups}{
	For a minimal cdga $X = \Lambda V$ we get
	$$ \pi_n(KX) = {V^n}^\ast. $$
}

\Corollary{minimal-cdga-EM-space}{
	For a cdga with one generator $X = \Lambda(v)$ with $d v = 0$ and $\deg{v} = n$. We conclude that $KX$ is a $K(\k^\ast, n)$-space.
}


\section{Equivalence on rational spaces}
For the equivalence of rational spaces and cdga's we need that the unit and counit of the adjunction in \CorollaryRef{minimal-model-adjunction} are in fact weak equivalences for rational spaces. More formally: for any (automatically cofibrant) $X \in \sSet$ and any minimal model $A \in \CDGA$, both rational, $1$-connected and of finite type, the following two natural maps are weak equivalences:
\begin{align*}
	X &\to K(M(X)) \\
	A &\to M(K(A))
\end{align*}
where the first of the two maps is given by the composition $X \to K(A(X)) \tot{K(m_X)} K(M(X))$,
and the second map is obtained by the map $A \to A(K(A))$ and using the bijection from \LemmaRef{minimal-model-bijection}: $[A, A(K(A))] \iso [A, M(K(A))]$. By the 2-out-of-3 property the map $A \to M(K(A))$ is a weak equivalence if and only if the ordinary unit $A \to A(K(A))$ is a weak equivalence.

\Lemma{}{
	(Base case) Let $A = (\Lambda(v), 0)$ be a minimal model with one generator of degree $\deg{v} = n \geq 1$. Then $A \we A(K(A))$.
}
\Proof{
	By \CorollaryRef{minimal-cdga-EM-space} we know that $K(A)$ is an Eilenberg-MacLane space of type $K(\Q^\ast, n)$. The cohomology of an Eilenberg-MacLane space with coefficients in $\Q$ is known:
	$$ H^\ast(K(\Q^\ast, n); \Q) = \Q[x], $$
	that is, the free commutative graded algebra with one generator $x$. This can be calculated, for example, with spectral sequences \cite{griffiths}.

	Now choose a cycle $z \in A(K(\Q^\ast, n))$ representing the class $x$ and define a map $A \to A(K(A))$ by sending the generator $v$ to $z$. This induces an isomorphism on cohomology. So $A$ is the minimal model for $A(K(A))$.
}

\Lemma{}{
	(Induction step) Let $A$ be a cofibrant, connected algebra. Let $B$ be the pushout in the following square, where $m \geq 1$:
	\begin{displaymath}
		\xymatrix{
		S(m+1) \arcof[d] \ar[r] \xypo & A \arcof[d] \\
		T(m) \ar[r] & B
		}
	\end{displaymath}
	Then if $A \to A(K(A))$ is a weak equivalence, so is $B \to A(K(B))$
}
\Proof{
	Applying $K$ to the above diagram gives a pullback diagram of simplicial sets, where the induced vertical maps are fibrations (since $K$ is right Quillen). In other words, the induced square is a homotopy pullback.

	Applying $A$ again gives the following cube of cdga's:
	\begin{displaymath}
		\xymatrix @=9pt{
		S(m+1) \arcof[dd] \ar[rr] \arwe[rd] \xypo & & A \arcof'[d][dd] \arwe[rd] & \\
		& A(K(S(m+1))) \ar[dd] \ar[rr] & & A(K(A)) \ar[dd] \\
		T(m) \ar'[r][rr] \arwe[rd] & & B \ar[rd] & \\
		& A(K(T(m))) \ar[rr] & & A(K(B))
		}
	\end{displaymath}
	Note that we have a weak equivalence in the top left corner, by the base case ($S(m+1) = (\Lambda(v), 0)$). The weak equivalence in the top right is by assumption. Finally the bottom left map is a weak equivalence because both cdga's are acyclic.

	To conclude that $B \to A(K(B))$ is a weak equivalence, we wish to prove that the front face of the cube is a homotopy pushout, as the back face clearly is one. This is a consequence of the Eilenberg-Moore spectral sequence \cite{mccleary}.
}

Now we wish to use the previous lemma as an induction step for minimal models. Let $(\Lambda V, d)$ be some minimal algebra. Write $V(n+1) = V(n) \oplus V'$ and let $v \in V'$ of degree $\deg{v} = k$, then the minimal algebra $(\Lambda (V(n) \oplus \Q \cdot v), d)$ is the pushout in the following diagram, where $f$ sends the generator $c$ to $dv$.
\begin{displaymath}
	\xymatrix{
	S(k) \arcof[d] \ar[r]^f \xypo & (\Lambda V(n), d) \ar[d] \\
	T(k-1) \ar[r] & (\Lambda (V(n) \oplus \Q \cdot v), d)
	}
\end{displaymath}
In particular if the vector space $V'$ is finitely generated, we can repeat this procedure for all basis elements (it does not matter in what order we do so, as $dv \in \Lambda V(n)$). So in this case, if $(\Lambda V(n), d) \to A(K(\Lambda V(n), d))$ is a weak equivalence, so is $(\Lambda V(n+1), d) \to A(K(\Lambda V(n+1), d))$

\Corollary{}{
	Let $(\Lambda V, d)$ be a $1$-connected minimal algebra with $V^i$ finite dimensional for all $i$. Then $(\Lambda V, d) \to A(K(\Lambda V, d))$ is a weak equivalence.
}
\Proof{
	Note that if we want to prove the isomorphism $H^i(\Lambda V, d) \to H^i(A(K(\Lambda V, d)))$ it is enough to prove that $H^i(\Lambda V^{\leq i}, d) \to H^i(A(K(\Lambda V^{\leq i}, d)))$ is an isomorphism (as the elements of higher degree do not change the isomorphism). By the $1$-connectedness we can choose our filtration to respect the degree by \LemmaRef{1-reduced-minimal-model}.

	Now $V(n)$ is finitely generated for all $n$ by assumption. By the inductive procedure above we see that $(\Lambda V(n), d) \to A(K(\Lambda V(n), d))$ is a weak equivalence for all $n$. Hence $(\Lambda V, d) \to A(K(\Lambda V, d))$ is a weak equivalence.
}

\todo{$X \to K(A(X))$}

We have proven the following theorem.
\Theorem{main-theorem}{
	The functors $A$ and $K$ induce an equivalence of homotopy categories, when restricted to rational, $1$-connected objects of finite type. more formally, we have:
	$$ \Ho(\sSet_1^{\Q,f}) \iso \Ho(\CDGA_{\Q,1,f}). $$

	Furthermore, for any $1$-connected space $X$ of finite type, we have the following isomorphism of groups:
	$$ \pi_i(X) \tensor \Q \iso {V^i}^\ast, $$
	where $(\Lambda V, d)$ is the minimal model of $A(X)$.
}

