
\Chapter{Homotopy Theory For cdga's}{HomotopyTheoryCDGA}

Recall that a cdga $A$ is a commutative differential graded algebra, meaning that
\begin{itemize}
	\item it has a grading: $A = \bigoplus_{n\in\N} A^n$,
	\item it has a differential: $d: A \to A$ with $d^2 = 0$,
	\item it has an associative and unital multiplication: $\mu: A \tensor A \to A$ and
	\item it is commutative: $x y = (-1)^{\deg{x}\cdot\deg{y}} y x$.
\end{itemize}
And all of the above structure is compatible with each other (e.g. the differential is a derivation of degree $1$, the maps are graded, \dots). We have a left adjoint $\Lambda$ to the forgetful functor $U$ which assigns the free graded commutative algebras $\Lambda V$ to a graded module $V$. This extends to an adjunction (also called $\Lambda$ and $U$) between commutative differential graded algebras and differential graded modules.

In homological algebra we are especially interested in \emph{quasi isomorphisms}, i.e. maps $f: A \to B$ inducing an isomorphism on cohomology: $H(f): HA \iso HB$. This notions makes sense for any object with a differential.

We furthermore have the following categorical properties of cdga's:
\begin{itemize}
	\item The finite coproduct in $\CDGA_\k$ is the (graded) tensor product.
	\item The finite product in $\CDGA_\k$ is the cartesian product (with pointwise operations).
	\item The equalizer (resp. coequalizer) of $f$ and $g$ is given by the kernel (resp. cokernel) of $f - g$. Together with the (co)products this defines pullbacks and pushouts.
	\item $\k$ and $0$ are the initial and final object.
\end{itemize}

In this chapter the ring $\k$ is assumed to be a field of characteristic zero. In particular the modules are vector spaces.

\todo{augmentations?}

\section{Cochain models for the $n$-disk and $n$-sphere}

We will first define some basic cochain complexes which model the $n$-disk and $n$-sphere. $D(n)$ is the cochain complex generated by one element $b \in D(n)^n$ and its differential $c = d(b) \in D(n)^{n+1}$. $S(n)$ is the cochain complex generated by one element $a \in S(n)^n$ which differential vanishes (i.e. $da = 0$). In other words:

$$ D(n) = ... \to 0 \to \k \to \k \to 0 \to ... $$
$$ S(n) = ... \to 0 \to \k \to 0 \to 0 \to ... $$

Note that $D(n)$ is acyclic for all $n$, or put in different words: $j_n : 0 \to D(n)$ is a quasi isomorphism. The sphere $S(n)$ has exactly one non-trivial cohomology group $H^n(S(n)) = \k \cdot [a]$. There is an injective function $i_n : S(n+1) \to D(n)$, sending $a$ to $c$. The maps $j_n$ and $i_n$ play the following important role in the model structure of cochain complexes:

\begin{claim}
	The set $I = \{i_n : S(n+1) \to D(n) \I n \in \N\}$ generates all cofibrations and the set $J = \{j_n : 0 \to D(n) \I n \in \N\}$ generates all trivial cofibrations.
\end{claim}

The proof is omitted but can be found in different sources \todo{Cite sources}. In the next section we will prove a similar result for cdga's, so the reader can also refer to that proof.

$S(n)$ plays a another special role: maps from $S(n)$ to some cochain complex $X$ correspond directly to elements in the kernel of $\restr{d}{X^n}$. Any such map is null-homotopic precisely when the corresponding elements in the kernel is a coboundary. So there is a natural isomorphism: $\Hom(S(n), X) / ~ \iso H^n(X)$. So the cohomology groups can be considered as honest homotopy groups.

By using the free cdga functor we can turn these cochain complexes into cdga's $\Lambda(D(n))$ and $\Lambda(S(n))$. So $\Lambda(D(n))$ consists of linear combinations of $b^n$ and $c b^n$ when $n$ is even, and $c^n b$ and $c^n$ when $n$ is odd. In both cases we can compute the differentials using the Leibniz rule:
$$ d(b^n) = n \cdot c b^{n-1} $$
$$ d(c b^n) = 0 $$

$$ d(c^n b) = c^{n+1} $$
$$ d(c^n) = 0 $$

Those cocycles are in fact coboundaries (remember that $\k$ is a field of characteristic $0$):
$$ c b^n = \frac{1}{n} d(b^{n+1}) $$
$$ c^n = d(b c^{n-1}) $$

There are no additional cocycles in $\Lambda(D(n))$ besides the constants and $c$. So we conclude that $\Lambda(D(n))$ is acyclic as an algebra. In other words $\Lambda(j_n): \k \to \Lambda D(n)$ is a quasi isomorphism.

The situation for $\Lambda S(n)$ is easier: when $n$ is even it is given by polynomials in $a$, if $n$ is odd it is an exterior algebra (i.e. $a^2 = 0$). Again the sets $\Lambda(I) = \{ \Lambda(i_n) : \Lambda S(n+1) \to \Lambda D(n) \I n \in \N\}$ and $\Lambda(J) = \{ \Lambda(j_n) : \k \to \Lambda D(n) \I n \in \N\}$ play an important role.

\begin{theorem}
	The sets $\Lambda(I)$ and $\Lambda(J)$ generate a model structure on $\CDGA_\k$ where:
	\begin{itemize}
		\item weak equivalences are quasi isomorphisms,
		\item fibrations are (degree wise) surjective maps and
		\item cofibrations are maps with the left lifting property against trivial fibrations.
	\end{itemize}
\end{theorem}

We will prove this theorem in the next section. Note that the functors $\Lambda$ and $U$ thus form a Quillen pair with this model structure.

\subsubsection{Why we need $\Char{\k} = 0$ for algebras}
The above Quillen pair $(\Lambda, U)$ fails to be a Quillen pair if $\Char{\k} = p \neq 0$. We will show this by proving that the maps $\Lambda(j_n)$ are not weak equivalences for even $n$. Consider $b^p \in D(n)$, then by the Leibniz rule:
$$ d(b^p) = p \cdot c b^{p-1} = 0. $$
So $b^p$ is a cocycle. Now assume $b^p = dx$ for some $x$ of degree $pn - 1$, then $x$ contains a factor $c$ for degree reasons. By the calculations above we see that any element containing $c$ has a trivial differential or has a factor $c$ in its differential, contradicting $b^p = dx$. So this cocycle is not a coboundary and $\Lambda D(n)$ is not acyclic.


\section{The Quillen model structure on \titleCDGA}

In this section we will define a model structure on cdga's over a field $\k$ of characteristic zero, where the weak equivalences are quasi isomorphisms and fibrations are surjective maps. The cofibrations are defined to be the maps with a left lifting property with respect to trivial fibrations.

\begin{proposition}
	There is a model structure on $\CDGA_\k$ where $f: A \to B$ is
	\begin{itemize}
		\item a \emph{weak equivalence} if $f$ is a quasi isomorphism,
		\item a \emph{fibration} if $f$ is an surjective and
		\item a \emph{cofibration} if $f$ has the LLP w.r.t. trivial fibrations
	\end{itemize}
\end{proposition}

We will prove the different axioms in the following lemmas. First observe that the classes as defined above are indeed closed under composition and contain all isomorphisms.

Note that with these classes, every cdga is a fibrant object.

\begin{lemma}
	[MC1] The category has all finite limits and colimits.
\end{lemma}
\begin{proof}
	As discussed earlier products are given by direct sums and equalizers are kernels. Furthermore the coproducts are tensor products and coequalizers are quotients.
\end{proof}

\begin{lemma}
	[MC2] The \emph{2-out-of-3} property for quasi isomorphisms.
\end{lemma}
\begin{proof}
	Let $f$ and $g$ be two maps such that two out of $f$, $g$ and $fg$ are weak equivalences. This means that two out of $H(f)$, $H(g)$ and $H(f)H(g)$ are isomorphisms. The 2-out-of-3 property holds for isomorphisms, proving the statement.
\end{proof}

\begin{lemma}
	[MC3] All three classes are closed under retracts
\end{lemma}
\begin{proof}
	For the class of weak equivalences and fibrations this follows easily from basic category theory. For cofibrations we consider the following diagram where the horizontal compositions are identities:
	\[ \xymatrix{
		A' \ar[r] \ar[d]^g & A \ar[r] \arcof[d]^f & A' \ar[d]^g \\
		B' \ar[r] & B \ar[r] & B'
	}\]
	We need to prove that $g$ is a cofibration, so for any lifting problem with a trivial fibration we need to find a lift. We are in the following situation:
		\[ \xymatrix{
		A' \ar[r] \ar[d]^g & A \ar[r] \arcof[d]^f & A' \ar[r] \ar[d]^g & X \artfib[d] \\
		B' \ar[r] & B \ar[r] & B' \ar[r] & Y
	}\]
	Now we can find a lift starting at $B$, since $f$ is a cofibration. By precomposition we obtain a lift $B' \to X$.
\end{proof}

Next we will prove the factorization property [MC5]. We will prove one part directly and the other by Quillen's small object argument. When proved, we get an easy way to prove the missing lifting property of [MC4]. For the Quillen's small object argument we use a class of generating cofibrations.

\begin{definition}
	Define the following objects and sets of maps:
	\begin{itemize}
		\item $\Lambda S(n)$ is the cdga generated by one element $a$ of degree $n$ such that $da = 0$.
		\item $\Lambda D(n)$ is the cdga generated by two elements $b$ and $c$ of degree $n$ and $n+1$ respectively, such that $db = c$ (and necessarily $dc = 0$).
		\item $I = \{ i_n: \k \to \Lambda D(n) \I n \in \N \}$ is the set of units.
		\item $J = \{ j_n: \Lambda S(n+1) \to \Lambda D(n) \I n \in \N \}$ is the set of inclusions $j_n$ defined by $j_n(a) = b$.
	\end{itemize}
\end{definition}

\Lemma{cdga-mc5a}{
	[MC5a] A map $f: A \to X$ can be factorized as $f = pi$ where $i$ is a trivial cofibration and $p$ a fibration.
}
\Proof{
	Consider the free cdga $C = \bigtensor_{x \in X} T(\deg{x})$. There is an obvious surjective map $p: C \to X$ which sends a generator corresponding to $x$ to $x$. Now define maps $\phi$ and $\psi$ in
	\[ A \tot{\phi} A \tensor C \tot{\psi} X\]
	by $\phi(a) = a \tensor 1$ and $\psi(a \tensor c) = f(a) \cdot p(c)$. Now $\psi$ is clearly surjective (as $p$ is) and $\phi$ is clearly a weak equivalence (by the Künneth theorem). Furthermore $\phi$ is a cofibration as we can construct lifts using the freeness of $C$.
}

\Remark{cdga-mc5a-left-inverse}{
	The map $\phi$ in the above construction has a left inverse $\overline{\phi}$ given by $\overline{\phi}(x \tensor c) = x \cdot \counit(c)$, where $\counit$ is the natural augmentation of a free cdga (i.e. it send $1$ to $1$ and all generators to $0$). Clearly $\overline{\phi} \phi = \id$, and so $\overline{\phi}$ is a fibration as well.

	Furthermore, if $f$ is a weak equivalence then by the 2-out-of-3 property both $\phi$ and $\psi$ are weak equivalences. Applying it once more, we find that $\overline{\phi}$ too is a weak equivalence. So for any weak equivalence $f: A \to X$ we find trivial fibrations $\overline{\phi} : A \tensor C \fib A$ and $\psi: A \tensor C \fib X$ compatible with $f$.
}

\begin{lemma}
	The maps $i_n$ are trivial cofibrations and the maps $j_n$ are cofibrations.
\end{lemma}
\begin{proof}
	Since $H(\Lambda D(n)) = \k$ (as stated earlier this uses \linebreak $\Char{\k} = 0$) we see that indeed $H(i_n)$ is an isomorphism. For the lifting property of $i_n$ and $j_n$ simply use surjectivity of the fibrations and the freeness of $\Lambda D(n)$ and $\Lambda S(n)$.
\end{proof}

\begin{lemma}
	The class of cofibrations is saturated.
\end{lemma}
\begin{proof}
	We need to prove that the classes are closed under retracts (this is already done), pushouts and transfinite compositions. For the class of cofibrations, this is easy as they are defined by the LLP and colimits behave nice with respect to such classes. 
\end{proof}

As a consequence of the above two lemmas, the class generated by $J$ is contained in the class of cofibrations. We can characterize trivial fibrations with $J$.

\begin{lemma}
	If $p: X \to Y$ has the RLP w.r.t. $J$ then $p$ is a trivial fibration.
\end{lemma}
\begin{proof}
	Let $y \in Y$ be of degree $n$ and $dy$ its boundary. By assumption we can find a lift in the following diagram:
	\[ \xymatrix{
		\Lambda S(n+1) \arcof[d]^{j_n} \ar[r]^-{a \mapsto 0} & X \ar[d]^f \\
		\Lambda D(n) \ar[r]^-{b \mapsto dy} & Y
	} \]
	The lift $h: D(n) \to X$ defines a preimage $x' = h(b)$ for $dy$. Now we can define a similar square to find a preimage $x$ of $y$ as follows:
	\[ \xymatrix{
		\Lambda S(n) \arcof[d]^{j_{n-1}} \ar[r]^-{a \mapsto x'} & X \ar[d]^f \\
		\Lambda D(n-1) \ar[r]^-{b \mapsto y} & Y
	} \]
	The lift $h : D(n-1) \to X$ defines $x = h(b)$. This proves that $f$ is surjective. Note that $dx = x'$.

	Now if $[y] \in H(Y)$ is some class, then $dy = 0$, and so by the above we find a preimage $x$ of $y$ such that $dx = 0$, proving that $H(f)$ is surjective. Now let $[x] \in H(X)$ such that $[f(x)] = 0$, then there is an element $\beta$ such that $f(x) = d\beta$, again by the above we can lift $\beta$ to get $x = d\alpha$., hence $H(f)$ is injective. Conclude that $f$ is a trivial fibration.
\end{proof}

We can use Quillen's small object argument with the set $J$. The argument directly proves the following lemma. Together with the above lemmas this translates to the required factorization.

\Lemma{cdga-mc5b}{
	A map $f: A \to X$ can be factorized as $f = pi$ where $i$ is in the class generated by $J$ and $p$ has the RLP w.r.t. $J$.
}
\Proof{
	This follows from Quillen's small object argument.
}

\Corollary{cdga-mc5b}{
	[MC5b] A map $f: A \to X$ can be factorized as $f = pi$ where $i$ is a cofibration and $p$ a trivial fibration.
}

\Lemma{cdga-mc4}{
	[MC4] The lifting properties.
}
\Proof{
	One part is already established by definition (cofibrations are defined by an LLP). It remains to show that we can lift in the following situation:
	\[\xymatrix{
		A \ar[r] \artcof[d]^f & X \arfib[d] \\
		B \ar[r] & Y
	}\]
	Now factor $f = pi$, where $p$ is a fibration and $i$ a trivial cofibration. By the 2-out-of-3 property $p$ is also a weak equivalence and we can find a lift in the following diagram:
	\[\xymatrix{
		A \ar[r]^i \arcof[d]^f & Z \artfib[d]^p \\
		B \ar[r]^\id \ar@{-->}[ur] & B
	}\]
	This defines $f$ as a retract of $i$. Now we know that $i$ has the LLP w.r.t. fibrations (by the small object argument above), hence $f$ has the LLP w.r.t. fibrations as well.
}


\section{Homotopy relations on \titleCDGA}

Although the abstract theory of model categories gives us tools to construct a homotopy relation (\DefinitionRef{homotopy}), it is useful to have a concrete notion of homotopic maps.

Consider the free cdga on one generator $\Lambda(t, dt)$, where $\deg{t} = 0$, this can be thought of as the (dual) unit interval with endpoints $1$ and $t$. We define two \emph{endpoint maps} as follows:
$$ d_0, d_1 : \Lambda(t, dt) \to \k $$
$$ d_0(t) = 1, \qquad d_1(t) = 0, $$
this extends linearly and multiplicatively. Note that it follows that we have $d_0(1-t) = 0$ and $d_1(1-t) = 1$. These two functions extend to tensor products as $d_0, d_1: \Lambda(t, dt) \tensor X \to \k \tensor X \tot{\iso} X$.

\Definition{cdga_homotopy}{
	We call $f, g: A \to X$ homotopic ($f \simeq g$) if there is a map
	$$ h: A \to \Lambda(t, dt) \tensor X, $$
	such that $d_0 h = g$ and $d_1 h = f$.
}

In terms of model categories, such a homotopy is a right homotopy and the object $\Lambda(t, dt) \tensor X$ is a path object for $X$. We can easily see that it is a very good path object. First note that $\Lambda(t, dt) \tensor X \tot{(d_0, d_1)} X \oplus X$ is surjective (for $(x, y) \in X \oplus X$ take $t \tensor x + (1-t) \tensor y$). Secondly we note that $\Lambda(t, dt) = \Lambda(D(0))$ and hence $\k \to \Lambda(t, dt)$ is a cofibration, by \LemmaRef{model-cats-coproducts} we have that $X \to \Lambda(t, dt) \tensor X$ is a (necessarily trivial) cofibration.

Clearly we have that $f \simeq g$ implies $f \simeq^r g$ (see \DefinitionRef{right_homotopy}), however the converse need not be true.

\Lemma{cdga_homotopy}{
	If $A$ is a cofibrant cdga and $f \simeq^r g: A \to X$, then $f \simeq g$ in the above sense.
}
\Proof{
	Because $A$ is cofibrant, there is a very good homotopy $H$. Consider a lifting problem to construct a map $Path_X \to \Lambda(t, dt) \tensor X$.
}

\Corollary{cdga_homotopy_eqrel}{
	For cofibrant $A$, $\simeq$ defines a equivalence relation.
}
\Definition{cdga_homotopy_classes}{
	For cofibrant $A$ define the set of equivalence classes as:
	$$ [A, X] = \Hom_{\CDGA_\k}(A, X) / \simeq. $$
}

The results from model categories immediately imply the following results. \todo{Refereer expliciet}
\Corollary{cdga_homotopy_properties}{
	Let $A$ be cofibrant.
	\begin{itemize}
		\item Let $i: A \to B$ be a trivial cofibration, then the induced map $i^\ast: [B, X] \to [A, X]$ is a bijection.
		\item Let $p: X \to Y$ be a trivial fibration, then the induced map $p_\ast: [A, X] \to [A, Y]$ is a bijection.
		\item Let $A$ and $X$ both be cofibrant, then $f: A \we X$ is a weak equivalence if and only if $f$ is a strong homotopy equivalence. Moreover, the two induced maps are bijections:
		$$ f_\ast: [Z, A] \tot{\iso} [Z, X], $$
		$$ f^\ast: [X, Z] \tot{\iso} [A, X]. $$
		\todo{De eerste werkt ook als $i$ gewoon een w.e. is. (Gebruik factorizatie.)}
	\end{itemize}
}

\Lemma{cdga_homotopy_homology}{
	Let $f, g: A \to X$ be two homotopic maps, then $H(f) = H(g): HA \to HX$.
}
\Proof{
	We only need to consider $H(d_0)$ and $H(d_1)$. \todo{Bewijs afmaken}
}


\section{Homotopy theory of augmented cdga's}

Recall that an augmented cdga is a cdga $A$ with an algebra map $A \tot{\counit} \k$ (this implies that $\counit \unit = \id$). This is precisely the dual notion of a pointed space. We will use the general fact that if $\cat{C}$ is a model category, then the over (resp. under) category $\cat{C} / A$ (resp. $A / \cat{C}$) for any object $A$ admit an induced model structure. In particular, the category of augmented cdga's (with augmentation preserving maps) has a model structure with the fibrations, cofibrations and weak equilavences as above.

Although the model structure is completely induced, it might still be fruitful to discuss the right notion of a homotopy for augmented cdga's. Consider the following pullback of cdga's:
\[ \xymatrix{
	\Lambda(t, dt) \overline{\tensor} A \ar[r] \xypb \ar[d] & \Lambda(t, dt) \tensor A \ar[d] \\
	\k \ar[r] & \k \tensor \Lambda(t, dt)
}\]
The pullback is the subspace of elements $x \tensor a$ in $\Lambda(t, dt) \tensor A$ such that $\counit(a) \cdot x \in \k$. Note that this construction is dual to a construction on topological spaces: in order to define a homotopy which is constant on the point $x_0$, we define the homotopy to be a map from a quotient ${X \times I} / {x_0 \times I}$.
\Definition{homotopy-augmented}{
	Two maps $f, g: A \to X$ between augmented cdga's are said to be \emph{homotopic} if there is a map
	$$h : A \to \Lambda(t, dt) \overline{\tensor} X$$
	such that $d_0 h = g$ and $d_1 h = f$.
}

In the next section homotopy groups of augmented cdga's will be defined. In order to define this we first need another tool.
\Definition{indecomposables}{
		Define the \Def{augmentation ideal} of $A$ as $\overline{A} = \ker \counit$. Define the \Def{cochain complex of indecomposables} of $A$ as $QA = \overline{A} / \overline{A} \cdot \overline{A}$.
} 

The first observation one should make is that $Q$ is a functor from algebras to modules (or differential algebras to differential modules) which is particularly nice for free algebras, as we have that $Q \Lambda V = V$ for any (differential) module $V$.

\todo{tensor}


\section{Homotopy groups of cdga's}

As the eventual goal is to compare the homotopy theory of spaces with the homotopy theory of cdga's, it is natural to investigate an analogue of homotopy groups in the category of cdga's. In topology we can only define homotopy groups on pointed spaces, dually we will consider augmented cdga's in this section.

\Definition{cdga-homotopy-groups}{
	The \Def{homotopy groups of an augmented cdga} $A$ are
	$$ \pi^i(A) = H^i(QA). $$
}

This construction is functorial (since both $Q$ and $H$ are) and, as the following lemma shows, homotopy invariant.

\Lemma{cdga-homotopic-maps-equal-pin}{
	Let $f: A \to X$ and $g: A \to X$ be a maps of augmented cdga's. If $f$ and $g$ are homotopic, then the induced maps are equal:
	$$ f_\ast = g_\ast : \pi_\ast(A) \to \pi_\ast(X). $$
}
\Proof{
	Let $h: A \to \Lambda(t, dt) \tensor X$ be a homotopy. We will, just as in \LemmaRef{cdga-homotopy-homology}, prove that the maps $HQ(d_0)$ and $HQ(d_1)$ are equal, then it follows that $HQ(f) = HQ(d_1 h) = HQ(d_0 h) = HQ(g)$.

	Using \LemmaRef{Q-preserves-copord} we can identify the induced maps $Q(d_i) : Q(\Lambda(t, dt) \tensor X) \to Q(X)$ with maps
	\[ Q(d_i) : Q(\Lambda(t, dt)) \oplus Q(A) \to Q(A). \]
	Now $Q(\Lambda(t, dt)) = D(0)$ and hence it is acyclic, so when we pass to homology, this term vanishes. In other words both maps ${d_i}_\ast : H(D(0)) \oplus H(Q(A)) \to H(Q(A))$ are the identity maps on $H(Q(A))$.
}

Consider the augmented cdga $V(n) = S(n) \oplus \k$, with trivial multiplication and where the term $\k$ is used for the unit and augmentation. This augmented cdga can be thought of as a specific model of the sphere. In particular the homotopy groups can be expressed as follows.

\Lemma{cdga-dual-homotopy-groups}{
	There is a natural bijection for any augmented cdga $A$
	$$ [A, V(n)] \tot{\iso} \Hom_\k(\pi^n(A), \k). $$
}
\Proof{
	Note that $Q(V(n))$ in degree $n$ is just $\k$ and $0$ in the other degrees, so its homotopy groups consists of a single $\k$ in degree $n$. This establishes the map:
	$$ \pi^n: \Hom(A, V(n)) \to \Hom_\k(\pi^n(A), \k). $$

	Now by \LemmaRef{cdga-homotopic-maps-equal-pin} we get a map from the set of homotopy classes $[A, V(n)]$ instead of the $\Hom$-set. It remains to prove that the map is an isomorphism. Surjectivity follows easily. Given a map $f: \pi^n(A) \to \k$, we can extend this to $A \to V(n)$ because the multiplication on $V(n)$ is trivial.

	For injectivity suppose $\phi, \psi: A \to V(n)$ be two maps such that $\pi^n(\phi) = \pi^n(\psi)$. We will first define a chain homotopy $D: A^\ast \to V(n)^{\ast - 1}$, for this we only need to specify the map $D^n: A^{n+1} \to V(n)^n = \Q$. Decompose the vector space $A^{n+1}$ as $A^{n+1} = \im d \oplus V$ for some $V$. Now set $D^n(v) = 0$ for all $v \in V$ and $D^n(db) = \phi(b) - \psi(b)$. We should check that $D$ is well defined. Note that for cycles we get $\phi(c) = \psi(c)$, as $H(Q(\phi)) = H(Q(\psi))$. So if $db = dc$, then we get $D(db) = \phi(b) - \psi(b) = \phi(c) - \psi(c) = D(dc)$, i.e. $D$ is well defined. We can now define a map of augmented cdga's:
	\begin{align*}
		h : X &\to \Lambda(t, dt) \overline{\tensor} V(n) \\
		    x &\mapsto dt \tensor D(x) + 1 \tensor \phi(x) - t \tensor \phi(x) + t \tensor \psi(x)
	\end{align*}
	This map commutes with the differential by the definition of $D$. Now we see that $d_0 h = \psi$ and $d_1 h = \phi$. Hence the two maps represent the same class, and we have proven the injectivity.
}

From now on the dual of a vector space will be denoted as $V^\ast = \Hom_\k(V, \k)$. So the above lemma states that there is a bijection $[A, V(n)] \iso \pi^n(A)^\ast$.

In topology we know that a fibration induces a long exact sequence of homotopy groups. In the case of cdga's a similar long exact sequence for a cofibration will exist.

\Lemma{long-exact-cdga-homotopy}{
	Given a pushout square of augmented cdga's
	\[ \xymatrix{
		A \ar[d]^-f \arcof[r]^-g \xypo & C \ar[d]^-i \\
		B \ar[r]^-j & P
	} \]
	where $g$ is a cofibration. There is a natural long exact sequence
	\[ \pi^o(V) \tot{(f_\ast, g_\ast)} \pi^0(B) \oplus \pi^0(C) \tot{j_\ast - i_\ast} \pi^0(P) \tot{\del} \pi^1(A) \to \cdots \]
}
\Proof{
	First note that $j$ is also a cofibration. By \LemmaRef{Q-preserves-cofibs} the maps $Qg$ and $Qj$ are injective in positive degrees. By applying $Q$ we get two exact sequence (in positive degrees) as shown in the following diagram. By the fact that $Q$ preserves pushouts (\CorollaryRef{Q-preserves-pushouts}) the cokernels coincide.
	\[ \xymatrix {
		0 \ar[r] & Q(A) \ar[r] \ar[d] \xypo & Q(C) \ar[r] \ar[d] & \coker(f_\ast) \ar[r] \ar[d] & 0 \\
		0 \ar[r] & Q(B) \ar[r] & Q(P) \ar[r] & \coker(f_\ast) \ar[r] & 0
	} \]
	Now the well known Mayer-Vietoris exact sequence can be constructed. This proves the statement.
}

\Corollary{long-exact-cdga-homotopy}{
	When we take $B = \k$ in the above situation, we get a long exact sequence
	\[ \pi^0(A) \tot{g_\ast} \pi^0(C) \to \pi^0(\coker(g)) \to \pi^1(A) \to \cdots \]
}



\Chapter{Polynomial Forms}{Adjunction}
\label{sec:cdga-of-polynomials}

\section{CDGA of Polynomials}

We will now give a cdga model for the $n$-simplex $\Delta^n$. This then allows for simplicial methods. In the following definition one should remember the topological $n$-simplex defined as convex span.

\Definition{apl}{
	For all $n \in \N$ define the following cdga:
	$$ (\Apl)_n = \frac{\Lambda(x_0, \ldots, x_n, d x_0, \ldots, d x_n)}{(\sum_{i=0}^n x_i - 1, \sum_{i=0}^n d x_i)}, $$
	where $\deg{x_i} = 0$. So it is the free cdga with $n+1$ generators and their differentials such that $\sum_{i=0}^n x_i = 1$ and in order to be well behaved $\sum_{i=0}^n d x_i = 0$.
}

Note that the inclusion $\Lambda(x_1, \ldots, x_n, d x_1, \ldots, d x_n) \to \Apl_n$ is an isomorphism of cdga's. So $\Apl_n$ is free and (algebra) maps from it are determined by their images on $x_i$ for $i = 1, \ldots, n$ (also note that this determines the images for $d x_i$). This fact will be used throughout. Also note that we have already seen the dual unit interval $\Lambda(t, dt)$ which is isomorphic to $\Apl_1$.

These cdga's will assemble into a simplicial cdga when we define the face and degeneracy maps as follows ($j = 1, \ldots, n$):

$$ d_i(x_j) = \begin{cases}
	x_{j-1}, &\text{ if } i < j \\
	0,       &\text{ if } i = j \\
	x_j,     &\text{ if } i > j
\end{cases} \qquad d_i : \Apl_n \to \Apl_{n-1} $$
$$ s_i(x_j) = \begin{cases}
	x_{j+1},       &\text{ if } i < j \\
	x_j + x_{j+1}, &\text{ if } i = j \\
	x_j,           &\text{ if } i > j	
\end{cases} \qquad s_i : \Apl_n \to \Apl_{n+1} $$

One can check that $\Apl \in \simplicial{\CDGA_\k}$. We will denote the subspace of homogeneous elements of degree $k$ as $\Apl^k$, this is a simplicial $\k$-module as the maps $d_i$ and $s_i$ are graded maps of degree $0$.

\Lemma{apl-contractible}{
	$\Apl^k$ is contractible.
}
\Proof{
	\todo{Note geometric interpretation} We will prove this by defining an extra degeneracy $s: \Apl_n \to \Apl_{n+1}$. Define for $i = 1, \ldots, n$:
	\begin{align*}
		s(1) &= (1-x_0)^2 \\
		s(x_i) &= (1-x_0) \cdot x_{i+1}
	\end{align*}
	Extend on the differentials and multiplicatively on $\Apl_n$. As $s(1) \neq 1$ this map is not an algebra map, however it well-defined as a map of cochain complexes. In particular when restricted to degree $k$ we get a linear map:
	$$ s: \Apl^k_n \to \Apl^k_{n+1}. $$
	Proving the necessary properties of an extra degeneracy is fairly easy. For $n \geq 1$ we get (on generators):
	\begin{align*}
		d_0 s(1)   &= d_0 (1 - x_0)^2 = (1 - 0) \cdot (1 - 0) = 1 \\
		d_0 s(x_i) &= d_0((1-x_0)x_{i+1}) = d_0(1-x_0) \cdot x_i  \\
		           &= (1-0) \cdot x_i = x_i
	\end{align*}
	So $d_0 s = \id$.
	\begin{align*}
		d_{i+1} s(1) &= d_{i+1} (1 - x_0)^2 = d_{i+1} (\sum_{j=1}^n x_j)^2 \\
		             &= (\sum_{j=1}^{n-1} x_j)^2 = (1-x_0)^2 = s d_i(1)    \\
		d_{i+1} s(x_j) &= d_{i+1}(1-x_0) d_{i+1}(x_j) = (1-x_0) d_i(x_{j+1}) = s d_i (x_j)
	\end{align*}
	So $d_{i+1} s = s d_i$. Similarly $s_{i+1} s = s s_i$. And finally for $n=0$ we have $d_1 s = 0$.

	So we have an extra degeneracy $s: \Apl^k \to \Apl^k$, and hence (see for example \cite{goerss}) we have that $\Apl^k$ is contractible. As a consequence $\Apl^k \to \ast$ is a weak equivalence.
}

\Lemma{apl-kan-complex}{
	$\Apl^k$ is a Kan complex.
}
\Proof{
	By the simple fact that $\Apl^k$ is a simplicial group, it is a Kan complex \cite{goerss}.
}

Combining these two lemmas gives us the following.
\Corollary{apl-extendable}{
	$\Apl^k \to \ast$ is a trivial fibration in the standard model structure on $\sSet$.
}

Besides the simplicial structure of $\Apl$, there is also the structure of a cochain complex.
\Lemma{apl-acyclic}{
	$\Apl_n$ is acyclic, i.e. $H(\Apl_n) = \k \cdot [1]$.
}
\Proof{
	This is clear for $\Apl_0 = \k \cdot 1$. For $\Apl_1$ we see that $\Apl_1 = \Lambda(x_1, d x_1) \iso \Lambda D(0)$, which we proved to be acyclic in the previous section.

	For general $n$ we can identify $\Apl_n \iso \bigtensor_{i=1}^n \Lambda(x_i, d x_i)$, because $\Lambda$ is left adjoint and hence preserves coproducts. By the Künneth theorem \TheoremRef{kunneth} we conclude $H(\Apl_n) \iso \bigtensor_{i=1}^n H \Lambda(x_i, d x_i) \iso \bigtensor_{i=1}^n H \Lambda D(0) \iso \k \cdot [1]$.

	So indeed $\Apl_n$ is acyclic for all $n$.
}



\section{Polynomial Forms on a Space}
\label{sec:polynomial-forms}

There is a general way to construct functors from $\sSet$ whenever we have some simplicial object. In our case we have the simplicial cdga $\Apl$ (which is nothing more than a functor $\opCat{\DELTA} \to \CDGA$) and we want to extend to a contravariant functor $\sSet \to \CDGA_\k$. This will be done via \Def{Kan extensions}.

Given a category $\cat{C}$ and a functor $F: \DELTA \to \cat{C}$, then define the following on objects:
\begin{align*}
	F_!(X)      &= \colim_{\Delta[n] \to X} F[n] & X \in \sSet \\
	F^\ast(C)_n &= \Hom_{\cat{C}}(F[n], Y)       & C \in \cat{C}
\end{align*}
A simplicial map $X \to Y$ induces a map of the diagrams of which we take colimits. Applying $F$ on these diagrams, make it clear that $F_!$ is functorial. Secondly we see readily that $F^\ast$ is functorial. By using the definition of colimit and the Yoneda lemma (Y) we can prove that $F_!$ is left adjoint to $F^\ast$:

\begin{align*}
	\Hom_\cat{C}(F_!(X), Y)
	&\iso \Hom_\cat{C}(\colim_{\Delta[n] \to X} F[n], Y) \\
	&\iso \lim_{\Delta[n] \to X} \Hom_\cat{C}(F[n], Y) \\
	&\iso \lim_{\Delta[n] \to X} F^\ast(Y)_n \\
	&\stackrel{\text{Y}}{\iso} \lim_{\Delta[n] \to X} \Hom_\sSet(\Delta[n], F^\ast(Y)) \\
	&\iso \Hom_\sSet(\colim_{\Delta[n] \to X} \Delta[n], F^\ast(Y)) \\
	&\iso \Hom_\sSet(X, F^\ast(Y)).
\end{align*}

Furthermore we have $F_! \circ \Delta[-] \iso F$. In short we have the following:

\cdiagram{Kan_Extension}

In our case where $F = \Apl$ and $\cat{C} = \CDGA_\k$ we get:

\cdiagram{Apl_Extension}


\subsection{The cochain complex of polynomial forms}

In our case we take the opposite category, so the definition of $A$ is in terms of a limit instead of colimit. This allows us to give a nicer description:

\begin{align*}
	A(X)
	&= \lim_{\Delta[n] \to X} \Apl_n 
	\stackrel{Y}{\iso} \lim_{\Delta[n] \to X} \Hom_\sSet(\Delta[n], \Apl) \\
	&\iso \Hom_\sSet(\colim_{\Delta[n] \to X}\Delta[n], \Apl)
	= \Hom_\sSet(X, \Apl),
\end{align*}

where the addition, multiplication and differential are defined pointwise. Conclude that we have the following contravariant functors (which form an adjoint pair):

\begin{align*}
	A(X) &= \Hom_\sSet(X, \Apl) & X \in \sSet \\
	K(C)_n &= \Hom_{\CDGA_\k}(C, \Apl_n) & C \in \CDGA_\k.
\end{align*}


\subsection{The singular cochain complex}

Another way to model the $n$-simplex is by the singular cochain complex associated to the topological $n$-simplices. Define the following (non-commutative) dga's \todo{Choose: normalized or not?}:
$$ C_n = C^\ast(\Delta^n; \k). $$
The inclusion maps $d^i : \Delta^n \to \Delta^{n+1}$ and the maps $s^i: \Delta^n \to \Delta^{n-1}$ induce face and degeneracy maps on the dga's $C_n$, turning $C$ into a simplicial dga. Again we can extend this to functors by Kan extensions

\cdiagram{C_Extension}

where the left adjoint is precisely the functor $C^\ast$ as noted in \cite{felix}. We will relate $\Apl$ and $C$ in order to obtain a natural quasi isomorphism $A(X) \we C^\ast(X)$ for every $X \in \sSet$. Furthermore this map preserves multiplication on the homology algebras.


\subsection{Integration and Stokes' theorem for polynomial forms}

In this section we will prove that the singular cochain complex is quasi isomorphic to the cochain complex of polynomial forms. In order to do so we will define an integration map $\int_n : \Apl_n^n \to \k$, which will induce a map $\oint_n : \Apl_n \to C_n$. For the simplices $\Delta[n]$ we already showed the cohomology agrees by the acyclicity of $\Apl_n = A(\Delta[n])$ (\LemmaRef{apl-acyclic}).

For any $v \in \Apl_n^n$, we can write $v$ as $v = p(x_1, \dots, x_n)dx_1 \dots dx_n$ where $p$ is a polynomial in $n$ variables. If $\Q \subset \k \subset \mathbb{C}$ we can integrate geometrically on the $n$-simplex:
$$ \int_n v = \int_0^1 \int_0^{1-x_n} \dots \int_0^{1 - x_2 - \dots - x_n} p(x_1, \dots, x_n) dx_1 dx_2 \dots dx_n, $$
which defines a well-defined linear map $\int_n : \Apl_n^n \to \k$. For general fields of characteristic zero we can define it formally on the generators of $\Apl_n^n$ (as vector space):
$$ \int_n x_1^{k_1} \dots x_n^{k_n} dx_1 \dots dx_n = \frac{k_1! \dots k_n!}{(k_1 + \dots + k_n + n)!}. $$

Let $x$ be a $k$-simplex of $\Delta[n]$, i.e. $x: \Delta[k] \to \Delta[n]$. Then $x$ induces a linear map $x^\ast: \Apl_n \to \Apl_k$. Let $v \in \Apl_n^k$, then $x^\ast(v) \in \Apl_k^k$, which we can integrate. Now define
$$ \oint_n(v)(x) = (-1)^\frac{k(k-1)}{2} \int_n x^\ast(v). $$
Note that $\oint_n(v): \Delta[n] \to \k$ is just a map, we can extend this linearly to chains on $\Delta[n]$ to obtain $\oint_n(v): \Z\Delta[n] \to \k$, in other words $\oint_n(v) \in C_n$. By linearity of $\int_n$ and $x^\ast$, we have a linear map $\oint_n: \Apl_n \to C_n$.

Next we will show that $\oint = \{\oint_n\}_n$ is a simplicial map and that each $\oint_n$ is a chain map, in other words $\oint : \Apl \to C_n$ is a simplicial chain map (of complexes). Let $\sigma: \Delta[n] \to \Delta[k]$, and $\sigma^\ast: \Apl_k \to \Apl_n$ its induced map. We need to prove $\oint_n \circ \sigma^\ast = \sigma^\ast \circ \oint_k$. We show this as follows:
\begin{align*}
	\oint_n (\sigma^\ast v)(x)
	&= (-1)^\frac{l(l-1)}{2} \int_l x^\ast(\sigma^\ast(v)) \\
	&= (-1)^\frac{l(l-1)}{2} \int_l (\sigma \circ x)^\ast(v) \\
	&= \oint_k (v)(\sigma \circ x) \\
	&= (\oint_k (v) \circ \sigma) (x) = \sigma^\ast (\oint_k(v)(x))
\end{align*}
For it to be a chain map, we need to prove $d \circ \oint_n = \oint_n \circ d$. This is precisely the same calculation as \emph{Stokes' theorem}. \todo{prove this?}

We now proved that $\oint$ is indeed a simplicial chain map. Note that $\oint_n$ need not to preserve multiplication, so it fails to be a map of cochain algebras. However $\oint(1) = 1$ and so the induced map on homology sends the class of $1$ in $H(\Apl_n) = \k \cdot [1]$ to the class of $1$ in $H(C_n) = \k \cdot [1]$. We have proven the following lemma.

\Lemma{apl-c-quasi-iso}{
	The map $\oint_n: \Apl_n \to C_n$ is a quasi isomorphism for all $n$.
}

Recall that we can identify $\Apl_n$ with $A(\Delta[n])$ and similarly for the singular cochain complex.
\Corollary{apl-c-quasi-iso}{
	The induced map $\oint: A(\Delta[n]) \to C^\ast(\Delta[n])$ is a quasi isomorphism for all $n$.
}

We will now prove that the map $\oint: A(X) \to C^\ast(X)$ is a quasi isomorphism for any space $X$. We will do this in several steps, the base case of simplices is already proven. With induction we will prove it for spaces with finitely many simplices. At last we will use a limit argument for the general case.

\Theorem{apl-c-quasi-iso}{
	The induced map $\oint: A(X) \to C^\ast(X)$ is a natural quasi isomorphism.
}
\Proof{
	Assume we have a simplicial set $X$ such that $\oint: A(X) \to C^\ast(X)$ is a quasi isomorphism. We can add a simplex by considering pushouts of the following form:
	\cdiagram{Apl_C_Quasi_Iso_Pushout}

	We can apply our two functors to it, and use the natural transformation $\oint$ to obtain the following cube:
	\cdiagram{Apl_C_Quasi_Iso_Cube}

	Note that $A(\Delta[n]) \we C^\ast(\Delta[n])$ by \CorollaryRef{apl-c-quasi-iso}, $A(X) \we C^\ast(X)$ by assumption and $A(\del \Delta[n]) \we C^\ast(\del \Delta[n])$ by induction. Secondly note that both $A$ and $C^\ast$ send injective maps to surjective maps, so we get fibrations on the right side of the diagram. Finally note that the front square and back square are pullbacks, by adjointness of $A$ and $C^\ast$. Apply the cube lemma (\LemmaRef{cube-lemma}, \cite[Lemma 5.2.6]{hovey}) to conclude that also $A(X') \we C^\ast(X')$.

	This proves $A(X) \we C^\ast(X)$ for any simplicial set with finitely many non-degenerate simplices. We can extend this to simplicial sets of finite dimension by attaching many simplices at once. For this observe that both $A$ and $C^\ast$ send coproducts to products and that cohomology commutes with products:
	$$ H(A(\coprod_\alpha X_\alpha)) \iso H(\prod_\alpha A(X_\alpha)) \iso \prod_\alpha H(A(X_\alpha)), $$
	$$ H(C^\ast(\coprod_\alpha X_\alpha)) \iso H(\prod_\alpha C^\ast(X_\alpha)) \iso \prod_\alpha H(C^\ast(X_\alpha)). $$

	This means that we can extend the previous argument to pushout of this form:
	\begin{displaymath}
		\xymatrix {
		\coprod_{\alpha \in A} \del \Delta[n] \arcof[d] \ar[r] \xypo & X \ar[d] \\
		\coprod_{\alpha \in A} \Delta[n] \ar[r] & X'
		}
	\end{displaymath}

	Finally we can write any simplicial set $X$ as a colimit of finite dimensional ones as:
	$$ sk_0 X \cof sk_1 X \cof sk_2 \cof \dots \colim sk_n X = X, $$
	where $sk_i X$ has no non-degenerate simplices in dimensions $n > i$. By the above $\oint$ gives a quasi isomorphism on all the terms $sk_i X$. So we are in the following situation:
	\begin{displaymath}
		\xymatrix @C=0.3cm{
		A(X) = \lim_i A(sk_i X) \ar[d]^\oint \ar@{-->>}[rr] & & A(sk_2 X) \arfib[r] \arwe[d]^\oint & A(sk_1 X) \arfib[r] \arwe[d]^\oint & A(sk_0 X) \arwe[d]^\oint \\
		C^\ast(X) = \lim_i C^\ast(sk_i X) \ar@{-->>}[rr] & & C^\ast(sk_2 X) \arfib[r] & C^\ast(sk_1 X) \arfib[r] & C^\ast(sk_0 X)
		}
	\end{displaymath}

	We will define long exact sequences for both sequences in the following way. As the following construction is quite general, consider arbitrary cochain algebras $B_i$ as follows:
	\begin{displaymath}
		\xymatrix{
		B = \lim_i B_i \ar@{-->>}[rr] & & B_2 \arfib[r]^-{b_1} & B_1 \arfib[r]^-{b_0} & B_0
		}
	\end{displaymath}
	Define a map $t: \prod_i B_i \to \prod_i B_i$ defined by $t(x_0, x_1, \dots) = (x_0 + b_0(x_1), x_1 + b_1(x_2), \dots)$. Note that $t$ is surjective and that $B \iso \ker(t)$. So we get the following natural short exact sequence and its associated natural long exact sequence in homology:
	$$ 0 \to B \tot{i} \prod_i B_i \tot{t} \prod_i B_i \to 0, $$
	$$ \cdots \tot{\Delta} H^n(B) \tot{i_\ast} H^n(\prod_i B_i) \tot{t_\ast} H^n(\prod_i B_i) \tot{\Delta} \cdots $$

	In our case we get two such long exact sequences with $\oint$ connecting them. As cohomology commutes with products we get isomorphisms on the left and right in the following diagram.
	\begin{displaymath}
		\xymatrix @C=0.3cm{
			\cdots \ar[r] & H^{n-1}(\prod_i A(sk_i X)) \ar[r] \ariso[d]^\oint & H^n(A(X)) \ar[r] \ar[d]^\oint & H^n(\prod_i A(sk_i X)) \ar[r] \ariso[d]^\oint & \cdots \\
			\cdots \ar[r] & H^{n-1}(\prod_i C^\ast(sk_i X)) \ar[r] & H^n(C^\ast(X)) \ar[r] & H^n(\prod_i C^\ast(sk_i X)) \ar[r] & \cdots \\
		}
	\end{displaymath}

	So by the five lemma we can conclude that the middle morphism is an isomorphism as well, proving the isomorphism $H^n(A(X)) \tot{\iso} H^n(C^\ast(X))$ for all $n$. This proves the statement for all $X$.
}





\section{Minimal models}

In this section we will discuss the so called minimal models. These are cdga's with the property that a quasi isomorphism between them is an actual isomorphism.

\begin{definition}
	An cdga $(A, d)$ is a \emph{Sullivan algebra} if
	\begin{itemize}
		\item $(A, d)$ is quasi-free (or semi-free), i.e. $A = \Lambda V$ is free as a cdga, and
		\item $V$ has a filtration $V(0) \subset V(1) \subset \cdots \subset \bigcup{k \in \N} V(k) = V$ such that $d(V(k)) \subset \Lambda V(k-1)$.
	\end{itemize}

	An cdga $(A, d)$ is a \emph{minimal (Sullivan) algebra} if in addition
	\begin{itemize}
		\item $d$ is decomposable, i.e. $\im(d) \subset \Lambda^{\geq 2}V$.
	\end{itemize}
\end{definition}

\begin{definition}
	Let $(A, d)$ be any cdga. A \emph{(minimal) Sullivan model} is a (minimal) Sullivan algebra $(M, d)$ with a weak equivalence:
	$$ (M, d) \we (A, d). $$
\end{definition}

The requirement that there exists a filtration can be replaced by a stronger statement.

\begin{lemma}
	Let $(A, d)$ be a cdga which is $1$-reduced, quasi-free and with a decomposable differential. Then $(A, d)$ is a minimal algebra.
\end{lemma}
\begin{proof}
	Take $V(n) = \bigoplus_{k=0}^n V^k$ (note that $V^0 = v^1 = 0$). Since $d$ is decomposable we see that for $v \in V^n$: $d(v) = x \cdot y$ for some $x, y \in A$. Assuming $dv$ to be non-zero we can compute the degrees:
	$$ \deg{x} + \deg{y} = \deg{xy} = \deg{dv} = \deg{v} + 1 = n + 1. $$
	As $A$ is $1$-reduced we have $\deg{x}, \deg{y} \geq 2$ and so by the above $\deg{x}, \deg{y} \leq n-1$. Conclude that $d(V(k)) \subset \Lambda(V(n-1))$.
\end{proof}


\subsection{Existence}

\begin{theorem}
	Let $(A, d)$ be an $1$-connected cdga, then it has a minimal model.
\end{theorem}
\begin{proof}
	We will construct a sequence of models $m_k: (M(k), d) \to (A, d)$ inductively.
	\begin{itemize}
		\item First define $V(0) = V(1) = 0$ and $m_0 = m_1 = 0$. Then set $V(2) = H^2(A)$ and define a map $m_2: V(2) \to A$ by picking representatives.
		\item Suppose $m_k: (\Lambda V(k), d) \to (A, d)$ is constructed. Choose cocycles $a_\alpha \in A^{k+1}$ and $z_\beta \in (\Lambda V(k))^{k+2}$ such that $H^{k+1}(A) = \im(H^{k+1}(m_k)) \oplus \bigoplus_\alpha \k \cdot [a_\alpha]$ (so $m_k$ together with $a_\alpha$ span $H^{k+1}(A)$) and $\ker(H^{k+2}(m_k)) = \bigoplus_\beta \k \cdot [z_\beta]$. Note that $m_k z_\beta = db_\beta$ for some $b_\beta \in A$.

		Define $V(k+1) = \bigoplus_\alpha \k \cdot v'_\alpha \oplus \bigoplus \k \cdot v''_\beta$ and set $dv'_\alpha = 0$, $dv''_\beta = z_\beta$, $m_k(v'_\alpha) = a_\alpha$ and $m_k(v''_\beta) = b_\beta$.
	\end{itemize}
	This ends the construction. We will prove the following assertion for $k \geq 2$:
	$$ H^i(m_k) \text{ is } \begin{cases}
		\text{an isomorphism} &\text{ if } i \leq k \\
		\text{injective}      &\text{ if } i = k + 1
	\end{cases}. $$
	\TODO{Finish proof: $m_k$ well behaved, above assertion.}
\end{proof}


\subsection{Uniqueness}

Before we state the uniqueness theorem we need some more properties of minimal models.

\begin{lemma}
	Sullivan algebras are cofibrant.
\end{lemma}
\begin{proof}
	Consider the following lifting problem, where $\Lambda V$ is a Sullivan algebra.

	\cimage[scale=0.5]{Sullivan_Lifting}

	By the left adjointness of $\Lambda$ we only have to specify a map $\phi: V \to X$ sucht that $p \circ \phi = g$. We will do this by induction.
	\begin{itemize}
		\item Suppose $\{v_\alpha\}$ is a basis for $V(0)$. Define $V(0) \to X$ by choosing preimages $x_\alpha$ such that $p(x_\alpha) = g(v_\alpha)$ ($p$ is surjective). Define $\phi(v_\alpha) = x_\alpha$.
		\item Suppose $\phi$ has been defined on $V(n)$. Write $V(n+1) = V(n) \oplus V'$ and let $\{v_\alpha\}$ be a basis for $V'$. Then $dv_\alpha \in \Lambda V(n)$, hence $\phi(dv_\alpha)$ is defined and
		$$ d \phi d v_\alpha = \phi d^2 v_\alpha = 0 $$
		$$ p \phi d v_\alpha = g d v_\alpha = d g v_\alpha. $$
		Now $\phi d v_\alpha$ is a cycle and $p \phi d v_\alpha$ is a boundary of $g v_\alpha$. By the following lemma there is a $x_\alpha \in X$ such that $d x_\alpha = \phi d v_\alpha$ and $p x_\alpha = g v_\alpha$. The former property proves that $\phi$ is a chain map, the latter proves the needed commutativity $p \circ \phi = g$.
	\end{itemize}
\end{proof}

\begin{lemma}
	Let $p: X \to Y$ be a trivial fibration, $x \in X$ a cycle, $p(x) \in Y$ a boundary of $y' \in Y$. Then there is a $x' \in X$ such that
	$$ dx' = x \quad\text{ and }\quad px' = y'. $$
\end{lemma}
\begin{proof}
	We have $p^\ast [x] = [px] = 0$, since $p^\ast$ is injective we have $x = d \overline{x}$ for some $\overline{x} \in X$. Now $p \overline{x} = y' + db$ for some $b \in Y$. Choose $a \in X$ with $p a = b$, then define $x' = \overline{x} - da$. Now check the requirements: $p x' = p \overline{x} - p a = y'$ and $d x' = d \overline{x} - d d a = d \overline{x} = x$.
\end{proof}

\begin{lemma}
	Let $f: X \we Y$ be a weak equivalence between cdga's and $M$ a minimal model for $X$. Then $f$ induces an bijection:
	$$ f_\ast: [M, X] \tot{\iso} [M, Y]. $$
\end{lemma}
\begin{proof}
	If $f$ is surjective this follows from the fact that $M$ is cofibrant and $f$ being a trivial fibration (see \cite[lemma 4.9]{dwyer}).

	In general we can construct a cdga $Z$ and trivial fibrations $X \to Z$ and $Y \to Z$ inducing bijections:
	$$ [M, X] \tot{\iso} [M, Z] \toti{\iso} [M, Y], $$
	compatible with $f_\ast$. \cite[Proposition 12.9]{felix}.
\end{proof}

\begin{lemma}
	Let $\phi: (M, d) \we (M', d')$ be a weak equivalence between minimal algebras. Then $\phi$ is an isomorphism.
\end{lemma}
\begin{proof}
	Let $M$ and $M'$ be generated by $V$ and $V'$. Then $\phi$ induces a weak equivalence on the linear part $\phi_0: V \we V'$ \cite[Theorem 1.5.10]{loday}. Since the differentials are decomposable, their linear part vanishes. So we see that $\phi_0: (V, 0) \tot{\iso} (V', 0)$ is an isomorphism.
	Conclude that $\phi = \Lambda \phi_0$ is an isomorphism.
\end{proof}

\begin{theorem}
	Let $m: (M, d) \we (A, d)$ and $m': (M', d') \we (A, d)$ be two minimal models. Then there is an isomorphism $\phi (M, d) \tot{\iso} (M', d')$ such that $m' \circ \phi \eq m$.
\end{theorem}
\begin{proof}
	By the previous lemmas we have $[M', M] \iso [M', A]$. By going from right to elft we get a map $\phi: M' \to M$ such that $m' \circ \phi \eq m$. On homology we get $H(m') \circ H(\phi) = H(m)$, proving that (2-out-of-3) $\phi$ is a weak equivalence. The previous lemma states that $\phi$ is then an isomorphism.
\end{proof}


\chapter{\texorpdfstring{$A$}{A} and \texorpdfstring{$K$}{K} form a Quillen pair}
\label{sec:a-k-quillen-pair}

We will prove that $A$ preserves cofibrations and trivial cofibrations. We only have to check this fact for the generating (trivial) cofibrations in $\sSet$. Note that the contravariance of $A$ means that a (trivial) cofibrations should be sent to a (trivial) fibration.

\begin{lemma}
	$A(i) : A(\Delta[n]) \to A(\del \Delta[n])$ is surjective.
\end{lemma}
\begin{proof}
	Let $\phi \in A(\del \Delta[n])$ be an element of degree $k$, hence it is a map $\del \Delta[n] \to \Apl^k$. We want to extend this to the whole simplex. By the fact that $\Apl^k$ is Kan and contractible we can find a lift $\overline{\phi}$ in the following diagram showing the surjectivity.

	\cimage[scale=0.5]{Extend_Boundary_Form}
\end{proof}

\begin{lemma}
	$A(j) : A(\Delta[n]) \to A(\Lambda^n_k)$ is surjective and a quasi isomorphism.
\end{lemma}
\begin{proof}
	As above we get surjectivity from the Kan condition. To prove that $A(j)$ is a quasi isomorphism we pass to the singular cochain complex and use that $C^\ast(j) : C^\ast(\Delta[n]) \we C^\ast(\Lambda^n_k)$ is a quasi isomorphism. Consider the following diagram and conclude that $A(j)$ is surjective and a quasi isomorphism.

	\cimage[scale=0.5]{A_Preserves_WCof}
\end{proof}

Since $A$ is a left adjoint, it preserves all colimits and by functoriality it preserves retracts. From this we can conclude the following corollary.

\begin{corollary}
	$A$ preserves all cofibrations and all trivial cofibrations and hence is a left Quillen functor.
\end{corollary}

\begin{corollary}
	$A$ and $K$ induce an adjunction on the homotopy categories:
	$$ \Ho(\sSet) \leftadj \opCat{\Ho(\CDGA)}. $$
\end{corollary}


\section{Homotopy groups of \texorpdfstring{$K(A)$}{K(A)}}
We are after an equivalence of homotopy categories, so it is natural to ask what the homotopy groups of $K(A)$ are for a cdga $A$. In order to do so, we will define homotopy groups of cdga's directly and compare the two notions.

Recall that an augmented cdga is a cdga $A$ with an algebra map $A \tot{\counit} \k$ such that $\counit \unit = \id$.

\Definition{cdga-homotopy-groups}{
	Define the \Def{augmentation ideal} of $A$ as $\overline{A} = \ker \counit$. Define the \Def{cochain complex of indecomposables} of $A$ as $QA = \overline{A} / \overline{A} \cdot \overline{A}$.

	Now define the \Def{homotopy groups of a cdga} $A$ as
	$$ \pi^i(A) = H^i(QA). $$
}

Note that for a free cdga $\Lambda C$ there is a natural augmentation and the chain complexes of indecomposables $Q \Lambda C$ is naturally isomorphic to $C$. Consider the augmented cdga $V(n) = D(n) \oplus \k$, with trivial multiplication and where the term $\k$ is used for the unit and augmentation. There is a weak equivalence $A(n) \to V(n)$ (recall \DefinitionRef{minimal-model-sphere}).

\Lemma{cdga-dual-homotopy-groups}{
	Let $A$ be an augmented cdga, then
	$$ [A, V(n)] \tot{\iso} \Hom_\k(\pi^n(A), \k). $$
}

We will denote the dual of a vector space as $V^\ast = \Hom_\k(V, \k)$.

\Theorem{cdga-dual-homotopy-groups}{
	Let $X$ be a cofibrant augmented cdga, then
	$$ \pi_n(KX) \iso \pi^n(X)^\ast. $$
}
\Proof{
	First note that $KX$ is a Kan complex (because it is a simplicial group). Using the homotopy adjunction and the lemma above we get:
	\begin{align*}
		\pi_n(KX) &= [S^n, KX]      \\
		          &\iso [X, A(S^n)] \\
		          &\iso [X, A(n)]   \\
		          &\iso [X, V(n)]   \\
		          &\iso \pi^n(X)^\ast
	\end{align*}
	\todo{Prove all isomorphisms.}
	\todo{Group structure?}
}

We get a particularly nice result for minimal cdga's, because the functor $Q$ is the left inverse of the functor $\Lambda$ and the differential is decomposable.

\Corollary{minimal-cdga-homotopy-groups}{
	For a minimal cdga $X = \Lambda V$ we get
	$$ \pi_n(KX) = {V^n}^\ast. $$
}

\Corollary{minimal-cdga-EM-space}{
	For a cdga with one generator $X = \Lambda(v)$ with $d v = 0$ and $\deg{v} = n$. We conclude that $KX$ is a $K(\k^\ast, n)$-space.
}


\section{Equivalence on rational spaces}
For the equivalence of rational spaces and cdga's we need that the unit and counit of the adjunction are in fact weak equivalences. More formally we want the following maps to be weak equivalences:
$$ X \to K(A(X)) \text{ for any rational space $X \in \sSet$ of finite type}, $$
$$ A \to A(K(A)) \text{ for any $A \in \CDGA_\Q$ of finite type}. $$

We need the assumption of finiteness because we are dualizing vector spaces.

