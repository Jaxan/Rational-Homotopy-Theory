
% clickable tocs
\usepackage{hyperref}

% floating figures
\usepackage{float}

\usepackage{listings}

\usepackage{tikz}
\usetikzlibrary{matrix, arrows, decorations}
\tikzset{node distance=2.5em, row sep=2.2em, column sep=2.7em, auto}

\usepackage{graphicx}
\graphicspath{ {./images/} }
\usepackage{caption}
\usepackage{subcaption}

% Matrices have a upper bound for its size
\setcounter{MaxMatrixCols}{20}

% Remove trailing `contents` after toc
\renewcommand{\contentsname}{}

% for the fib arrow
\usepackage{amssymb}

% mathbb for lowercase
\usepackage{bbm}

% for slanted text/symbols
\usepackage{slantsc}

\DeclareMathOperator*{\colim}{colim}
\DeclareMathOperator*{\tensor}{\otimes}
\DeclareMathOperator*{\bigtensor}{\bigotimes}

\newcommand{\N}{\mathbb{N}}
\newcommand{\Np}{{\mathbb{N}^{>0}}}
\newcommand{\Z}{\mathbb{Z}}
\newcommand{\R}{\mathbb{R}}
\renewcommand{\k}{\mathbbm{k}}

\newcommand{\cat}[1]{\mathbf{#1}}
\newcommand{\Set}{\cat{Set}}
\newcommand{\sSet}{\cat{sSet}}
\newcommand{\Top}{\cat{Top}}
\newcommand{\DELTA}{\cat{\Delta}}
\newcommand{\grMod}[1]{\cat{gr-{#1}Mod}}
\newcommand{\grAlg}[1]{\cat{gr-{#1}Alg}}

\newcommand{\Hom}[3]{\mathbf{Hom}_{#1}(#2, #3)}
\newcommand{\id}{\mathbf{id}}

\newcommand{\I}{\,\mid\,}
\newcommand{\del}{\partial}				% boundary
\newcommand{\iso}{\cong}				% isomorphic
\newcommand{\eq}{\sim}					% homotopic
\newcommand{\tot}[1]{\xrightarrow{\,\,{#1}\,\,}}	% arrow with name
\newcommand{\mapstot}[1]{\xmapsto{\,\,{#1}\,\,}}	% mapsto with name
\newcommand{\cof}{\hookrightarrow}		% cofibration
\newcommand{\fib}{\twoheadrightarrow}	% fibration
\newcommand{\we}{\tot{\simeq}}			% weak equivalence
\renewcommand{\deg}[1]{|{#1}|}

\newcommand\restr[2]{{% we make the whole thing an ordinary symbol
  \left.\kern-\nulldelimiterspace % automatically resize the bar with \right
  #1 % the function
  \vphantom{\big|} % pretend it's a little taller at normal size
  \right|_{#2} % this is the delimiter
  }}

\newcommand{\todo}[1]{
	\addcontentsline{tdo}{todo}{\protect{#1}}
	$\ast$ \marginpar{\tiny $\ast$ #1}
}

\theoremstyle{plain}
\newtheorem{theorem}{Theorem}[section]
\newtheorem{proposition}[theorem]{Proposition}
\newtheorem{lemma}[theorem]{Lemma}
\newtheorem{corollary}[theorem]{Corollary}

\theoremstyle{definition}
\newtheorem{definition}[theorem]{Definition}
\newtheorem{example}[theorem]{Example}

\newcommand*{\thead}[1]{\multicolumn{1}{c}{\bfseries #1}}
