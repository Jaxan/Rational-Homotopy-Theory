
\chapter{The main equivalence}

In this section we aim to prove that the homotopy theory of rational spaces is the same as the homotopy theory of cdga's over $\Q$. Before we prove the equivalence, we will show that $A$ and $K$ form a Quillen pair. This already provides an adjunction between the homotopy categories. Besides the equivalence of the homotopy categories we will also investigate homotopy groups on a cdga directly. The homotopy groups of a space will be dual to the homotopy groups of the associated cdga.

We will prove that $A$ preserves cofibrations and trivial cofibrations. We only have to check this fact for the generating (trivial) cofibrations in $\sSet$. Note that the contravariance of $A$ means that a (trivial) cofibrations should be sent to a (trivial) fibration.

\begin{lemma}
	$A(i) : A(\Delta[n]) \to A(\del \Delta[n])$ is surjective.
\end{lemma}
\begin{proof}
	Let $\phi \in A(\del \Delta[n])$ be an element of degree $k$, hence it is a map $\del \Delta[n] \to \Apl^k$. We want to extend this to the whole simplex. By the fact that $\Apl^k$ is Kan and contractible we can find a lift $\overline{\phi}$ in the following diagram showing the surjectivity.
	\begin{displaymath}
		\xymatrix {
		\del \Delta[n] \ar[r]^\phi \arcof[d]^i & \Apl^k \\
		\Delta[n] \ar@{-->}[ru]_{\overline{\phi}}
		}
	\end{displaymath}
\end{proof}

\begin{lemma}
	$A(j) : A(\Delta[n]) \to A(\Lambda^k_n)$ is surjective and a quasi isomorphism.
\end{lemma}
\begin{proof}
	As above we get surjectivity from the Kan condition. To prove that $A(j)$ is a quasi isomorphism we pass to the singular cochain complex and use that $C^\ast(j) : C^\ast(\Delta[n]) \we C^\ast(\Lambda^n_k)$ is a quasi isomorphism. Consider the following diagram and conclude that $A(j)$ is surjective and a quasi isomorphism.
	\begin{displaymath}
		\xymatrix {
		A(\Delta[n]) \ar[r]^{A(j)} \arwe[d]^\oint & A(\Lambda^k_n) \arwe[d]^\oint \\
		C^\ast(\Delta[n]) \ar[r]^{C^\ast(j)} & C^\ast(\Lambda^k_n)
		}
	\end{displaymath}
\end{proof}

Since $A$ is a left adjoint, it preserves all colimits and by functoriality it preserves retracts. From this we can conclude the following corollary.

\begin{corollary}
	$A$ preserves all cofibrations and all trivial cofibrations and hence is a left Quillen functor.
\end{corollary}

\begin{corollary}
	$A$ and $K$ induce an adjunction on the homotopy categories:
	$$ LA : \Ho(\sSet) \leftadj \opCat{\Ho(\CDGA)} : RK. $$
\end{corollary}

The induced adjunction in the previous corollary is given by $LA(X) = A(X)$ for $X \in \sSet$ (note that every simplicial set is already cofibrant) and $RK(Y) = K(Y^{cof})$ for $Y \in \CDGA$. By the use of minimal models, and in particular the functor $M$. We get the following adjunction between $1$-connected objects:

\Corollary{minimal-model-adjunction}{
	There is an adjunction:
	$$ M : \Ho(\sSet_1) \leftadj \opCat{\Ho(\text{Minimal models}^1)} : RK, $$
	where $M$ is given by $M(X) = M(A(X))$ and $RK$ is given by $RK(Y) = K(Y)$ (because minimal models are always cofibrant).
}


\section{Homotopy groups of cdga's}
We are after an equivalence of homotopy categories, so it is natural to ask what the homotopy groups of $K(A)$ are for a cdga $A$. In order to do so, we will define homotopy groups of cdga's directly and compare the two notions.

Recall that an augmented cdga is a cdga $A$ with an algebra map $A \tot{\counit} \k$ such that $\counit \unit = \id$.

\Definition{cdga-homotopy-groups}{
	Define the \Def{augmentation ideal} of $A$ as $\overline{A} = \ker \counit$. Define the \Def{cochain complex of indecomposables} of $A$ as $QA = \overline{A} / \overline{A} \cdot \overline{A}$.

	Now define the \Def{homotopy groups of a cdga} $A$ as
	$$ \pi^i(A) = H^i(QA). $$
}

Note that for a free cdga $\Lambda C$ there is a natural augmentation and the chain complexes of indecomposables $Q \Lambda C$ is naturally isomorphic to $C$. Consider the augmented cdga $V(n) = D(n) \oplus \k$, with trivial multiplication and where the term $\k$ is used for the unit and augmentation. There is a weak equivalence $A(n) \to V(n)$ (recall \DefinitionRef{minimal-model-sphere}).

\Lemma{cdga-dual-homotopy-groups}{
	Let $A$ be an augmented cdga, then
	$$ [A, V(n)] \tot{\iso} \Hom_\k(\pi^n(A), \k). $$
}
\todo{prove}

We will denote the dual of a vector space as $V^\ast = \Hom_\k(V, \k)$.

\Theorem{cdga-dual-homotopy-groups}{
	Let $X$ be a cofibrant augmented cdga, then
	$$ \pi_n(KX) \iso \pi^n(X)^\ast. $$
}
\Proof{
	First note that $KX$ is a Kan complex (because it is a simplicial group). Using the homotopy adjunction and the lemma above we get:
	\begin{align*}
		\pi_n(KX) &= [S^n, KX]      \\
		          &\iso [X, A(S^n)] \\
		          &\iso [X, A(n)]   \\
		          &\iso [X, V(n)]   \\
		          &\iso \pi^n(X)^\ast
	\end{align*}
	\todo{Prove all isomorphisms.}
	\todo{Group structure?}
}

We get a particularly nice result for minimal cdga's, because the functor $Q$ is the left inverse of the functor $\Lambda$ and the differential is decomposable.

\Corollary{minimal-cdga-homotopy-groups}{
	For a minimal cdga $X = \Lambda V$ we get
	$$ \pi_n(KX) = {V^n}^\ast. $$
}

\Corollary{minimal-cdga-EM-space}{
	For a cdga with one generator $X = \Lambda(v)$ with $d v = 0$ and $\deg{v} = n$. We conclude that $KX$ is a $K(\k^\ast, n)$-space.
}


\section{Equivalence on rational spaces}
For the equivalence of rational spaces and cdga's we need that the unit and counit of the adjunction in \CorollaryRef{minimal-model-adjunction} are in fact weak equivalences for rational spaces. More formally: for any (automatically cofibrant) $X \in \sSet$ and any minimal model $A \in \CDGA$, both rational, $1$-connected and of finite type, the following two natural maps are weak equivalences:
\begin{align*}
	X &\to K(M(X)) \\
	A &\to M(K(A))
\end{align*}
where the first of the two maps is given by the composition $X \to K(A(X)) \tot{K(m_X)} K(M(X))$,
and the second map is obtained by the map $A \to A(K(A))$ and using the bijection from \LemmaRef{minimal-model-bijection}: $[A, A(K(A))] \iso [A, M(K(A))]$. By the 2-out-of-3 property the map $A \to M(K(A))$ is a weak equivalence if and only if the ordinary unit $A \to A(K(A))$ is a weak equivalence.

\Lemma{}{
	(Base case) Let $A = (\Lambda(v), 0)$ be a minimal model with one generator of degree $\deg{v} = n \geq 1$. Then $A \we A(K(A))$.
}
\Proof{
	By \CorollaryRef{minimal-cdga-EM-space} we know that $K(A)$ is an Eilenberg-MacLane space of type $K(\Q^\ast, n)$. The cohomology of an Eilenberg-MacLane space with coefficients in $\Q$ is known:
	$$ H^\ast(K(\Q^\ast, n); \Q) = \Q[x], $$
	that is, the free commutative graded algebra with one generator $x$. This can be calculated, for example, with spectral sequences \cite{griffiths}.

	Now choose a cycle $z \in A(K(\Q^\ast, n))$ representing the class $x$ and define a map $A \to A(K(A))$ by sending the generator $v$ to $z$. This induces an isomorphism on cohomology. So $A$ is the minimal model for $A(K(A))$.
}

\Lemma{}{
	(Induction step) Let $A$ be a cofibrant, connected algebra. Let $B$ be the pushout in the following square, where $m \geq 1$:
	\begin{displaymath}
		\xymatrix{
		S(m+1) \arcof[d] \ar[r] \xypo & A \arcof[d] \\
		T(m) \ar[r] & B
		}
	\end{displaymath}
	Then if $A \to A(K(A))$ is a weak equivalence, so is $B \to A(K(B))$
}
\Proof{
	Applying $K$ to the above diagram gives a pullback diagram of simplicial sets, where the induced vertical maps are fibrations (since $K$ is right Quillen). In other words, the induced square is a homotopy pullback.

	Applying $A$ again gives the following cube of cdga's:
	\begin{displaymath}
		\xymatrix @=9pt{
		S(m+1) \arcof[dd] \ar[rr] \arwe[rd] \xypo & & A \arcof'[d][dd] \arwe[rd] & \\
		& A(K(S(m+1))) \ar[dd] \ar[rr] & & A(K(A)) \ar[dd] \\
		T(m) \ar'[r][rr] \arwe[rd] & & B \ar[rd] & \\
		& A(K(T(m))) \ar[rr] & & A(K(B))
		}
	\end{displaymath}
	Note that we have a weak equivalence in the top left corner, by the base case ($S(m+1) = (\Lambda(v), 0)$). The weak equivalence in the top right is by assumption. Finally the bottom left map is a weak equivalence because both cdga's are acyclic.

	To conclude that $B \to A(K(B))$ is a weak equivalence, we wish to prove that the front face of the cube is a homotopy pushout, as the back face clearly is one. This is a consequence of the Eilenberg-Moore spectral sequence \cite{mccleary}.
}

Now we wish to use the previous lemma as an induction step for minimal models. Let $(\Lambda V, d)$ be some minimal algebra. Write $V(n+1) = V(n) \oplus V'$ and let $v \in V'$ of degree $\deg{v} = k$, then the minimal algebra $(\Lambda (V(n) \oplus \Q \cdot v), d)$ is the pushout in the following diagram, where $f$ sends the generator $c$ to $dv$.
\begin{displaymath}
	\xymatrix{
	S(k) \arcof[d] \ar[r]^f \xypo & (\Lambda V(n), d) \ar[d] \\
	T(k-1) \ar[r] & (\Lambda (V(n) \oplus \Q \cdot v), d)
	}
\end{displaymath}
In particular if the vector space $V'$ is finitely generated, we can repeat this procedure for all basis elements (it does not matter in what order we do so, as $dv \in \Lambda V(n)$). So in this case, if $(\Lambda V(n), d) \to A(K(\Lambda V(n), d))$ is a weak equivalence, so is $(\Lambda V(n+1), d) \to A(K(\Lambda V(n+1), d))$

\Corollary{}{
	Let $(\Lambda V, d)$ be a $1$-connected minimal algebra with $V^i$ finite dimensional for all $i$. Then $(\Lambda V, d) \to A(K(\Lambda V, d))$ is a weak equivalence.
}
\Proof{
	Note that if we want to prove the isomorphism $H^i(\Lambda V, d) \to H^i(A(K(\Lambda V, d)))$ it is enough to prove that $H^i(\Lambda V^{\leq i}, d) \to H^i(A(K(\Lambda V^{\leq i}, d)))$ is an isomorphism (as the elements of higher degree do not change the isomorphism). By the $1$-connectedness we can choose our filtration to respect the degree by \LemmaRef{1-reduced-minimal-model}.

	Now $V(n)$ is finitely generated for all $n$ by assumption. By the inductive procedure above we see that $(\Lambda V(n), d) \to A(K(\Lambda V(n), d))$ is a weak equivalence for all $n$. Hence $(\Lambda V, d) \to A(K(\Lambda V, d))$ is a weak equivalence.
}

\todo{$X \to K(A(X))$}

We have proven the following theorem.
\Theorem{main-theorem}{
	The functors $A$ and $K$ induce an equivalence of homotopy categories, when restricted to rational, $1$-connected objects of finite type. more formally, we have:
	$$ \Ho(\sSet_1^{\Q,f}) \iso \Ho(\CDGA_{\Q,1,f}). $$

	Furthermore, for any $1$-connected space $X$ of finite type, we have the following isomorphism of groups:
	$$ \pi_i(X) \tensor \Q \iso {V^i}^\ast, $$
	where $(\Lambda V, d)$ is the minimal model of $A(X)$.
}
