
\chapter{Rational homotopy theory}
\label{sec:basics}

In this section we will state the aim of rational homotopy theory. Moreover we will recall classical theorems from algebraic topology and deduce rational versions of them.

In the following definition \emph{space} is to be understood as a topological space or a simplicial set.

\Definition{rational-space}{
	A $0$-connected space $X$ with abelian fundamental group is a \emph{rational space} if
	$$ \pi_i(X) \text{ is a $\Q$-vector space } \quad \forall i > 0. $$
}

\Definition{rational-homotopy-groups}{
	We define the \emph{rational homotopy groups} of a $0$-connected space $X$ with abelian fundamental group as:
	$$ \pi_i(X) \tensor \Q \quad \forall i > 0.$$
}

In order to define the tensor product $\pi_1(X) \tensor \Q$ we need that the fundamental group is abelian, the higher homotopy groups are always abelian. There is a more general approach using \Def{nilpotent groups}, which admit $\Q$-completions \cite{bousfield}. Since this is rather technical we will often restrict ourselves to such spaces or even simply connected spaces.

Note that for a rational space $X$, the ordinary homotopy groups are isomorphic to the rational homotopy groups, i.e. $\pi_i(X) \tensor \Q \iso \pi_i(X)$.

\Definition{rational-homotopy-equivalence}{
	A map $f: X \to Y$ is a \emph{rational homotopy equivalence} if $\pi_i(f) \tensor \Q$ is a linear isomorphism for all $i > 0$.
}

\Definition{rationalization}{
	A map $f: X \to X_0$ is a \emph{rationalization} if $X_0$ is rational and $f$ is a rational homotopy equivalence.
}

Note that a weak equivalence (and hence also a homotopy equivalence) is always a rational homotopy theory. Furthermore if $f: X \to Y$ is a map between rational spaces, then $f$ is a rational homotopy equivalence if and only if $f$ is a weak equivalence.

The theory of rational homotopy is the study of spaces with rational equivalences. Quillen defines a model structure on simply connected simplicial sets with rational equivalences as weak equivalences \cite{quillen}. This means that there is a homotopy category $\Ho^\Q(\sSet_1)$. However we will later prove that every simply connected space has a rationalization, so that $\Ho^\Q(\sSet_1) = \Ho(\sSet^\Q_1)$ are equivalent categories. This means that we do not need the model structure defined by Quillen, but we can simply restrict ourselves to rational spaces (with ordinary weak equivalences).


\section{Classical results from algebraic topology}

We will now recall known results from algebraic topology, without proof. One can find many of these results in basic text books, such as \cite{may, dold}.

\Theorem{relative-hurewicz}{
	(Relative Hurewicz) For any inclusion of spaces $Y \subset X$ and all $i > 0$, there is a natural map
	$$ h_i : \pi_i(X, Y) \to H_i(X, Y). $$
	If furthermore $(X, Y)$ is $n$-connected ($n > 0$), then the map $h_i$ is an isomorphism for all $i \leq n + 1$.
}

\Theorem{serre-les}{
	(Long exact sequence) Let $f: X \to Y$ be a Serre fibration, then there is a long exact sequence:
	$$ \cdots \tot{\del} \pi_i(F) \tot{i_\ast} \pi_i(X) \tot{f_\ast} \pi_i(Y) \tot{\del} \cdots \to \pi_0(Y) \to \ast, $$
	where $F$ is the fiber of $f$.
}

Using an inductive argument and the previous two theorems, one can show the following theorem (as for example shown in \cite{griffiths}).
\Theorem{whitehead-homology}{
	(Whitehead) For any map $f: X \to Y$ between $1$-connected spaces, $ \pi_i(f) $ is an isomorphism $\forall 0 < i < r$ if and only if $H_i(f)$ is an isomorphism $\forall 0 < i < r$.
	In particular we see that $f$ is a weak equivalence if and only if it induces an isomorphism on homology.
}

The following two theorems can be found in textbooks about homological algebra such as \cite{weibel, rotman}. Note that when the degrees are left out, $H(X; A)$ denotes the graded homology module with coefficients in $A$.

\Theorem{universal-coefficient}{
	(Universal Coefficient Theorem)
	For any space $X$ and abelian group $A$, there are natural short exact sequences
	$$ 0 \to H_n(X) \tensor A \to H_n(X; A) \to \Tor(H_{n-1}(X), A) \to 0, $$
	$$ 0 \to \Ext(H_{n-1}(X), A) \to H^n(X; A) \to \Hom(H_n(X), A) \to 0. $$
}

\Theorem{kunneth}{
	(Künneth Theorem)
	For spaces $X$ and $Y$, there is a short exact sequence
	\[ \footnotesize \xymatrix @C=0.3cm{
		0 \ar[r] & H(X; A) \tensor H(Y; A) \ar[r] & H(X \times Y; A) \ar[r] & \Tor_{\ast-1}(H(X; A), H(Y; A)) \ar[r] & 0
	},\]
	where $H(X; A)$ and $H(X; A)$ are considered as graded modules and their tensor product and torsion groups are graded.
}

\section{Consequences for rational homotopy theory}

The latter two theorems have a direct consequence for rational homotopy theory. By taking $A = \Q$ we see that the torsion groups vanish. We have the immediate corollary.

\Corollary{rational-corollaries}{
	We have the following natural isomorphisms in rational homology, and we can relate rational cohomology naturally to rational homology
	\begin{align*}
		H_\ast(X) \tensor \Q &\tot{\iso} H_\ast(X; \Q), \\
		H_\ast(X; \Q) \tensor H_\ast(Y; \Q) &\tot{\iso} H_\ast(X \times Y; \Q), \\
		H^\ast(X; \Q) &\tot{\iso} \Hom(H_\ast(X); \Q). 
	\end{align*}

}

The long exact sequence for a Serre fibration also has a direct consequence for rational homotopy theory.
\Corollary{rational-les}{
	Let $f: X \to Y$ be a Serre fibration with fiber $F$, all $0$-connected with abelian fundamental group, then there is a natural long exact sequence of rational homotopy groups:
	$$ \cdots \tot{\del} \pi_i(F) \tensor \Q \tot{i_\ast} \pi_i(X) \tensor \Q \tot{f_\ast} \pi_i(Y) \tensor \Q \tot{\del} \cdots. $$
}

In the next sections we will prove the rational Hurewicz and rational Whitehead theorems. These theorems are due to Serre \cite{serre}.

