
\section{Rational homotopy theory}
\label{sec:rational}

In this section we will state the aim of rational homotopy theory. Moreover we will recall classical theorems from algebraic topology and deduce rational versions of them.

In the following definition \emph{space} is to be understood as a topological space or a simplicial set. We will restrict to simply connected spaces.

\Definition{rational-space}{
	A space $X$ is a \emph{rational space} if
	$$ \pi_i(X) \text{ is a $\Q$-vectorspace } \quad\forall i > 0. $$
}

\Definition{rational-homotopy-groups}{
	We define the \emph{rational homotopy groups} of a space $X$ as:
	$$ \pi_i(X) \tensor \Q \quad \forall i > 0.$$
}

Note that for a rational space $X$, the homotopy groups are isomorphic to the rational homotopy groups, i.e. $\pi_i(X) \tensor \Q \iso \pi_i(X)$.

\Definition{rational-homotopy-equivalence}{
	A map $f: X \to Y$ is a \emph{rational homotopy equivalence} if $\pi_i(f) \tensor \Q$ is a linear isomorphism for all $i > 0$.
}

\Definition{rationalization}{
	A map $f: X \to X_0$ is a \emph{rationalization} if $X_0$ is rational and $f$ is a rational homotopy equivalence.
}

Note that a weak equivalence (and hence also a homotopy equivalence) is always a rational homotopy theory. Furthermore if $f: X \to Y$ is a map between rational spaces, then $f$ is a rational homotopy equivalence iff $f$ is a weak equivalence.

We will later see that any space admits a rationalization. The theory of rational homotopy theory is then the study of the homotopy category $\Ho_\Q(\Top) \iso \Ho(\Top_\Q)$, which is on its own turn equivalent to $\Ho(\sSet_\Q) \iso \Ho_\Q(\sSet)$.

\subsection{Classical results from algebraic topology}

We will now recall known results from algebraic topology, without proof. One can find many of these results in basic text books, such as [May, Dold, ...]. Note that all spaces are assumed to be $1$-connected.

\Theorem{relative-hurewicz}{
	(Relative Hurewicz) For any inclusion of spaces $A \subset X$ and all $i > 0$, there is a natural map
	$$ h_i : \pi_i(X, A) \to H_i(X, A). $$
	If furhtermore $(X,A)$ is $n$-connected, then the map $h_i$ is an isomorphism for all $i \leq n + 1$
}

\Theorem{serre-les}{
	(Long exact sequence) Let $f: X \to Y$ be a Serre fibration, then there is a long exact sequence:
	$$ \cdots \tot{\del} \pi_i(F) \tot{i_\ast} \pi_i(X) \tot{f_\ast} \pi_i(Y) \tot{\del} \cdots \to \pi_0(Y) \to \ast, $$
	where $F$ is the fibre of $f$.
}

Using an inductive argument and the previous two theorems, one can show the following theorem (as for example shown in \cite{griffith}).
\Theorem{whitehead-homology}{
	(Whitehead) For any map $f: X \to Y$ we have
	$$ \pi_i(f) \text{ is an isomorphism } \forall 0 < i < r \iff H_i(f) \text{ is an isomorphism } \forall 0 < i < r. $$
	In particular we see that $f$ is a weak equivalence iff it induces an isomorphism on homology.
}

The following two theorems can be found in textbooks about homological algebra, such as [Weibel].
\Theorem{universal-coefficient}{
	(Universal Coefficient Theorem)
	For any space $X$ and abelian group $A$, there are natural short exact sequcenes
	$$ 0 \to H_n(X) \tensor A \to H_n(X; A) \to \Tor(H_{n-1}(X), A) \to 0, $$
	$$ 0 \to \Ext(H_{n-1}(X), A) \to H^n(X; A) \to \Hom(H_n(X), A) \to 0. $$
}

\Theorem{kunneth}{
	(Künneth Theorem)
	For spaces $X$ and $Y$, there is a short exact sequence
	$$ 0 \to H(X; A) \tensor H(Y; A) \to H(X \times Y; A) \to \Tor(H(X; A), H(Y; A)) \to 0, $$
	where $H(X; A)$ and $H(X; A)$ are considered as graded modules and their tensor product and torsion groups are graded.
}

\subsection{Immediate results for rational homotopy theory}

The latter two theorems have a direct consequence for rational homotopy theory. By taking $A = \Q$ we see that the torsion groups vanish. We have the immediate corollary.

\Corollary{rational-corollaries}{
	We have the following natural isomorphisms
	$$ H(X) \tensor \Q \tot{\iso} H(X; \Q), $$
	$$ H^n(X; \Q) \tot{\iso} \Hom(H(X); \Q), $$
	$$ H(X \times Y) \tot{\iso} H(X) \tensor H(Y). $$
}

The long exact sequence for a Serre fibration also has a direct consequence for rational homotopy theory.
\Corollary{rational-les}{
	Let $f: X \to Y$ be a Serre fibration, then there is a natural long exact sequence of rational homotopy groups:
	$$ \cdots \tot{\del} \pi_i(F) \tensor \Q \tot{i_\ast} \pi_i(X) \tensor \Q \tot{f_\ast} \pi_i(Y) \tensor \Q \tot{\del} \cdots, $$
}

In the next sections we will prove the rational Hurewicz and rational Whitehead theorems. These theorems are due to Serre [Serre].

