
\Chapter{Homotopy Theory For cdga's}{HomotopyTheoryCDGA}

Recall that a cdga $A$ is a commutative differential graded algebra, meaning that
\begin{itemize}\itemsep0em
	\item it has a grading: $A = \bigoplus_{n\in\N} A^n$,
	\item it has a differential: $d: A \to A$ with $d^2 = 0$,
	\item it has a multiplication: $\mu: A \tensor A \to A$ which is associative and unital and
	\item it is commutative: $x y = (-1)^{\deg{x}\cdot\deg{y}} y x$.
\end{itemize}
And all of the above structure is compatible with each other (e.g. the differential is a derivation of degree $1$, the maps are graded, \dots). The exact requirements are stated in the appendix on algebra. An algebra $A$ is augmented if it has a specified map (of algebras) $A \tot{\counit} \k$. Furthermore we adopt the notation $A^{\leq n} = \bigoplus_{k \leq n} A^k$ and similarly for $\geq n$.

There is a left adjoint $\Lambda$ to the forgetful functor $U$ which assigns the free graded commutative algebras $\Lambda V$ to a graded module $V$. This extends to an adjunction (also called $\Lambda$ and $U$) between commutative differential graded algebras and differential graded modules. We denote the subspace of elements of wordlength $n$ by $\Lambda^n V$ (note that this has nothing to do with the grading on $V$).

In homological algebra we are especially interested in \emph{quasi isomorphisms}, i.e. maps $f: A \to B$ inducing an isomorphism on cohomology: $H(f): HA \iso HB$. This notions makes sense for any object with a differential.

We furthermore have the following categorical properties of cdga's:
\begin{itemize}\itemsep0em
	\item The finite coproduct in $\CDGA_\k$ is the (graded) tensor product.
	\item The finite product in $\CDGA_\k$ is the cartesian product (with pointwise operations).
	\item The equalizer (resp. coequalizer) of $f$ and $g$ is given by the kernel (resp. cokernel) of $f - g$. Together with the (co)products this defines pullbacks and pushouts.
	\item $\k$ and $0$ are the initial and final object.
\end{itemize}

\section{Cochain models for the $n$-disk and $n$-sphere}

We will first define some basic cochain complexes which model the $n$-disk and $n$-sphere. $D(n)$ is the cochain complex generated by one element $b \in D(n)^n$ and its differential $c = d(b) \in D(n)^{n+1}$. $S(n)$ is the cochain complex generated by one element $a \in S(n)^n$ which differential vanishes (i.e. $da = 0$). In other words:

$$ D(n) = ... \to 0 \to \k \to \k \to 0 \to ... $$
$$ S(n) = ... \to 0 \to \k \to 0 \to 0 \to ... $$

Note that $D(n)$ is acyclic for all $n$, or put in different words: $j_n : 0 \to D(n)$ is a quasi isomorphism. The sphere $S(n)$ has exactly one non-trivial cohomology group $H^n(S(n)) = \k \cdot [a]$. There is an injective function $i_n : S(n+1) \to D(n)$, sending $a$ to $c$. The maps $j_n$ and $i_n$ play the following important role in the model structure of cochain complexes:

\begin{claim}
	The set $I = \{i_n : S(n+1) \to D(n) \I n \in \N\}$ generates all cofibrations and the set $J = \{j_n : 0 \to D(n) \I n \in \N\}$ generates all trivial cofibrations.
\end{claim}

The proof is omitted but can be found in different sources \todo{Cite sources}. In the next section we will prove a similar result for cdga's, so the reader can also refer to that proof.

$S(n)$ plays a another special role: maps from $S(n)$ to some cochain complex $X$ correspond directly to elements in the kernel of $\restr{d}{X^n}$. Any such map is null-homotopic precisely when the corresponding elements in the kernel is a coboundary. So there is a natural isomorphism: $\Hom(S(n), X) / ~ \iso H^n(X)$. So the cohomology groups can be considered as honest homotopy groups.

By using the free cdga functor we can turn these cochain complexes into cdga's $\Lambda(D(n))$ and $\Lambda(S(n))$. So $\Lambda(D(n))$ consists of linear combinations of $b^n$ and $c b^n$ when $n$ is even, and $c^n b$ and $c^n$ when $n$ is odd. In both cases we can compute the differentials using the Leibniz rule:
$$ d(b^n) = n \cdot c b^{n-1} $$
$$ d(c b^n) = 0 $$

$$ d(c^n b) = c^{n+1} $$
$$ d(c^n) = 0 $$

Those cocycles are in fact coboundaries (remember that $\k$ is a field of characteristic $0$):
$$ c b^n = \frac{1}{n} d(b^{n+1}) $$
$$ c^n = d(b c^{n-1}) $$

There are no additional cocycles in $\Lambda(D(n))$ besides the constants and $c$. So we conclude that $\Lambda(D(n))$ is acyclic as an algebra. In other words $\Lambda(j_n): \k \to \Lambda D(n)$ is a quasi isomorphism.

The situation for $\Lambda S(n)$ is easier: when $n$ is even it is given by polynomials in $a$, if $n$ is odd it is an exterior algebra (i.e. $a^2 = 0$). Again the sets $\Lambda(I) = \{ \Lambda(i_n) : \Lambda S(n+1) \to \Lambda D(n) \I n \in \N\}$ and $\Lambda(J) = \{ \Lambda(j_n) : \k \to \Lambda D(n) \I n \in \N\}$ play an important role.

\begin{theorem}
	The sets $\Lambda(I)$ and $\Lambda(J)$ generate a model structure on $\CDGA_\k$ where:
	\begin{itemize}
		\item weak equivalences are quasi isomorphisms,
		\item fibrations are (degree wise) surjective maps and
		\item cofibrations are maps with the left lifting property against trivial fibrations.
	\end{itemize}
\end{theorem}

We will prove this theorem in the next section. Note that the functors $\Lambda$ and $U$ thus form a Quillen pair with this model structure.

\subsubsection{Why we need $\Char{\k} = 0$ for algebras}
The above Quillen pair $(\Lambda, U)$ fails to be a Quillen pair if $\Char{\k} = p \neq 0$. We will show this by proving that the maps $\Lambda(j_n)$ are not weak equivalences for even $n$. Consider $b^p \in D(n)$, then by the Leibniz rule:
$$ d(b^p) = p \cdot c b^{p-1} = 0. $$
So $b^p$ is a cocycle. Now assume $b^p = dx$ for some $x$ of degree $pn - 1$, then $x$ contains a factor $c$ for degree reasons. By the calculations above we see that any element containing $c$ has a trivial differential or has a factor $c$ in its differential, contradicting $b^p = dx$. So this cocycle is not a coboundary and $\Lambda D(n)$ is not acyclic.


\section{The Quillen model structure on \titleCDGA}

In this section we will define a model structure on cdga's over a field $\k$ of characteristic zero, where the weak equivalences are quasi isomorphisms and fibrations are surjective maps. The cofibrations are defined to be the maps with a left lifting property with respect to trivial fibrations.

\begin{proposition}
	There is a model structure on $\CDGA_\k$ where $f: A \to B$ is
	\begin{itemize}
		\item a \emph{weak equivalence} if $f$ is a quasi isomorphism,
		\item a \emph{fibration} if $f$ is an surjective and
		\item a \emph{cofibration} if $f$ has the LLP w.r.t. trivial fibrations
	\end{itemize}
\end{proposition}

We will prove the different axioms in the following lemmas. First observe that the classes as defined above are indeed closed under composition and contain all isomorphisms.

Note that with these classes, every cdga is a fibrant object.

\begin{lemma}
	[MC1] The category has all finite limits and colimits.
\end{lemma}
\begin{proof}
	As discussed earlier products are given by direct sums and equalizers are kernels. Furthermore the coproducts are tensor products and coequalizers are quotients.
\end{proof}

\begin{lemma}
	[MC2] The \emph{2-out-of-3} property for quasi isomorphisms.
\end{lemma}
\begin{proof}
	Let $f$ and $g$ be two maps such that two out of $f$, $g$ and $fg$ are weak equivalences. This means that two out of $H(f)$, $H(g)$ and $H(f)H(g)$ are isomorphisms. The 2-out-of-3 property holds for isomorphisms, proving the statement.
\end{proof}

\begin{lemma}
	[MC3] All three classes are closed under retracts
\end{lemma}
\begin{proof}
	For the class of weak equivalences and fibrations this follows easily from basic category theory. For cofibrations we consider the following diagram where the horizontal compositions are identities:
	\[ \xymatrix{
		A' \ar[r] \ar[d]^g & A \ar[r] \arcof[d]^f & A' \ar[d]^g \\
		B' \ar[r] & B \ar[r] & B'
	}\]
	We need to prove that $g$ is a cofibration, so for any lifting problem with a trivial fibration we need to find a lift. We are in the following situation:
		\[ \xymatrix{
		A' \ar[r] \ar[d]^g & A \ar[r] \arcof[d]^f & A' \ar[r] \ar[d]^g & X \artfib[d] \\
		B' \ar[r] & B \ar[r] & B' \ar[r] & Y
	}\]
	Now we can find a lift starting at $B$, since $f$ is a cofibration. By precomposition we obtain a lift $B' \to X$.
\end{proof}

Next we will prove the factorization property [MC5]. We will prove one part directly and the other by Quillen's small object argument. When proved, we get an easy way to prove the missing lifting property of [MC4]. For the Quillen's small object argument we use a class of generating cofibrations.

\begin{definition}
	Define the following objects and sets of maps:
	\begin{itemize}
		\item $\Lambda S(n)$ is the cdga generated by one element $a$ of degree $n$ such that $da = 0$.
		\item $\Lambda D(n)$ is the cdga generated by two elements $b$ and $c$ of degree $n$ and $n+1$ respectively, such that $db = c$ (and necessarily $dc = 0$).
		\item $I = \{ i_n: \k \to \Lambda D(n) \I n \in \N \}$ is the set of units.
		\item $J = \{ j_n: \Lambda S(n+1) \to \Lambda D(n) \I n \in \N \}$ is the set of inclusions $j_n$ defined by $j_n(a) = b$.
	\end{itemize}
\end{definition}

\Lemma{cdga-mc5a}{
	[MC5a] A map $f: A \to X$ can be factorized as $f = pi$ where $i$ is a trivial cofibration and $p$ a fibration.
}
\Proof{
	Consider the free cdga $C = \bigtensor_{x \in X} T(\deg{x})$. There is an obvious surjective map $p: C \to X$ which sends a generator corresponding to $x$ to $x$. Now define maps $\phi$ and $\psi$ in
	\[ A \tot{\phi} A \tensor C \tot{\psi} X\]
	by $\phi(a) = a \tensor 1$ and $\psi(a \tensor c) = f(a) \cdot p(c)$. Now $\psi$ is clearly surjective (as $p$ is) and $\phi$ is clearly a weak equivalence (by the Künneth theorem). Furthermore $\phi$ is a cofibration as we can construct lifts using the freeness of $C$.
}

\Remark{cdga-mc5a-left-inverse}{
	The map $\phi$ in the above construction has a left inverse $\overline{\phi}$ given by $\overline{\phi}(x \tensor c) = x \cdot \counit(c)$, where $\counit$ is the natural augmentation of a free cdga (i.e. it send $1$ to $1$ and all generators to $0$). Clearly $\overline{\phi} \phi = \id$, and so $\overline{\phi}$ is a fibration as well.

	Furthermore, if $f$ is a weak equivalence then by the 2-out-of-3 property both $\phi$ and $\psi$ are weak equivalences. Applying it once more, we find that $\overline{\phi}$ too is a weak equivalence. So for any weak equivalence $f: A \to X$ we find trivial fibrations $\overline{\phi} : A \tensor C \fib A$ and $\psi: A \tensor C \fib X$ compatible with $f$.
}

\begin{lemma}
	The maps $i_n$ are trivial cofibrations and the maps $j_n$ are cofibrations.
\end{lemma}
\begin{proof}
	Since $H(\Lambda D(n)) = \k$ (as stated earlier this uses \linebreak $\Char{\k} = 0$) we see that indeed $H(i_n)$ is an isomorphism. For the lifting property of $i_n$ and $j_n$ simply use surjectivity of the fibrations and the freeness of $\Lambda D(n)$ and $\Lambda S(n)$.
\end{proof}

\begin{lemma}
	The class of cofibrations is saturated.
\end{lemma}
\begin{proof}
	We need to prove that the classes are closed under retracts (this is already done), pushouts and transfinite compositions. For the class of cofibrations, this is easy as they are defined by the LLP and colimits behave nice with respect to such classes. 
\end{proof}

As a consequence of the above two lemmas, the class generated by $J$ is contained in the class of cofibrations. We can characterize trivial fibrations with $J$.

\begin{lemma}
	If $p: X \to Y$ has the RLP w.r.t. $J$ then $p$ is a trivial fibration.
\end{lemma}
\begin{proof}
	Let $y \in Y$ be of degree $n$ and $dy$ its boundary. By assumption we can find a lift in the following diagram:
	\[ \xymatrix{
		\Lambda S(n+1) \arcof[d]^{j_n} \ar[r]^-{a \mapsto 0} & X \ar[d]^f \\
		\Lambda D(n) \ar[r]^-{b \mapsto dy} & Y
	} \]
	The lift $h: D(n) \to X$ defines a preimage $x' = h(b)$ for $dy$. Now we can define a similar square to find a preimage $x$ of $y$ as follows:
	\[ \xymatrix{
		\Lambda S(n) \arcof[d]^{j_{n-1}} \ar[r]^-{a \mapsto x'} & X \ar[d]^f \\
		\Lambda D(n-1) \ar[r]^-{b \mapsto y} & Y
	} \]
	The lift $h : D(n-1) \to X$ defines $x = h(b)$. This proves that $f$ is surjective. Note that $dx = x'$.

	Now if $[y] \in H(Y)$ is some class, then $dy = 0$, and so by the above we find a preimage $x$ of $y$ such that $dx = 0$, proving that $H(f)$ is surjective. Now let $[x] \in H(X)$ such that $[f(x)] = 0$, then there is an element $\beta$ such that $f(x) = d\beta$, again by the above we can lift $\beta$ to get $x = d\alpha$., hence $H(f)$ is injective. Conclude that $f$ is a trivial fibration.
\end{proof}

We can use Quillen's small object argument with the set $J$. The argument directly proves the following lemma. Together with the above lemmas this translates to the required factorization.

\Lemma{cdga-mc5b}{
	A map $f: A \to X$ can be factorized as $f = pi$ where $i$ is in the class generated by $J$ and $p$ has the RLP w.r.t. $J$.
}
\Proof{
	This follows from Quillen's small object argument.
}

\Corollary{cdga-mc5b}{
	[MC5b] A map $f: A \to X$ can be factorized as $f = pi$ where $i$ is a cofibration and $p$ a trivial fibration.
}

\Lemma{cdga-mc4}{
	[MC4] The lifting properties.
}
\Proof{
	One part is already established by definition (cofibrations are defined by an LLP). It remains to show that we can lift in the following situation:
	\[\xymatrix{
		A \ar[r] \artcof[d]^f & X \arfib[d] \\
		B \ar[r] & Y
	}\]
	Now factor $f = pi$, where $p$ is a fibration and $i$ a trivial cofibration. By the 2-out-of-3 property $p$ is also a weak equivalence and we can find a lift in the following diagram:
	\[\xymatrix{
		A \ar[r]^i \arcof[d]^f & Z \artfib[d]^p \\
		B \ar[r]^\id \ar@{-->}[ur] & B
	}\]
	This defines $f$ as a retract of $i$. Now we know that $i$ has the LLP w.r.t. fibrations (by the small object argument above), hence $f$ has the LLP w.r.t. fibrations as well.
}


\section{Homotopy relations on \titleCDGA}

Although the abstract theory of model categories gives us tools to construct a homotopy relation (\DefinitionRef{homotopy}), it is useful to have a concrete notion of homotopic maps.

Consider the free cdga on one generator $\Lambda(t, dt)$, where $\deg{t} = 0$, this can be thought of as the (dual) unit interval with endpoints $1$ and $t$. We define two \emph{endpoint maps} as follows:
$$ d_0, d_1 : \Lambda(t, dt) \to \k $$
$$ d_0(t) = 1, \qquad d_1(t) = 0, $$
this extends linearly and multiplicatively. Note that it follows that we have $d_0(1-t) = 0$ and $d_1(1-t) = 1$. These two functions extend to tensor products as $d_0, d_1: \Lambda(t, dt) \tensor X \to \k \tensor X \tot{\iso} X$.

\Definition{cdga_homotopy}{
	We call $f, g: A \to X$ homotopic ($f \simeq g$) if there is a map
	$$ h: A \to \Lambda(t, dt) \tensor X, $$
	such that $d_0 h = g$ and $d_1 h = f$.
}

In terms of model categories, such a homotopy is a right homotopy and the object $\Lambda(t, dt) \tensor X$ is a path object for $X$. We can easily see that it is a very good path object. First note that $\Lambda(t, dt) \tensor X \tot{(d_0, d_1)} X \oplus X$ is surjective (for $(x, y) \in X \oplus X$ take $t \tensor x + (1-t) \tensor y$). Secondly we note that $\Lambda(t, dt) = \Lambda(D(0))$ and hence $\k \to \Lambda(t, dt)$ is a cofibration, by \LemmaRef{model-cats-coproducts} we have that $X \to \Lambda(t, dt) \tensor X$ is a (necessarily trivial) cofibration.

Clearly we have that $f \simeq g$ implies $f \simeq^r g$ (see \DefinitionRef{right_homotopy}), however the converse need not be true.

\Lemma{cdga_homotopy}{
	If $A$ is a cofibrant cdga and $f \simeq^r g: A \to X$, then $f \simeq g$ in the above sense.
}
\Proof{
	Because $A$ is cofibrant, there is a very good homotopy $H$. Consider a lifting problem to construct a map $Path_X \to \Lambda(t, dt) \tensor X$.
}

\Corollary{cdga_homotopy_eqrel}{
	For cofibrant $A$, $\simeq$ defines a equivalence relation.
}
\Definition{cdga_homotopy_classes}{
	For cofibrant $A$ define the set of equivalence classes as:
	$$ [A, X] = \Hom_{\CDGA_\k}(A, X) / \simeq. $$
}

The results from model categories immediately imply the following results. \todo{Refereer expliciet}
\Corollary{cdga_homotopy_properties}{
	Let $A$ be cofibrant.
	\begin{itemize}
		\item Let $i: A \to B$ be a trivial cofibration, then the induced map $i^\ast: [B, X] \to [A, X]$ is a bijection.
		\item Let $p: X \to Y$ be a trivial fibration, then the induced map $p_\ast: [A, X] \to [A, Y]$ is a bijection.
		\item Let $A$ and $X$ both be cofibrant, then $f: A \we X$ is a weak equivalence if and only if $f$ is a strong homotopy equivalence. Moreover, the two induced maps are bijections:
		$$ f_\ast: [Z, A] \tot{\iso} [Z, X], $$
		$$ f^\ast: [X, Z] \tot{\iso} [A, X]. $$
		\todo{De eerste werkt ook als $i$ gewoon een w.e. is. (Gebruik factorizatie.)}
	\end{itemize}
}

\Lemma{cdga_homotopy_homology}{
	Let $f, g: A \to X$ be two homotopic maps, then $H(f) = H(g): HA \to HX$.
}
\Proof{
	We only need to consider $H(d_0)$ and $H(d_1)$. \todo{Bewijs afmaken}
}


\section{Homotopy theory of augmented cdga's}

Recall that an augmented cdga is a cdga $A$ with an algebra map $A \tot{\counit} \k$ (this implies that $\counit \unit = \id$). This is precisely the dual notion of a pointed space. We will use the general fact that if $\cat{C}$ is a model category, then the over (resp. under) category $\cat{C} / A$ (resp. $A / \cat{C}$) for any object $A$ admit an induced model structure. In particular, the category of augmented cdga's (with augmentation preserving maps) has a model structure with the fibrations, cofibrations and weak equilavences as above.

Although the model structure is completely induced, it might still be fruitful to discuss the right notion of a homotopy for augmented cdga's. Consider the following pullback of cdga's:
\[ \xymatrix{
	\Lambda(t, dt) \overline{\tensor} A \ar[r] \xypb \ar[d] & \Lambda(t, dt) \tensor A \ar[d] \\
	\k \ar[r] & \k \tensor \Lambda(t, dt)
}\]
The pullback is the subspace of elements $x \tensor a$ in $\Lambda(t, dt) \tensor A$ such that $\counit(a) \cdot x \in \k$. Note that this construction is dual to a construction on topological spaces: in order to define a homotopy which is constant on the point $x_0$, we define the homotopy to be a map from a quotient ${X \times I} / {x_0 \times I}$.
\Definition{homotopy-augmented}{
	Two maps $f, g: A \to X$ between augmented cdga's are said to be \emph{homotopic} if there is a map
	$$h : A \to \Lambda(t, dt) \overline{\tensor} X$$
	such that $d_0 h = g$ and $d_1 h = f$.
}

In the next section homotopy groups of augmented cdga's will be defined. In order to define this we first need another tool.
\Definition{indecomposables}{
		Define the \Def{augmentation ideal} of $A$ as $\overline{A} = \ker \counit$. Define the \Def{cochain complex of indecomposables} of $A$ as $QA = \overline{A} / \overline{A} \cdot \overline{A}$.
} 

The first observation one should make is that $Q$ is a functor from algebras to modules (or differential algebras to differential modules) which is particularly nice for free algebras, as we have that $Q \Lambda V = V$ for any (differential) module $V$.

\todo{tensor}


\section{Homotopy groups of cdga's}

As the eventual goal is to compare the homotopy theory of spaces with the homotopy theory of cdga's, it is natural to investigate an analogue of homotopy groups in the category of cdga's. In topology we can only define homotopy groups on pointed spaces, dually we will consider augmented cdga's in this section.

\Definition{cdga-homotopy-groups}{
	The \Def{homotopy groups of an augmented cdga} $A$ are
	$$ \pi^i(A) = H^i(QA). $$
}

This construction is functorial (since both $Q$ and $H$ are) and, as the following lemma shows, homotopy invariant.

\Lemma{cdga-homotopic-maps-equal-pin}{
	Let $f: A \to X$ and $g: A \to X$ be a maps of augmented cdga's. If $f$ and $g$ are homotopic, then the induced maps are equal:
	$$ f_\ast = g_\ast : \pi_\ast(A) \to \pi_\ast(X). $$
}
\Proof{
	Let $h: A \to \Lambda(t, dt) \tensor X$ be a homotopy. We will, just as in \LemmaRef{cdga-homotopy-homology}, prove that the maps $HQ(d_0)$ and $HQ(d_1)$ are equal, then it follows that $HQ(f) = HQ(d_1 h) = HQ(d_0 h) = HQ(g)$.

	Using \LemmaRef{Q-preserves-copord} we can identify the induced maps $Q(d_i) : Q(\Lambda(t, dt) \tensor X) \to Q(X)$ with maps
	\[ Q(d_i) : Q(\Lambda(t, dt)) \oplus Q(A) \to Q(A). \]
	Now $Q(\Lambda(t, dt)) = D(0)$ and hence it is acyclic, so when we pass to homology, this term vanishes. In other words both maps ${d_i}_\ast : H(D(0)) \oplus H(Q(A)) \to H(Q(A))$ are the identity maps on $H(Q(A))$.
}

Consider the augmented cdga $V(n) = S(n) \oplus \k$, with trivial multiplication and where the term $\k$ is used for the unit and augmentation. This augmented cdga can be thought of as a specific model of the sphere. In particular the homotopy groups can be expressed as follows.

\Lemma{cdga-dual-homotopy-groups}{
	There is a natural bijection for any augmented cdga $A$
	$$ [A, V(n)] \tot{\iso} \Hom_\k(\pi^n(A), \k). $$
}
\Proof{
	Note that $Q(V(n))$ in degree $n$ is just $\k$ and $0$ in the other degrees, so its homotopy groups consists of a single $\k$ in degree $n$. This establishes the map:
	$$ \pi^n: \Hom(A, V(n)) \to \Hom_\k(\pi^n(A), \k). $$

	Now by \LemmaRef{cdga-homotopic-maps-equal-pin} we get a map from the set of homotopy classes $[A, V(n)]$ instead of the $\Hom$-set. It remains to prove that the map is an isomorphism. Surjectivity follows easily. Given a map $f: \pi^n(A) \to \k$, we can extend this to $A \to V(n)$ because the multiplication on $V(n)$ is trivial.

	For injectivity suppose $\phi, \psi: A \to V(n)$ be two maps such that $\pi^n(\phi) = \pi^n(\psi)$. We will first define a chain homotopy $D: A^\ast \to V(n)^{\ast - 1}$, for this we only need to specify the map $D^n: A^{n+1} \to V(n)^n = \Q$. Decompose the vector space $A^{n+1}$ as $A^{n+1} = \im d \oplus V$ for some $V$. Now set $D^n(v) = 0$ for all $v \in V$ and $D^n(db) = \phi(b) - \psi(b)$. We should check that $D$ is well defined. Note that for cycles we get $\phi(c) = \psi(c)$, as $H(Q(\phi)) = H(Q(\psi))$. So if $db = dc$, then we get $D(db) = \phi(b) - \psi(b) = \phi(c) - \psi(c) = D(dc)$, i.e. $D$ is well defined. We can now define a map of augmented cdga's:
	\begin{align*}
		h : X &\to \Lambda(t, dt) \overline{\tensor} V(n) \\
		    x &\mapsto dt \tensor D(x) + 1 \tensor \phi(x) - t \tensor \phi(x) + t \tensor \psi(x)
	\end{align*}
	This map commutes with the differential by the definition of $D$. Now we see that $d_0 h = \psi$ and $d_1 h = \phi$. Hence the two maps represent the same class, and we have proven the injectivity.
}

From now on the dual of a vector space will be denoted as $V^\ast = \Hom_\k(V, \k)$. So the above lemma states that there is a bijection $[A, V(n)] \iso \pi^n(A)^\ast$.

In topology we know that a fibration induces a long exact sequence of homotopy groups. In the case of cdga's a similar long exact sequence for a cofibration will exist.

\Lemma{long-exact-cdga-homotopy}{
	Given a pushout square of augmented cdga's
	\[ \xymatrix{
		A \ar[d]^-f \arcof[r]^-g \xypo & C \ar[d]^-i \\
		B \ar[r]^-j & P
	} \]
	where $g$ is a cofibration. There is a natural long exact sequence
	\[ \pi^o(V) \tot{(f_\ast, g_\ast)} \pi^0(B) \oplus \pi^0(C) \tot{j_\ast - i_\ast} \pi^0(P) \tot{\del} \pi^1(A) \to \cdots \]
}
\Proof{
	First note that $j$ is also a cofibration. By \LemmaRef{Q-preserves-cofibs} the maps $Qg$ and $Qj$ are injective in positive degrees. By applying $Q$ we get two exact sequence (in positive degrees) as shown in the following diagram. By the fact that $Q$ preserves pushouts (\CorollaryRef{Q-preserves-pushouts}) the cokernels coincide.
	\[ \xymatrix {
		0 \ar[r] & Q(A) \ar[r] \ar[d] \xypo & Q(C) \ar[r] \ar[d] & \coker(f_\ast) \ar[r] \ar[d] & 0 \\
		0 \ar[r] & Q(B) \ar[r] & Q(P) \ar[r] & \coker(f_\ast) \ar[r] & 0
	} \]
	Now the well known Mayer-Vietoris exact sequence can be constructed. This proves the statement.
}

\Corollary{long-exact-cdga-homotopy}{
	When we take $B = \k$ in the above situation, we get a long exact sequence
	\[ \pi^0(A) \tot{g_\ast} \pi^0(C) \to \pi^0(\coker(g)) \to \pi^1(A) \to \cdots \]
}

