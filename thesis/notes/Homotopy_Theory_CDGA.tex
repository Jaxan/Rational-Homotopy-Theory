
\Chapter{Homotopy Theory For cdga's}{HomotopyTheoryCDGA}

Recall that a cdga $A$ is a commutative differential graded algebra, meaning that
\begin{itemize}\itemsep0em
	\item it has a grading: $A = \bigoplus_{n\in\N} A^n$,
	\item it has a differential: $d: A \to A$ with $d^2 = 0$,
	\item it has a multiplication: $\mu: A \tensor A \to A$ which is associative and unital and
	\item it is commutative: $x y = (-1)^{\deg{x}\cdot\deg{y}} y x$.
\end{itemize}
And all of the above structure is compatible with each other (e.g. the differential is a derivation of degree $1$, the maps are graded, \dots). The exact requirements are stated in the appendix on algebra. An algebra $A$ is augmented if it has a specified map (of algebras) $A \tot{\counit} \k$. Furthermore we adopt the notation $A^{\leq n} = \bigoplus_{k \leq n} A^k$ and similarly for $\geq n$.

There is a left adjoint $\Lambda$ to the forgetful functor $U$ which assigns the free graded commutative algebras $\Lambda V$ to a graded module $V$. This extends to an adjunction (also called $\Lambda$ and $U$) between commutative differential graded algebras and differential graded modules. We denote the subspace of elements of wordlength $n$ by $\Lambda^n V$ (note that this has nothing to do with the grading on $V$).

In homological algebra we are especially interested in \emph{quasi isomorphisms}, i.e. maps $f: A \to B$ inducing an isomorphism on cohomology: $H(f): HA \iso HB$. This notions makes sense for any object with a differential.

We furthermore have the following categorical properties of cdga's:
\begin{itemize}\itemsep0em
	\item The finite coproduct in $\CDGA_\k$ is the (graded) tensor product.
	\item The finite product in $\CDGA_\k$ is the cartesian product (with pointwise operations).
	\item The equalizer (resp. coequalizer) of $f$ and $g$ is given by the kernel (resp. cokernel) of $f - g$. Together with the (co)products this defines pullbacks and pushouts.
	\item $\k$ and $0$ are the initial and final object.
\end{itemize}

\section{Cochain models for the $n$-disk and $n$-sphere}

We will first define some basic cochain complexes which model the $n$-disk and $n$-sphere. $D(n)$ is the cochain complex generated by one element $b \in D(n)^n$ and its differential $c = d(b) \in D(n)^{n+1}$. On the other hand we define $S(n)$ to be the cochain complex generated by one element $a \in S(n)^n$ with trivial differential (i.e. $d a = 0$). In other words:
$$ D(n) = ... \to 0 \to \k \to \k \to 0 \to ... $$
$$ S(n) = ... \to 0 \to \k \to 0 \to 0 \to ... $$

Note that $D(n)$ is acyclic for all $n$, or put in different words: $j_n : 0 \to D(n)$ induces an isomorphism in cohomology. The sphere $S(n)$ has exactly one non-trivial cohomology group \linebreak $H^n(S(n)) = \k \cdot [a]$. There is an injective function $i_n : S(n+1) \to D(n)$, sending $a$ to $c$. The maps $j_n$ and $i_n$ play the following important role in the model structure of cochain complexes, where weak equivalences are quasi isomorphisms, fibrations are degreewise surjective and cofibrations are degreewise injective for positive degrees \cite[Example 1.6]{goerss2}.

\begin{claim}
	The set $I = \{i_n : S(n+1) \to D(n) \I n \in \N\}$ generates all cofibrations and the set $J = \{j_n : 0 \to D(n) \I n \in \N\}$ generates all trivial cofibrations.
\end{claim}

As we do not directly need this claim, we omit the proof. However, in the next section we will prove a similar result for cdga's in detail.

$S(n)$ plays a another special role: maps from $S(n)$ to some cochain complex $X$ correspond directly to elements in the kernel of $\restr{d}{X^n}$. Any such map is null-homotopic precisely when the corresponding elements in the kernel is a coboundary. So there is a natural isomorphism: $\Hom(S(n), X) / {\simeq} \iso H^n(X)$.

By using the free cdga functor we can turn these cochain complexes into cdga's $\Lambda D(n)$ and $\Lambda S(n) $. So $\Lambda D(n)$ consists of linear combinations of $b^k$ and $c b^k$ when $n$ is even, and it consists of linear combinations of $c^k b$ and $c^k$ when $n$ is odd. In both cases we can compute the differentials using the Leibniz rule:

\[\xymatrix @R=0cm{
	d(b^k) = k \cdot c b^{k-1}   &   d(c^k b) = c^{k+1}\\
	d(c b^k) = 0   &   d(c^k) = 0
	}
\]

Those cocycles are in fact coboundaries (using that $\k$ is a field of characteristic $0$):
$$ c b^k = \frac{1}{k} d(b^{k+1}) $$
$$ c^k = d(b c^{k-1}) $$

There are no additional cocycles in $\Lambda D(n)$ besides the constants and $c$. So we conclude that $\Lambda D(n)$ is acyclic as an augmented algebra. In other words $\Lambda(j_n): \k \to \Lambda D(n)$ is a quasi isomorphism.

The situation for $\Lambda S(n)$ is easier as it has only one generator (as algebra). For even $n$ this means it is given by polynomials in $a$. For odd $n$ it is an exterior algebra, meaning $a^2 = 0$. Again the sets $\Lambda(I) = \{ \Lambda(i_n) : \Lambda S(n+1) \to \Lambda D(n) \I n \in \N\}$ and $\Lambda(J) = \{ \Lambda(j_n) : \k \to \Lambda D(n) \I n \in \N\}$ play an important role.

\begin{theorem}
	The sets $\Lambda(I)$ and $\Lambda(J)$ generate a model structure on $\CDGA_\k$ where:
	\begin{itemize}
		\item weak equivalences are quasi isomorphisms,
		\item fibrations are (degree wise) surjective maps and
		\item cofibrations are maps with the left lifting property against trivial fibrations.
	\end{itemize}
\end{theorem}

We will prove this theorem in the next section. Note that the functors $\Lambda$ and $U$ thus form a Quillen pair with this model structure.

\subsection{Why we need $\Char{\k} = 0$ for algebras}
The above Quillen pair $(\Lambda, U)$ fails to be a Quillen pair if \linebreak $\Char{\k} = p \neq 0$. We will show this by proving that the maps $\Lambda(j_n)$ are not weak equivalences for even $n$. Consider $b^p \in \Lambda D(n)$, then by the Leibniz rule:
$$ d(b^p) = p \cdot c b^{p-1} = 0. $$
So $b^p$ is a cocycle. Now assume $b^p = d x$ for some $x$ of degree $p n - 1$, then $x$ contains a factor $c$ for degree reasons. By the calculations above we see that any element containing $c$ has a trivial differential or has a factor $c$ in its differential, contradicting $b^p =  d x$. So this cocycle is not a coboundary and $\Lambda D(n)$ is not acyclic.


\section{The Quillen model structure on \titleCDGA}

\section{Model structure on \texorpdfstring{$\CDGA_\k$}{CDGA}}
\label{sec:model-of-cdga}

\TODO{First discuss the model structure on (co)chain complexes. Then discuss that we want the adjunction $(\Lambda, U)$ to be a Quillen pair. Then state that (co)chain complexes are cofib. generated, so we can cofib. generate CDGAs.}

In this section we will define a model structure on CDGAs over a field $\k$ of characteristic zero\todo{Can $\k$ be a c. ring here?}, where the weak equivalences are quasi isomorphisms and fibrations are surjective maps. The cofibrations are defined to be the maps with a left lifting property with respect to trivial fibrations.

\begin{proposition}
	There is a model structure on $\CDGA_\k$ where $f: A \to B$ is
	\begin{itemize}
		\item a \emph{weak equivalence} if $f$ is a quasi isomorphism,
		\item a \emph{fibration} if $f$ is an surjective and
		\item a \emph{cofibration} if $f$ has the LLP w.r.t. trivial fibrations
	\end{itemize}
\end{proposition}

We will prove the different axioms in the following lemmas. First observe that the classes as defined above are indeed closed under multiplication and contain all isomorphisms.

Note that with these classes, every cdga is a fibrant object.

\begin{lemma}
	[MC1] The category has all finite limits and colimits.
\end{lemma}
\begin{proof}
	As discussed earlier \todo{really discuss this somewhere} products are given by direct sums and equalizers are kernels. Furthermore the coproducts are tensor products and coequalizers are quotients.
\end{proof}

\begin{lemma}
	[MC2] The \emph{2-out-of-3} property for quasi isomorphisms.
\end{lemma}
\begin{proof}
	Let $f$ and $g$ be two maps such that two out of $f$, $g$ and $fg$ are weak equivalences. This means that two out of $H(f)$, $H(g)$ and $H(f)H(g)$ are isomorphisms. The \emph{2-out-of-3} property holds for isomorphisms, proving the statement.
\end{proof}

\begin{lemma}
	[MC3] All three classes are closed under retracts
\end{lemma}
\begin{proof}
	\todo{Make some diagrams and write it out}
\end{proof}

Next we will prove the factorization property [MC5]. We will do this by Quillen's small object argument. When proved, we get an easy way to prove the missing lifting property of [MC4]. For the Quillen's small object argument we use classes of generating cofibrations.

\begin{definition}
	Define the following objects and sets of maps:
	\begin{itemize}
		\item $S(n)$ is the CDGA generated by one element $a$ of degree $n$ such that $da = 0$.
		\item $T(n)$ is the CDGA generated by two element $b$ and $c$ of degree $n$ and $n+1$ respectively, such that $db = c$ (and necessarily $dc = 0$).
		\item $I = \{ i_n: \k \to T(n) \I n \in \N \}$ is the set of units of $T(n)$.
		\item $J = \{ j_n: S(n+1) \to T(n) \I n \in \N \}$ is the set of inclusions $j_n$ defined by $j_n(a) = b$.
	\end{itemize}
\end{definition}

\begin{lemma}
	The maps $i_n$ are trivial cofibrations and the maps $j_n$ are cofibrations.
\end{lemma}
\begin{proof}
	Since $H(T(n)) = \k$ \todo{Note that this only hold when characteristic = 0} we see that indeed $H(i_n)$ is an isomorphism. For the lifting property of $i_n$ and $j_n$ simply use surjectivity of the fibrations. \todo{give a bit more detail}
\end{proof}

\begin{lemma}
	The class of (trivial) cofibrations is saturated.
\end{lemma}
\begin{proof}
	\todo{prove this}
\end{proof}

As a consequence of the above two lemmas, the class generated by $I$ is contained in the class of trivial cofibrations. Similarly the class generated by $J$ is contained in the class of cofibrations. We also have a similar lemma about (trivial) fibrations.

\begin{lemma}
	If $p: X \to Y$ has the RLP w.r.t. $I$ then $p$ is a fibration.
\end{lemma}
\begin{proof}
	Easy\todo{Define a lift}.
\end{proof}

\begin{lemma}
	If $p: X \to Y$ has the RLP w.r.t. $J$ then $p$ is a trivial fibration.
\end{lemma}
\begin{proof}
	As $p$ has the RLP w.r.t. $J$, it also has the RLP w.r.t. $I$. From the previous lemma it follows that $p$ is a fibration. To show that $p$ is a weak equivalence ... \todo{write out}
\end{proof}

We can use Quillen's small object argument with these sets. The argument directly proves the following lemma. Together with the above lemmas this translates to the required factorization.

\begin{lemma}
	A map $f: A \to X$ can be factorized as $f = pi$ where $i$ is in the class generated by $I$ and $p$ has the RLP w.r.t. $I$.
\end{lemma}
\begin{proof}
	Quillen's small object argument. \todo{small = finitely generated?}
\end{proof}

\begin{corollary}
	[MC5a] A map $f: A \to X$ can be factorized as $f = pi$ where $i$ is a trivial cofibration and $p$ a fibration.
\end{corollary}

The previous factorization can also be described explicitly as seen in \cite{bousfield}. Let $f: A \to X$ be a map, define $E = A \tensor \bigtensor_{x \in X}T(\deg{x})$. Then $f$ factors as:
$$ A \tot{i} E \tot{p} X, $$
where $i$ is the obvious inclusion $i(a) = a \tensor 1$ and $p$ maps (products of) generators $a \tensor b_x$ with $b_x \in T(\deg{x})$ to $f(a) \cdot x \in X$.

\begin{lemma}
	A map $f: A \to X$ can be factorized as $f = pi$ where $i$ is in the class generated by $J$ and $p$ has the RLP w.r.t. $J$.
\end{lemma}
\begin{proof}
	Quillen's small object argument.
\end{proof}

\begin{corollary}
	[MC5b] A map $f: A \to X$ can be factorized as $f = pi$ where $i$ is a cofibration and $p$ a trivial fibration.
\end{corollary}


\subsection{Homotopy relation on \texorpdfstring{$\CDGA_\k$}{CDGA}}
Although the abstract theory of model categories gives us tools to construct a homotopy relation (\DefinitionRef{homotopy}), it is useful to have a concrete notion of homotopic maps.

Consider the free cdga on one generator $\Lambda(t, dt)$, this can be thought of as the (dual) unit interval. Indeed there is an isomorphism $\Lambda(t, dt) \iso \Apl_1$ and so we have maps for the two endpoint: $d_0, d_1: \Lambda(t, dt) \to \k \iso \Apl_0$. Given a cdga $X$ we will consider $d_0, d_1: \Lambda(t, dt) \tensor X \to \k \tensor X \iso X$.

\Definition{cdga_homotopy}{
	We call $f, g: A \to X$ homotopic ($f \simeq g$) if there is a map
	$$ h: A \to \Lambda(t, dt) \tensor X, $$
	such that $d_0 h = g$ and $d_1 h = f$.
}

In terms of model categories, such a homotopy is a right homotopy and the object $\Lambda(t, dt) \tensor X$ is a path object for $X$. We can easily see that it is a very good path object, first note that $\Lambda(t, dt) \tensor X \to X \oplus X$ is surjective (for $(x, y) \in X \oplus X$ take $t \tensor x + 1 \tensor y$), secondly $\Apl_0 \to \Apl_1$ is a cofibration and so is $X \to \Lambda(t, dt) \tensor X$.

Clearly we have that $f \simeq g$ implies $f \simeq^r g$ (see \DefinitionRef{right_homotopy}), however the converse need not be true.

\Lemma{cdga_homotopy}{
	If $A$ is a cofibrant cdga and $f \simeq^r g: A \to X$, then $f \simeq g$ in the above sense.
}
\Proof{
	Because $A$ is cofibrant, there is a very good homotopy $H$. Consider a lifting problem to construct a map $Path_X \to \Lambda(t, dt) \tensor X$.
}

\Corollary{cdga_homotopy_eqrel}{
	For cofibrant $A$, $\simeq$ defines a equivalence relation.
}
\Definition{cdga_homotopy_classes}{
	For cofibrant $A$ define the set of equivalence classes as:
	$$ [A, X] = \Hom_{\CDGA_\k}(A, X) / \simeq. $$
}

The results from model categories immediately imply the following results.
\Corollary{cdga_homotopy_properties}{
	Let $A$ be cofibrant.
	\begin{itemize}
		\item Let $i: A \to B$ be a trivial cofibration, then the induced map $i^\ast: [B, X] \to [A, X]$ is a bijection.
		\item Let $p: X \to Y$ be a trivial fibration, then the induced map $p_\ast: [A, X] \to [A, Y]$ is a bijection.
		\item Let $A$ and $X$ both be cofibrant, then $f: A \we X$ is a weak equivalence if and only if $f$ is a strong homotopy equivalence. Moreover, the two induced maps are bijections:
		$$ f_\ast: [Z, A] \tot{\iso} [Z, X], $$
		$$ f^\ast: [X, Z] \tot{\iso} [A, X]. $$
	\end{itemize}
}

\Lemma{cdga_homotopy_homology}{
	Let $f, g: A \to X$ be two homotopic maps, then $H(f) = H(g): HA \to HX$.
}
\Proof{
	We only need to consider $H(d_0)$ and $H(d_1)$.
}

\section{Homotopy relations on \titleCDGA}

Although the abstract theory of model categories gives us tools to construct a homotopy relation (\DefinitionRef{homotopy}), it is useful to have a concrete notion of homotopic maps.

Consider the free cdga on one generator $\Lambda(t, dt)$, where $\deg{t} = 0$, this can be thought of as the (dual) unit interval with endpoints $1$ and $t$. Notice that this cdga is isomorphic to $\Lambda(D(0))$ as defined in the previous section. We define two \emph{endpoint maps} as follows:
$$ d_0, d_1 : \Lambda(t, dt) \to \k $$
$$ d_0(t) = 1, \qquad d_1(t) = 0, $$
this extends linearly and multiplicatively. Note that it follows that we have $d_0(1-t) = 0$ and $d_1(1-t) = 1$. These two functions extend to tensor products as $d_0, d_1: \Lambda(t, dt) \tensor X \to \k \tensor X \tot{\iso} X$.

\Definition{cdga_homotopy}{
	We call $f, g: A \to X$ homotopic ($f \simeq g$) if there is a map
	$$ h: A \to \Lambda(t, dt) \tensor X, $$
	such that $d_0 h = g$ and $d_1 h = f$.
}

In terms of model categories, such a homotopy is a right homotopy and the object $\Lambda(t, dt) \tensor X$ is a path object for $X$. We can easily see that it is a very good path object\todo{Refereer}. First note that $\Lambda(t, dt) \tensor X \tot{(d_0, d_1)} X \oplus X$ is surjective (for $(x, y) \in X \oplus X$ take $t \tensor x + (1-t) \tensor y$). Secondly we note that $\Lambda(t, dt) = \Lambda(D(0))$ and hence $\k \to \Lambda(t, dt)$ is a cofibration, by \LemmaRef{model-cats-coproducts} we have that $X \to \Lambda(t, dt) \tensor X$ is a (necessarily trivial) cofibration.

Clearly we have that $f \simeq g$ implies $f \simeq^r g$ (see \DefinitionRef{right_homotopy}), however the converse need not be true.

\Lemma{cdga_homotopy}{
	If $A$ is a cofibrant cdga and $f \simeq^r g: A \to X$, then $f \simeq g$ in the above sense.
}
\Proof{
	Because $A$ is cofibrant, there is a very good homotopy $H$. Consider a lifting problem to construct a map $Path_X \to \Lambda(t, dt) \tensor X$.
}

\Corollary{cdga_homotopy_eqrel}{
	For cofibrant $A$, $\simeq$ defines a equivalence relation.
}
\Definition{cdga_homotopy_classes}{
	For cofibrant $A$ define the set of equivalence classes as:
	$$ [A, X] = \Hom_{\CDGA_\k}(A, X) / \simeq. $$
}

The results from model categories immediately imply the following results. Here we use Lemma \ref{lem:left_homotopy_properties}, \ref{lem:right_homotopy_properties} and \ref{lem:weak_strong_homotopy}.
\Corollary{cdga_homotopy_properties}{
	Let $A$ be cofibrant.
	\begin{itemize}
		\item Let $i: A \to B$ be a trivial cofibration, then the induced map $i^\ast: [B, X] \to [A, X]$ is a bijection.
		\item Let $p: X \to Y$ be a trivial fibration, then the induced map $p_\ast: [A, X] \to [A, Y]$ is a bijection.
		\item Let $A$ and $X$ both be cofibrant, then $f: A \we X$ is a weak equivalence if and only if $f$ is a strong homotopy equivalence. Moreover, the two induced maps are bijections:
		\begin{align*}
			f_\ast: [Z, A] &\tot{\iso} [Z, X], \\
			f^\ast: [X, Z] &\tot{\iso} [A, X].
		\end{align*}
	\end{itemize}
}
\Remark{cdga-weak-eq-bijection}{
	By \RemarkRef{cdga-mc5a-left-inverse} we can generalize the second item to arbitrary weak equivalences: If $A$ is cofibrant and $f : X \to Y$ a weak equivalence, then the induced map $f_\ast : [A, X] \to [A, Y]$ is a bijection, as seen from the following diagram:
	\[ \xymatrix{
		& [A, X \tensor C] \ar[dl]_{\overline{\phi}_\ast}^\iso \ar[dr]^{\psi_\ast}_\iso & \\
		[A, X] \ar[rr]^{f_\ast} & & [A, Y]
	}\]
}

\Lemma{cdga-homotopy-homology}{
	Let $f, g: A \to X$ be two homotopic maps, then $H(f) = H(g): HA \to HX$.
}
\Proof{
	Let $h$ be the homotopy such that $f = d_1 h$ and $g = d_0 h$. By the Künneth theorem we get the following commuting square for $i = 0, 1$:
	\[ \xymatrix{
		H(\Lambda(t, dt)) \tensor H(A) \ar[r]^-{d_i \tensor \id} \ar[d]^-{\iso} & \k \tensor H(A) \ar[d]^-{\iso} \\
		H(\Lambda(t, dt) \tensor A) \ar[r]^-{d_i} & H(\k \tensor A)
	} \]
	Now we know that $H(d_0) = H(d_1) : H(\Lambda(t, dt)) \to \k$ as $\Lambda(t, dt)$ is acyclic and the induced map sends $1$ to $1$. So the two bottom maps in the diagram are equal as well. Now we conclude $H(f) = H(d_1)H(h) = H(d_0)H(h) = H(g)$.
}


\section{Homotopy theory of augmented cdga's}

Recall that an augmented cdga is a cdga $A$ with an algebra map $A \tot{\counit} \k$ (this implies that $\counit \unit = \id$). This is precisely the dual notion of a pointed space. We will use the general fact that if $\cat{C}$ is a model category, then the over (resp. under) category $\cat{C} / A$ (resp. $A / \cat{C}$) for any object $A$ admit an induced model structure. In particular, the category of augmented cdga's (with augmentation preserving maps) has a model structure with the fibrations, cofibrations and weak equilavences as above.

Although the model structure is completely induced, it might still be fruitful to discuss the right notion of a homotopy for augmented cdga's. Consider the following pullback of cdga's:
\[ \xymatrix{
	\Lambda(t, dt) \overline{\tensor} A \ar[r] \xypb \ar[d] & \Lambda(t, dt) \tensor A \ar[d] \\
	\k \ar[r] & \k \tensor \Lambda(t, dt)
}\]
The pullback is the subspace of elements $x \tensor a$ in $\Lambda(t, dt) \tensor A$ such that $\counit(a) \cdot x \in \k$. Note that this construction is dual to a construction on topological spaces: in order to define a homotopy which is constant on the point $x_0$, we define the homotopy to be a map from a quotient ${X \times I} / {x_0 \times I}$.
\Definition{homotopy-augmented}{
	Two maps $f, g: A \to X$ between augmented cdga's are said to be \emph{homotopic} if there is a map
	$$h : A \to \Lambda(t, dt) \overline{\tensor} X$$
	such that $d_0 h = g$ and $d_1 h = f$.
}

In the next section homotopy groups of augmented cdga's will be defined. In order to define this we first need another tool.
\Definition{indecomposables}{
		Define the \Def{augmentation ideal} of $A$ as $\overline{A} = \ker \counit$. Define the \Def{cochain complex of indecomposables} of $A$ as $QA = \overline{A} / \overline{A} \cdot \overline{A}$.
} 

The first observation one should make is that $Q$ is a functor from algebras to modules (or differential algebras to differential modules) which is particularly nice for free algebras, as we have that $Q \Lambda V = V$ for any (differential) module $V$.

\todo{tensor}


\section{Homotopy groups of cdga's}

As the eventual goal is to compare the homotopy theory of spaces with the homotopy theory of cdga's, it is natural to investigate an analogue of homotopy groups in the category of cdga's. In topology we can only define homotopy groups on pointed spaces, dually we will consider augmented cdga's in this section. Recall that an augmented cdga is a cdga $A$ with an algebra map $A \tot{\counit} \k$ such that $\counit \unit = \id$.

\Definition{cdga-homotopy-groups}{
	Define the \Def{augmentation ideal} of $A$ as $\overline{A} = \ker \counit$. Define the \Def{cochain complex of indecomposables} of $A$ as $QA = \overline{A} / \overline{A} \cdot \overline{A}$.

	Now define the \Def{homotopy groups of a cdga} $A$ as
	$$ \pi^i(A) = H^i(QA). $$
}

This construction is functorial and, as the following lemma shows, homotopy invariant.

\Lemma{cdga-homotopic-maps-equal-pin}{
	Let $f: A \to B$ be a map of augmented cdga's. Then there is an functorial induced map on the homotopy groups. Moreover if $g: A \to B$ is homotopic to $f$, then the induced maps are equal:
	$$ f_\ast = g_\ast : \pi_\ast(A) \to \pi_\ast(B). $$
}
\Proof{
	Let $\phi: A \to B$ be a map of algebras. Then clearly we get an induced map $\overline{A} \to \overline{B}$ as $\phi$ preserves the augmentation. By composition we get a map $\phi': \overline{A} \to Q(B)$ for which we have $\phi'(xy) = \phi'(x)\phi'(y) = 0$. So it induces a map $Q(\phi): Q(A) \to Q(B)$. By functoriality of taking homology we get $f_\ast : \pi^n(A) \to \pi^n(B)$.

	Now if $f$ and $g$ are homotopic, then there is a homotopy $h: A \to \Lambda(t, dt) \tensor B$. By the Künneth theorem we have:
	$$ {d_0}_\ast = {d_1}_\ast : H(\Lambda(t, dt) \tensor Q(B)) \to H(Q(B)). $$
	This means that $f_\ast = {d_1}_\ast h_\ast = {d_0}_\ast h_\ast = g_\ast$. \todo{detail}
}

Consider the augmented cdga $V(n) = S(n) \oplus \k$, with trivial multiplication and where the term $\k$ is used for the unit and augmentation. This augmented cdga can be thought of as a specific model of the sphere. In particular the homotopy groups can be expressed as follows.

\Lemma{cdga-dual-homotopy-groups}{
	There is a natural bijection for any augmented cdga $A$
	$$ [A, V(n)] \tot{\iso} \Hom_\k(\pi^n(A), \k). $$
}
\Proof{
	Note that $Q(V(n))$ in degree $n$ is just $\k$ and $0$ in the other degrees, so its homotopy groups consists of a single $\k$ in degree $n$. This establishes the map:
	$$ \Phi: \Hom_\CDGA(A, V(n)) \to \Hom_\k(\pi^n(A), \k). $$

	Now by \LemmaRef{cdga-homotopic-maps-equal-pin} we get a map from the set of homotopy classes $[A, V(n)]$ instead of just maps. \todo{injective, surjective}
}

From now on the dual of a vector space will be denoted as $V^\ast = \Hom_\k(V, \k)$. So the above lemma states that there is a bijection $[A, V(n)] \iso \pi^n(A)^\ast$.

\todo{long exact sequence}

