
\section{Model structure on \texorpdfstring{$\CDGA_\k$}{CDGA}}
\label{sec:model-of-cdga}

\TODO{First discuss the model structure on (co)chain complexes. Then discuss that we want the adjunction $(\Lambda, U)$ to be a Quillen pair. Then state that (co)chain complexes are cofib. generated, so we can cofib. generate CDGAs.}

In this section we will define a model structure on CDGAs over a field $\k$ of characteristic zero\todo{Can $\k$ be a c. ring here?}, where the weak equivalences are quasi isomorphisms and fibrations are surjective maps. The cofibrations are defined to be the maps with a left lifting property with respect to trivial fibrations.

\begin{proposition}
	There is a model structure on $\CDGA_\k$ where $f: A \to B$ is
	\begin{itemize}
		\item a \emph{weak equivalence} if $f$ is a quasi isomorphism,
		\item a \emph{fibration} if $f$ is an surjective and
		\item a \emph{cofibration} if $f$ has the LLP w.r.t. trivial fibrations
	\end{itemize}
\end{proposition}

We will prove the different axioms in the following lemmas. First observe that the classes as defined above are indeed closed under multiplication and contain all isomorphisms.

\begin{lemma}
	[MC1] The category has all finite limits and colimits.
\end{lemma}
\begin{proof}
	As discussed earlier \todo{really discuss this somewhere} products are given by direct sums and equalizers are kernels. Furthermore the coproducts are tensor products and coequalizers are quotients.
\end{proof}

\begin{lemma}
	[MC2] The \emph{2-out-of-3} property for quasi isomorphisms.
\end{lemma}
\begin{proof}
	Let $f$ and $g$ be two maps such that two out of $f$, $g$ and $fg$ are weak equivalences. This means that two out of $H(f)$, $H(g)$ and $H(f)H(g)$ are isomorphisms. The \emph{2-out-of-3} property holds for isomorphisms, proving the statement.
\end{proof}

\begin{lemma}
	[MC3] All three classes are closed under retracts
\end{lemma}
\begin{proof}
	\todo{Make some diagrams and write it out}
\end{proof}

Next we will prove the factorization property [MC5]. We will do this by Quillen's small object argument. When proved, we get an easy way to prove the missing lifting property of [MC4]. For the Quillen's small object argument we use classes of generating cofibrations.

\begin{definition}
	Define the following objects and sets of maps:
	\begin{itemize}
		\item $S(n)$ is the CDGA generated by one element $a$ of degree $n$ such that $da = 0$.
		\item $T(n)$ is the CDGA generated by two element $b$ and $c$ of degree $n$ and $n+1$ respectively, such that $db = c$ (and necessarily $dc = 0$).
		\item $I = \{ i_n: \k \to T(n) \I n \in \N \}$ is the set of units of $T(n)$.
		\item $J = \{ j_n: S(n+1) \to T(n) \I n \in \N \}$ is the set of inclusions $j_n$ defined by $j_n(a) = b$.
	\end{itemize}
\end{definition}

\begin{lemma}
	The maps $i_n$ are trivial cofibrations and the maps $j_n$ are cofibrations.
\end{lemma}
\begin{proof}
	Since $H(T(n)) = \k$ \todo{Note that this only hold when characteristic = 0} we see that indeed $H(i_n)$ is an isomorphism. For the lifting property of $i_n$ and $j_n$ simply use surjectivity of the fibrations. \todo{give a bit more detail}
\end{proof}

\begin{lemma}
	The class of (trivial) cofibrations is saturated.
\end{lemma}
\begin{proof}
	\todo{prove this}
\end{proof}

As a consequence of the above two lemmas, the class generated by $I$ is contained in the class of trivial cofibrations. Similarly the class generated by $J$ is contained in the class of cofibrations. We also have a similar lemma about (trivial) fibrations.

\begin{lemma}
	If $p: X \to Y$ has the RLP w.r.t. $I$ then $p$ is a fibration.
\end{lemma}
\begin{proof}
	Easy\todo{Define a lift}.
\end{proof}

\begin{lemma}
	If $p: X \to Y$ has the RLP w.r.t. $J$ then $p$ is a trivial fibration.
\end{lemma}
\begin{proof}
	As $p$ has the RLP w.r.t. $J$, it also has the RLP w.r.t. $I$. From the previous lemma it follows that $p$ is a fibration. To show that $p$ is a weak equivalence ... \todo{write out}
\end{proof}

We can use Quillen's small object argument with these sets. The argument directly proves the following lemma. Together with the above lemmas this translates to the required factorization.

\begin{lemma}
	A map $f: A \to X$ can be factorized as $f = pi$ where $i$ is in the class generated by $I$ and $p$ has the RLP w.r.t. $I$.
\end{lemma}
\begin{proof}
	Quillen's small object argument. \todo{small = finitely generated?}
\end{proof}

\begin{corollary}
	[MC5a] A map $f: A \to X$ can be factorized as $f = pi$ where $i$ is a trivial cofibration and $p$ a fibration.
\end{corollary}

The previous factorization can also be described explicitly as seen in \cite{bousfield}. Let $f: A \to X$ be a map, define $E = A \tensor \bigtensor_{x \in X}T(\deg{x})$. Then $f$ factors as:
$$ A \tot{i} E \tot{p} X, $$
where $i$ is the obvious inclusion $i(a) = a \tensor 1$ and $p$ maps (products of) generators $a \tensor b_x$ with $b_x \in T(\deg{x})$ to $f(a) \cdot x \in X$.

\begin{lemma}
	A map $f: A \to X$ can be factorized as $f = pi$ where $i$ is in the class generated by $J$ and $p$ has the RLP w.r.t. $J$.
\end{lemma}
\begin{proof}
	Quillen's small object argument.
\end{proof}

\begin{corollary}
	[MC5b] A map $f: A \to X$ can be factorized as $f = pi$ where $i$ is a cofibration and $p$ a trivial fibration.
\end{corollary}


