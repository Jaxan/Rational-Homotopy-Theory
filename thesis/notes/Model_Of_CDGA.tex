
In this section we will define a model structure on cdga's over a field $\k$ of characteristic zero, where the weak equivalences are quasi isomorphisms and fibrations are surjective maps. The cofibrations are defined to be the maps with a left lifting property with respect to trivial fibrations.

\begin{proposition}
	There is a model structure on $\CDGA_\k$ where $f: A \to B$ is
	\begin{itemize}
		\item a \emph{weak equivalence} if $f$ is a quasi isomorphism,
		\item a \emph{fibration} if $f$ is an surjective and
		\item a \emph{cofibration} if $f$ has the LLP w.r.t. trivial fibrations
	\end{itemize}
\end{proposition}

We will prove the different axioms in the following lemmas. First observe that the classes as defined above are indeed closed under composition and contain all isomorphisms.

Note that with these classes, every cdga is a fibrant object.

\begin{lemma}
	[MC1] The category has all finite limits and colimits.
\end{lemma}
\begin{proof}
	As discussed earlier products are given by direct sums and equalizers are kernels. Furthermore the coproducts are tensor products and coequalizers are quotients.
\end{proof}

\begin{lemma}
	[MC2] The \emph{2-out-of-3} property for quasi isomorphisms.
\end{lemma}
\begin{proof}
	Let $f$ and $g$ be two maps such that two out of $f$, $g$ and $fg$ are weak equivalences. This means that two out of $H(f)$, $H(g)$ and $H(f)H(g)$ are isomorphisms. The 2-out-of-3 property holds for isomorphisms, proving the statement.
\end{proof}

\begin{lemma}
	[MC3] All three classes are closed under retracts
\end{lemma}
\begin{proof}
	For the class of weak equivalences and fibrations this follows easily from basic category theory. For cofibrations we consider the following diagram where the horizontal compositions are identities:
	\[ \xymatrix{
		A' \ar[r] \ar[d]^g & A \ar[r] \arcof[d]^f & A' \ar[d]^g \\
		B' \ar[r] & B \ar[r] & B'
	}\]
	We need to prove that $g$ is a cofibration, so for any lifting problem with a trivial fibration we need to find a lift. We are in the following situation:
		\[ \xymatrix{
		A' \ar[r] \ar[d]^g & A \ar[r] \arcof[d]^f & A' \ar[r] \ar[d]^g & X \artfib[d] \\
		B' \ar[r] & B \ar[r] & B' \ar[r] & Y
	}\]
	Now we can find a lift starting at $B$, since $f$ is a cofibration. By precomposition we obtain a lift $B' \to X$.
\end{proof}

Next we will prove the factorization property [MC5]. We will do this by Quillen's small object argument. When proved, we get an easy way to prove the missing lifting property of [MC4]. For the Quillen's small object argument we use classes of generating cofibrations.

\begin{definition}
	Define the following objects and sets of maps:
	\begin{itemize}
		\item $\Lambda S(n)$ is the cdga generated by one element $a$ of degree $n$ such that $da = 0$.
		\item $\Lambda D(n)$ is the CDGA generated by two element $b$ and $c$ of degree $n$ and $n+1$ respectively, such that $db = c$ (and necessarily $dc = 0$).
		\item $I = \{ i_n: \k \to \Lambda D(n) \I n \in \N \}$ is the set of units of $\Lambda D(n)$.
		\item $J = \{ j_n: \Lambda S(n+1) \to \Lambda D(n) \I n \in \N \}$ is the set of inclusions $j_n$ defined by $j_n(a) = b$.
	\end{itemize}
\end{definition}

\begin{lemma}
	The maps $i_n$ are trivial cofibrations and the maps $j_n$ are cofibrations.
\end{lemma}
\begin{proof}
	Since $H(\Lambda D(n)) = \k$ (as stated earlier this uses $\Char{\k} = 0$) we see that indeed $H(i_n)$ is an isomorphism. For the lifting property of $i_n$ and $j_n$ simply use surjectivity of the fibrations and the freeness of $\Lambda D(n)$ and $\Lambda S(n)$.
\end{proof}

\begin{lemma}
	The class of (trivial) cofibrations is saturated.
\end{lemma}
\begin{proof}
	We need to prove that the classes are closed under retracts (this is already done), pushouts and transfinite compositions. For the class of cofibrations, this is easy as they are defined by the LLP and colimits behave nice with respect to such classes. 

	However the case of trivial cofibrations does not follow immediately, as we still need to prove that quasi isomorphisms behave as such.\todo{THIS IS HARD}
\end{proof}

As a consequence of the above two lemmas, the class generated by $I$ is contained in the class of trivial cofibrations. Similarly the class generated by $J$ is contained in the class of cofibrations. We also have a similar lemma about (trivial) fibrations.

\begin{lemma}
	If $p: X \to Y$ has the RLP w.r.t. $I$ then $p$ is a fibration.
\end{lemma}
\begin{proof}
	Let $y \in Y^n$ be an element of degree $n$, then we have the following commuting diagram:
	\cdiagram{CDGA_Model_I_Fib}
	where $g$ sends the generator $b$ to $y$ and $c$ to $dy$. By assumption there exists a lift $h$. Now $h(b) \in X^n$ is a preimage for $y$, proving that $p$ is surjective.
\end{proof}

\begin{lemma}
	If $p: X \to Y$ has the RLP w.r.t. $J$ then $p$ is a trivial fibration.
\end{lemma}
\begin{proof}
	\todo{bewijzen}
\end{proof}

We can use Quillen's small object argument with these sets. The argument directly proves the following lemma. Together with the above lemmas this translates to the required factorization. \todo{Definieer wat ``small'' betkent en geef een referentie}

\begin{lemma}
	A map $f: A \to X$ can be factorized as $f = pi$ where $i$ is in the class generated by $I$ and $p$ has the RLP w.r.t. $I$.
\end{lemma}
\begin{proof}
	This follows from Quillen's small object argument.
\end{proof}

\begin{corollary}
	[MC5a] A map $f: A \to X$ can be factorized as $f = pi$ where $i$ is a trivial cofibration and $p$ a fibration.
\end{corollary}

The previous factorization can also be described explicitly as seen in \cite{bousfield}. Let $f: A \to X$ be a map, define $E = A \tensor \bigtensor_{x \in X}T(\deg{x})$. Then $f$ factors as:\todo{This is later defined as $A \tensor \Lambda(C(X))$, which is precisely the same}
$$ A \tot{i} E \tot{p} X, $$
where $i$ is the obvious inclusion $i(a) = a \tensor 1$ and $p$ maps (products of) generators $a \tensor b_x$ with $b_x \in T(\deg{x})$ to $f(a) \cdot x \in X$.

\begin{lemma}
	A map $f: A \to X$ can be factorized as $f = pi$ where $i$ is in the class generated by $J$ and $p$ has the RLP w.r.t. $J$.
\end{lemma}
\begin{proof}
	Quillen's small object argument.
\end{proof}

\begin{corollary}
	[MC5b] A map $f: A \to X$ can be factorized as $f = pi$ where $i$ is a cofibration and $p$ a trivial fibration.
\end{corollary}

\begin{lemma}
	[MC4]
\end{lemma}
\Proof{
	One part is already established by definition (cofibrations are defined by an LLP). It remains to show that we can lift in the following situation:
	\[\xymatrix{
		A \ar[r] \artcof[d]^f & X \arfib[d] \\
		B \ar[r] & Y
	}\]
	Now factor $f = pi$, where $p$ is a fibration and $i$ a trivial cofibration. By the 2-out-of-3 property $p$ is also a weak equivalence and we can find a lift in the following diagram:
	\[\xymatrix{
		A \ar[r]^i \arcof[d]^f & Z \artfib[d]^p \\
		B \ar[r]^\id \ar@{-->}[ur] & B
	}\]
	This defines $f$ as a retract of $i$. Now we know that $i$ has the LLP w.r.t. fibrations (by the small object argument above), hence $f$ has the LLP w.r.t. fibrations as well.
}
