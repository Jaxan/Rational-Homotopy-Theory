
\section{Polynomial Forms}
\label{sec:polynomial-forms}

There is a general way to construct functors from $\sSet$ whenever we have some simplicial object. In our case we have the simplicial cdga $\Apl$ (which is nothing more than a functor $\opCat{\DELTA} \to \CDGA$) and we want to extend to a contravariant functor $\sSet \to \CDGA_\k$. This will be done via \Def{Kan extensions}.

Given a category $\cat{C}$ and a functor $F: \DELTA \to \cat{C}$, then define the following on objects:
\begin{align*}
	F_!(X)      &= \colim_{\Delta[n] \to X} F[n] & X \in \sSet \\
	F^\ast(C)_n &= \Hom_{\cat{C}}(F[n], Y)       & C \in \cat{C}
\end{align*}
A simplicial map $X \to Y$ induces a map of the diagrams of which we take colimits. Applying $F$ on these diagrams, make it clear that $F_!$ is functorial. Secondly we see readily that $F^\ast$ is functorial. By using the definition of colimit and the Yoneda lemma (Y) we can prove that $F_!$ is left adjoint to $F^\ast$:

\begin{align*}
	\Hom_\cat{C}(F_!(X), Y) &\iso \Hom_\cat{C}(\colim_{\Delta[n] \to X} F[n], Y) \iso \lim_{\Delta[n] \to X} \Hom_\cat{C}(F[n], Y) \iso \lim_{\Delta[n] \to X} F^\ast(Y)_n \\
	&\stackrel{\text{Y}}{\iso} \lim_{\Delta[n] \to X} \Hom_\sSet(\Delta[n], F^\ast(Y)) \iso \Hom_\sSet(\colim_{\Delta[n] \to X} \Delta[n], F^\ast(Y)) \\
	&\iso \Hom_\sSet(X, F^\ast(Y)).
\end{align*}

Furthermore we have $F_! \circ \Delta[-] = F$. In short we have the following:

\cimage[scale=0.5]{Kan_Extension}

In our case where $F = \Apl$ and $\cat{C} = \CDGA_\k$ we get:

\cimage[scale=0.5]{Apl_Extension}


\subsection{The cochain complex of polynomial forms}

In our case we take the opposite category, so the definition of $A$ is in terms of a limit instead of colimit. This allows us to give a nicer description:

\begin{align*}
	A(X) &= \lim_{\Delta[n] \to X} \Apl_n \stackrel{Y}{\iso} \lim_{\Delta[n] \to X} \Hom_\sSet(\Delta[n], \Apl) \iso \Hom_\sSet(\colim_{\Delta[n] \to X}\Delta[n], \Apl) \\
	&= \Hom_\sSet(X, \Apl),
\end{align*}

where the addition, multiplication and differential are defined pointwise. Conclude that we have the following contravariant functors (which form an adjoint pair):

\begin{align*}
	A(X) &= \Hom_\sSet(X, \Apl) & X \in \sSet \\
	K(C)_n &= \Hom_{\CDGA_\k}(C, \Apl_n) & C \in \CDGA_\k.
\end{align*}


\subsection{The singular cochain complex}

Another way to model the $n$-simplex is by the singular cochain complex associated to the topological $n$-simplices. Define the following (non-commutative) dga's \todo{Choose: normalized or not?}:
$$ C_n = C^\ast(\Delta^n; \k). $$
The inclusion maps $d^i : \Delta^n \to \Delta^{n+1}$ and the maps $s^i: \Delta^n \to \Delta^{n-1}$ induce face and degeneracy maps on the dga's $C_n$, turning $C$ into a simplicial dga. Again we can extend this to functors by Kan extensions

\cimage[scale=0.5]{C_Extension}

where the left adjoint is precisely the functor $C^\ast$ as noted in \cite{felix}. We will relate $\Apl$ and $C$ in order to obtain a natural quasi isomorphism $A(X) \we C^\ast(X)$ for every $X \in \sSet$. Furthermore this map preserves multiplication on the homology algebras.


\subsection{Integration and Stokes' theorem for polynomial forms}

In this section we will prove that the singular cochain complex is quasi isomorphic to the cochain complex of polynomial forms. In order to do so we will define an integration map $\int_n : \Apl_n^n \to \k$, which will induce a map $\oint_n : \Apl_n \to C_n$. For the simplices $\Delta[n]$ we already showed the cohomology agrees by the acyclicity of $\Apl_n = A(\Delta[n])$ (\LemmaRef{apl-acyclic}).

Let $v \in \Apl_n^n$, then we can always write it as $v = p(x_1, \dots, x_n)dx_1 \dots dx_n$ where $p$ is a polynomial in $n$ variables. If $\Q \subset \k \subset \mathbb{C}$ we can integrate analytically
$$ \int_n v = \int_0^1 \int_0^{1-x_n} \dots \int_0^{1 - x_2 - \dots - x_n} p(x_1, \dots, x_n) dx_1 dx_2 \dots dx_n, $$
which defines a well-defined linear map $\int_n : \Apl_n^n \to \k$. For general fields of characteristic zero we can define it formally on the generators of $\Apl_n^n$ (as vector space):
$$ \int_n x_1^{k_1} \dots x_n^{k_n} dx_1 \dots dx_n = \frac{k_1! \dots k_n!}{(k_1 + \dots + k_n)!}. $$

Let $x$ be a $k$-simplex of $\Delta[n]$, i.e. $x: \Delta[k] \to \Delta[n]$. Then $x$ induces a linear map $x^\ast: \Apl_n \to \Apl_k$. Let $v \in \Apl_n^k$, then $x^\ast(v) \in \Apl_k^k$, which we can integrate. Now define
$$ \oint_n(v)(x) = (-1)^\frac{k(k-1)}{2} \int_n x^\ast(v). $$
Note that $\oint_n(v): \Delta[n] \to \k$ is just a map, we can extend this linearly to chains on $\Delta[n]$ to obtain $\oint_n(v): \Z\Delta[n] \to \k$, in other words $\oint_n(v) \in C_n$. By linearity of $\int_n$ and $x^\ast$, we have a linear map $\oint_n: \Apl_n \to C_n$.

Next we will show that $\oint_n$ is a chain map and taking $\oint = \{\oint_n\}_n$ gives a simplicial chain map.

