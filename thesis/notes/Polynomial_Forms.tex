
\Chapter{Polynomial Forms}{Adjunction}
\label{sec:cdga-of-polynomials}

\section{CDGA of Polynomials}

We will now give a cdga model for the $n$-simplex $\Delta^n$. This then allows for simplicial methods. In the following definition one should remember the topological $n$-simplex defined as convex span.

\Definition{apl}{
	For all $n \in \N$ define the following cdga:
	$$ (\Apl)_n = \frac{\Lambda(x_0, \ldots, x_n, d x_0, \ldots, d x_n)}{(\sum_{i=0}^n x_i - 1, \sum_{i=0}^n d x_i)}, $$
	where $\deg{x_i} = 0$. So it is the free cdga with $n+1$ generators and their differentials such that $\sum_{i=0}^n x_i = 1$ and in order to be well behaved $\sum_{i=0}^n d x_i = 0$.
}

Note that the inclusion $\Lambda(x_1, \ldots, x_n, d x_1, \ldots, d x_n) \to \Apl_n$ is an isomorphism of cdga's. So $\Apl_n$ is free and (algebra) maps from it are determined by their images on $x_i$ for $i = 1, \ldots, n$ (also note that this determines the images for $d x_i$). This fact will be used throughout. Also note that we have already seen the dual unit interval $\Lambda(t, dt)$ which is isomorphic to $\Apl_1$.

These cdga's will assemble into a simplicial cdga when we define the face and degeneracy maps as follows ($j = 1, \ldots, n$):

$$ d_i(x_j) = \begin{cases}
	x_{j-1}, &\text{ if } i < j \\
	0,       &\text{ if } i = j \\
	x_j,     &\text{ if } i > j
\end{cases} \qquad d_i : \Apl_n \to \Apl_{n-1} $$
$$ s_i(x_j) = \begin{cases}
	x_{j+1},       &\text{ if } i < j \\
	x_j + x_{j+1}, &\text{ if } i = j \\
	x_j,           &\text{ if } i > j	
\end{cases} \qquad s_i : \Apl_n \to \Apl_{n+1} $$

One can check that $\Apl \in \simplicial{\CDGA_\k}$. We will denote the subspace of homogeneous elements of degree $k$ as $\Apl^k$, this is a simplicial $\k$-module as the maps $d_i$ and $s_i$ are graded maps of degree $0$.

\pagebreak
\Lemma{apl-contractible}{
	$\Apl^k$ is contractible.
}
\Proof{
	We will prove this by defining an extra degeneracy $s: \Apl_n \to \Apl_{n+1}$. In the more geometric context of topological $n$-simplices we would achieve this by dividing by $1-x_0$. However, since this algebra consists of polynomials only, this cannot be done. Instead, we will multiply everything by $(1-x_0)^2$, so that we can divide by $1-x_0$. Define for $i = 1, \ldots, n$:
	\begin{align*}
		s(1) &= (1-x_0)^2 \\
		s(x_i) &= (1-x_0) \cdot x_{i+1}
	\end{align*}
	Extend on the differentials and multiplicatively on $\Apl_n$. As $s(1) \neq 1$ this map is not an algebra map, however it well-defined as a map of cochain complexes. In particular when restricted to degree $k$ we get a linear map:
	$$ s: \Apl^k_n \to \Apl^k_{n+1}. $$
	Proving the necessary properties of an extra degeneracy is fairly easy. For $n \geq 1$ we get (on generators):
	\begin{align*}
		d_0 s(1)   &= d_0 (1 - x_0)^2 = (1 - 0) \cdot (1 - 0) = 1 \\
		d_0 s(x_i) &= d_0((1-x_0)x_{i+1}) = d_0(1-x_0) \cdot x_i  \\
		           &= (1-0) \cdot x_i = x_i
	\end{align*}
	So $d_0 s = \id$.
	\begin{align*}
		d_{i+1} s(1) &= d_{i+1} (1 - x_0)^2 = d_{i+1} (\sum_{j=1}^n x_j)^2 \\
		             &= (\sum_{j=1}^{n-1} x_j)^2 = (1-x_0)^2 = s d_i(1)    \\
		d_{i+1} s(x_j) &= d_{i+1}(1-x_0) d_{i+1}(x_j) = (1-x_0) d_i(x_{j+1}) = s d_i (x_j)
	\end{align*}
	So $d_{i+1} s = s d_i$. Similarly $s_{i+1} s = s s_i$. And finally for $n=0$ we have $d_1 s = 0$.

	So we have an extra degeneracy $s: \Apl^k \to \Apl^k$, and hence (see for example \cite{goerss}) we have that $\Apl^k$ is contractible. As a consequence $\Apl^k \to \ast$ is a weak equivalence.
}

\Lemma{apl-kan-complex}{
	$\Apl^k$ is a Kan complex.
}
\Proof{
	By the simple fact that $\Apl^k$ is a simplicial group, it is a Kan complex \cite{goerss}.
}

Combining these two lemmas gives us the following.
\Corollary{apl-extendable}{
	$\Apl^k \to \ast$ is a trivial fibration in the standard model structure on $\sSet$.
}

Besides the simplicial structure of $\Apl$, there is also the structure of a cochain complex.
\Lemma{apl-acyclic}{
	$\Apl_n$ is acyclic, i.e. $H(\Apl_n) = \k \cdot [1]$.
}
\Proof{
	This is clear for $\Apl_0 = \k \cdot 1$. For $\Apl_1$ we see that $\Apl_1 = \Lambda(x_1, d x_1) \iso \Lambda D(0)$, which we proved to be acyclic in the previous section.

	For general $n$ we can identify $\Apl_n \iso \bigtensor_{i=1}^n \Lambda(x_i, d x_i)$, because $\Lambda$ is left adjoint and hence preserves coproducts. By the Künneth theorem \TheoremRef{kunneth} we conclude $H(\Apl_n) \iso \bigtensor_{i=1}^n H \Lambda(x_i, d x_i) \iso \bigtensor_{i=1}^n H \Lambda D(0) \iso \k \cdot [1]$.

	So indeed $\Apl_n$ is acyclic for all $n$.
}


\section{Polynomial Forms on a Space}
\label{sec:polynomial-forms}

There is a general way to construct contravariant functors from $\sSet$ whenever we have some simplicial object. In our case we have the simplicial cdga $\Apl$ (which is nothing more than a functor $\opCat{\DELTA} \to \CDGA$) and we want to extend to a contravariant functor $\sSet \to \CDGA_\k$. This will be done via \Def{Kan extensions}.

Given a category $\cat{C}$ and a functor $F: \DELTA \to \cat{C}$, then define the following on objects:
\begin{align*}
	F_!(X)      &= \colim_{\Delta[n] \to X} F[n] & X \in \sSet \\
	F^\ast(C)_n &= \Hom_{\cat{C}}(F[n], Y)       & C \in \cat{C}
\end{align*}
A simplicial map $X \to Y$ induces a map of the diagrams of which we take colimits. Applying $F$ on these diagrams, make it clear that $F_!$ is functorial. Secondly we see readily that $F^\ast$ is functorial. By using the definition of colimit and the Yoneda lemma (Y) we can prove that $F_!$ is left adjoint to $F^\ast$ by the following calculation:

\begin{align*}
	\Hom_\cat{C}(F_!(X), Y)
	&\iso \Hom_\cat{C}(\colim_{\Delta[n] \to X} F[n], Y) \\
	&\iso \lim_{\Delta[n] \to X} \Hom_\cat{C}(F[n], Y) \\
	&\iso \lim_{\Delta[n] \to X} F^\ast(Y)_n
\end{align*}
\begin{align*}
	&\stackrel{\text{Y}}{\iso} \lim_{\Delta[n] \to X} \Hom_\sSet(\Delta[n], F^\ast(Y)) \\
	&\iso \Hom_\sSet(\colim_{\Delta[n] \to X} \Delta[n], F^\ast(Y)) \\
	&\iso \Hom_\sSet(X, F^\ast(Y))
\end{align*}

Furthermore we have $F_! \circ \Delta[-] \iso F$. In short we have the following:
\cdiagram{Kan_Extension}


\subsection{The cochain complex of polynomial forms}
In our case where $F = \Apl$ and $\cat{C} = \CDGA_\k$ we get:
\cdiagram{Apl_Extension}
Note that we have the opposite category of cdga's, so the definition of $A$ is in terms of a limit instead of colimit. This allows us to give a nicer description:
\begin{align*}
	A(X)
	&= \lim_{\Delta[n] \to X} \Apl_n 
	\stackrel{Y}{\iso} \lim_{\Delta[n] \to X} \Hom_\sSet(\Delta[n], \Apl) \\
	&\iso \Hom_\sSet(\colim_{\Delta[n] \to X}\Delta[n], \Apl)
	= \Hom_\sSet(X, \Apl),
\end{align*}
where the addition, multiplication and differential are defined pointwise. Conclude that we have the following contravariant functors (which form an adjoint pair):
\begin{align*}
	A(X) &= \Hom_\sSet(X, \Apl) & X \in \sSet \\
	K(C)_n &= \Hom_{\CDGA_\k}(C, \Apl_n) & C \in \CDGA_\k
\end{align*}


\subsection{The singular cochain complex}

Another way to model the $n$-simplex is by the singular cochain complex associated to the topological $n$-simplices. Define the following (non-commutative) dga's:
$$ C_n = C^\ast(\Delta^n; \k), $$
where $C^\ast(\Delta^n; \k)$ is the (normalized) singular cochain complex of $\Delta^n$ with coefficients in $\k$. The inclusion maps $d^i : \Delta^n \to \Delta^{n+1}$ and the maps $s^i: \Delta^n \to \Delta^{n-1}$ induce face and degeneracy maps on the dga's $C_n$, turning $C$ into a simplicial dga. Again we can extend this to functors by Kan extensions
\cdiagram{C_Extension}

This left adjoint functor $C^\ast : \sSet \to \opCat{\DGA_\k}$ is (just as above) defined as $C^\ast(X) = \Hom_\sSet(X, C^\ast(\Delta[-]; \k))$. To see that this is precisely the classical definition of the singular \linebreak cochain complex, we make the following calculation.
\begin{align*}
	C^\ast(X) &= \Hom(X, C^\ast(\Delta[n])) \\
	          &= \Hom(X, \Hom(N \Z \Delta[n], \k)) \\
	          &\ison{1} \Hom(X, \Gamma(\k)) \\
	          &\iso \Hom(N \Z (X), \k)
\end{align*}
where $\Z$ is the free simplicial abelian group, $N$ is the normalized chain complex (this is the Dold-Kan equivalence) and $\Gamma$ its right adjoint. At \refison{1} we use the definition of this right adjoint $\Gamma(C) = \Hom(N \Z \Delta[n], C)$. Now the conclusion of this calculation is that $C^\ast(X)$ is precisely the dual complex of $N \Z (X)$, which is the singular (normalized) chain complex.

We will relate $\Apl$ and $C$ in order to obtain a natural quasi isomorphism $A(X) \we C^\ast(X)$ for every $X \in \sSet$. Furthermore this map preserves multiplication on the homology algebras.


\subsection{Integration and Stokes' theorem for polynomial forms}

In this section we will prove that the singular cochain complex is quasi isomorphic to the cochain complex of polynomial forms. In order to do so we will define an integration map $\int_n : \Apl_n^n \to \k$, which will induce a map $\oint_n : \Apl_n \to C_n$. For the simplices $\Delta[n]$ we already showed the cohomology agrees by the acyclicity of $\Apl_n = A(\Delta[n])$ (\LemmaRef{apl-acyclic}).

For any $v \in \Apl_n^n$, we can write $v$ as $v = p(x_1, \dots, x_n)dx_1 \dots dx_n$ where $p$ is a polynomial in $n$ variables. If $\Q \subset \k \subset \mathbb{C}$ we can integrate geometrically on the $n$-simplex:
$$ \int_n v = \int_0^1 \int_0^{1-x_n} \dots \int_0^{1 - x_2 - \dots - x_n} p(x_1, \dots, x_n) dx_1 dx_2 \dots dx_n, $$
which defines a well-defined linear map $\int_n : \Apl_n^n \to \k$. For general fields of characteristic zero we can define it formally on the generators of $\Apl_n^n$ (as vector space):
$$ \int_n x_1^{k_1} \dots x_n^{k_n} dx_1 \dots dx_n = \frac{k_1! \dots k_n!}{(k_1 + \dots + k_n + n)!}. $$

Let $x$ be a $k$-simplex of $\Delta[n]$. Then $x$ induces a linear map $x^\ast: \Apl_n \to \Apl_k$. Now if we have an element $v \in \Apl_n^k$, then observe that $x^\ast(v) \in \Apl_k^k$ is an element we can integrate. Now define
$$ \oint_n(v)(x) = (-1)^\frac{k(k-1)}{2} \int_n x^\ast(v). $$
Note that $\oint_n(v): \Delta[n] \to \k$ is just a map of sets, so we can extend this linearly to chains on $\Delta[n]$ to obtain a linear map $\oint_n(v): \Z\Delta[n] \to \k$. 
If $x$ is a degenerate simplex $x = s_j x'$, then $x^\ast(v) = s_j^\ast (x'^\ast (v))$. Now $x'^\ast (v) \in \Apl_{k-1}^k = 0$ and so the integral vanishes on degenerate simplices. In other words we get $\oint_n(v) \in C_n$. By linearity of $\int_n$ and $x^\ast$, we have a linear map $\oint_n: \Apl_n \to C_n$.

Next we will show that $\oint = \{\oint_n\}_n$ is a simplicial map and that each $\oint_n$ is a chain map, in other words $\oint : \Apl \to C$ is a simplicial chain map (of complexes). Let $\sigma: \Delta[n] \to \Delta[k]$, and $\sigma^\ast: \Apl_k \to \Apl_n$ its induced map. We need to prove $\oint_n \circ \sigma^\ast = \sigma^\ast \circ \oint_k$. We show this as follows:
\begin{align*}
	\oint_n (\sigma^\ast v)(x)
	&= (-1)^\frac{l(l-1)}{2} \int_l x^\ast(\sigma^\ast(v)) \\
	&= (-1)^\frac{l(l-1)}{2} \int_l (\sigma \circ x)^\ast(v) \\
	&= \oint_k (v)(\sigma \circ x) \\
	&= (\oint_k (v) \circ \sigma) (x) = \sigma^\ast (\oint_k(v)(x))
\end{align*}

For it to be a chain map, we need to prove $d \circ \oint_n = \oint_n \circ d$. This is precisely \emph{Stokes' theorem} and any prove will apply here as well \cite{bousfield}.

We now proved that $\oint$ is indeed a simplicial chain map. Note that $\oint_n$ need not to preserve multiplication, so it fails to be a map of cochain algebras. However $\oint(1) = 1$ and so the induced map on homology sends the class of $1$ in $H(\Apl_n) = \k \cdot [1]$ to the class of $1$ in $H(C_n) = \k \cdot [1]$. We have proven the following lemma.

\Lemma{apl-c-quasi-iso}{
	The map $\oint_n: \Apl_n \to C_n$ is a quasi isomorphism for all $n$.
}

Recall that we can identify $\Apl_n$ with $A(\Delta[n])$ and similarly for the singular cochain complex.
\Corollary{apl-c-quasi-iso}{
	The induced map $\oint: A(\Delta[n]) \to C^\ast(\Delta[n])$ is a quasi isomorphism for all $n$.
}

We will now prove that the map $\oint: A(X) \to C^\ast(X)$ is a quasi isomorphism for any space $X$. We will do this in several steps, the base case of simplices is already proven. With induction we will prove it for spaces with finitely many simplices. At last we will use a limit argument for the general case.

\Theorem{apl-c-quasi-iso}{
	The induced map $\oint: A(X) \to C^\ast(X)$ is a natural quasi isomorphism.
}
\Proof{
	Assume we have a simplicial set $X$ such that $\oint: A(X) \to C^\ast(X)$ is a quasi isomorphism. We can add a simplex by considering pushouts of the following form:
	\cdiagram{Apl_C_Quasi_Iso_Pushout}

	We can apply our two functors to it, and use the natural transformation $\oint$ to obtain the following cube:
	\cdiagram{Apl_C_Quasi_Iso_Cube}

	Note that $A(\Delta[n]) \we C^\ast(\Delta[n])$ by \CorollaryRef{apl-c-quasi-iso}, $A(X) \we C^\ast(X)$ by assumption and $A(\del \Delta[n]) \we C^\ast(\del \Delta[n])$ by induction. Secondly note that both $A$ and $C^\ast$ send injective maps to surjective maps, so we get fibrations on the right side of the diagram. Finally note that the front square and back square are pullbacks, by adjointness of $A$ and $C^\ast$. Apply the cube lemma (\LemmaRef{cube-lemma}) to conclude that also $A(X') \we C^\ast(X')$.

	This proves $A(X) \we C^\ast(X)$ for any simplicial set with finitely many non-degenerate simplices. We can extend this to simplicial sets of finite dimension by attaching many simplices at once. For this we observe that both $A$ and $C^\ast$ send coproducts to products and that cohomology commutes with products:
	$$ H(A(\coprod_\alpha X_\alpha)) \iso H(\prod_\alpha A(X_\alpha)) \iso \prod_\alpha H(A(X_\alpha)), $$
	$$ H(C^\ast(\coprod_\alpha X_\alpha)) \iso H(\prod_\alpha C^\ast(X_\alpha)) \iso \prod_\alpha H(C^\ast(X_\alpha)). $$

	This means that we can extend the previous argument to pushout of this form:
	\begin{displaymath}
		\xymatrix {
		\coprod_{\alpha \in A} \del \Delta[n] \arcof[d] \ar[r] \xypo & X \ar[d] \\
		\coprod_{\alpha \in A} \Delta[n] \ar[r] & X'
		}
	\end{displaymath}

	Finally we can write any simplicial set $X$ as a colimit of finite dimensional ones as:
	$$ sk_0 X \cof sk_1 X \cof sk_2 \cof \dots \colim sk_n X = X, $$
	where $sk_i X$ has no non-degenerate simplices in dimensions $n > i$. By the above $\oint$ gives a quasi isomorphism on all the terms $sk_i X$. So we are in the following situation:
	\begin{displaymath}
		\xymatrix @C=0.3cm{
		A(X) = \lim_i A(sk_i X) \ar[d]^\oint \ar@{-->>}[rr] & & A(sk_2 X) \arfib[r] \arwe[d]^\oint & A(sk_1 X) \arfib[r] \arwe[d]^\oint & A(sk_0 X) \arwe[d]^\oint \\
		C^\ast(X) = \lim_i C^\ast(sk_i X) \ar@{-->>}[rr] & & C^\ast(sk_2 X) \arfib[r] & C^\ast(sk_1 X) \arfib[r] & C^\ast(sk_0 X)
		}
	\end{displaymath}

	We will define long exact sequences for both sequences in the following way. As the following construction is quite general, consider arbitrary cochain algebras $B_i$ as follows:
	\begin{displaymath}
		\xymatrix{
		B = \lim_i B_i \ar@{-->>}[rr] & & B_2 \arfib[r]^-{b_1} & B_1 \arfib[r]^-{b_0} & B_0
		}
	\end{displaymath}
	Define a map $t: \prod_i B_i \to \prod_i B_i$ defined by $t(x_0, x_1, \dots) = (x_0 + b_0(x_1), x_1 + b_1(x_2), \dots)$. Note that $t$ is surjective and that \linebreak $B \iso \ker(t)$. So we get the following natural short exact sequence and its associated natural long exact sequence in homology:
	$$ 0 \to B \tot{i} \prod_i B_i \tot{t} \prod_i B_i \to 0, $$
	$$ \cdots \tot{\Delta} H^n(B) \tot{i_\ast} H^n(\prod_i B_i) \tot{t_\ast} H^n(\prod_i B_i) \tot{\Delta} \cdots $$

	In our case we get two such long exact sequences with $\oint$ connecting them. As cohomology commutes with products we get isomorphisms on the left and right in the following diagram.
	\begin{displaymath}
		\xymatrix @C=0.3cm{
			\cdots \ar[r] & H^{n-1}(\prod_i A(sk_i X)) \ar[r] \ariso[d]^\oint & H^n(A(X)) \ar[r] \ar[d]^\oint & H^n(\prod_i A(sk_i X)) \ar[r] \ariso[d]^\oint & \cdots \\
			\cdots \ar[r] & H^{n-1}(\prod_i C^\ast(sk_i X)) \ar[r] & H^n(C^\ast(X)) \ar[r] & H^n(\prod_i C^\ast(sk_i X)) \ar[r] & \cdots \\
		}
	\end{displaymath}

	So by the five lemma we can conclude that the middle morphism is an isomorphism as well, proving the isomorphism $H^n(A(X)) \tot{\iso} H^n(C^\ast(X))$ for all $n$. This proves the statement for all $X$.
}
