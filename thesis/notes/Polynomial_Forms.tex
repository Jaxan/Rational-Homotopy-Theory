
\section{Polynomial Forms}
\label{sec:polynomial-forms}

There is a general way to construct functors from $\sSet$ whenever we have some simplicial object. In our case we have the simplicial cdga $\Apl$ (which is nothing more than a functor $\opCat{\DELTA} \to \CDGA$) and we want to extend to a contravariant functor $\sSet \to \CDGA_\k$. This will be done via \Def{Kan extensions}.

Given a category $\cat{C}$ and a functor $F: \DELTA \to \cat{C}$, then define the following on objects:
\begin{align*}
	F_!(X)      &= \colim_{\Delta[n] \to X} F[n] & X \in \sSet \\
	F^\ast(C)_n &= \Hom_{\cat{C}}(F[n], Y)       & C \in \cat{C}
\end{align*}
A simplicial map $X \to Y$ induces a map of the diagrams of which we take colimits. Applying $F$ on these diagrams, make it clear that $F_!$ is functorial. Secondly we see readily that $F^\ast$ is functorial. By using the definition of colimit and the Yoneda lemma (Y) we can prove that $F_!$ is left adjoint to $F^\ast$:

\begin{align*}
	\Hom_\cat{C}(F_!(X), Y) &\iso \Hom_\cat{C}(\colim_{\Delta[n] \to X} F[n], Y) \iso \lim_{\Delta[n] \to X} \Hom_\cat{C}(F[n], Y) \iso \lim_{\Delta[n] \to X} F^\ast(Y)_n \\
	&\stackrel{\text{Y}}{\iso} \lim_{\Delta[n] \to X} \Hom_\sSet(\Delta[n], F^\ast(Y)) \iso \Hom_\sSet(\colim_{\Delta[n] \to X} \Delta[n], F^\ast(Y)) \\
	&\iso \Hom_\sSet(X, F^\ast(Y)).
\end{align*}

Furthermore we have $F_! \circ \Delta[-] = F$. In short we have the following:

\cimage[scale=0.5]{Kan_Extension}

In our case where $F = \Apl$ and $\cat{C} = \CDGA_\k$ we get:

\cimage[scale=0.5]{Apl_Extension}


\subsection{The cochain complex of polynomial forms}

In our case we take the opposite category, so the definition of $A$ is in terms of a limit instead of colimit. This allows us to give a nicer description:

\begin{align*}
	A(X) &= \lim_{\Delta[n] \to X} \Apl_n \stackrel{Y}{\iso} \lim_{\Delta[n] \to X} \Hom_\sSet(\Delta[n], \Apl) \iso \Hom_\sSet(\colim_{\Delta[n] \to X}\Delta[n], \Apl) \\
	&= \Hom_\sSet(X, \Apl),
\end{align*}

where the addition, multiplication and differential are defined pointwise. Conclude that we have the following contravariant functors (which form an adjoint pair):

\begin{align*}
	A(X) &= \Hom_\sSet(X, \Apl) & X \in \sSet \\
	K(C)_n &= \Hom_{\CDGA_\k}(C, \Apl_n) & C \in \CDGA_\k.
\end{align*}


\subsection{The singular cochain complex}

Another way to model the $n$-simplex is by the singular cochain complex associated to the topological $n$-simplices. Define the following (non-commutative) dga's:
$$ C_n = C^\ast(\Delta^n; \k). $$
The inclusion maps $d^i : \Delta^n \to \Delta^{n+1}$ and the maps $s^i: \Delta^n \to \Delta^{n-1}$ induce face and degeneracy maps on the dga's $C_n$, turning $C$ into a simplicial dga. Again we can extend this to functors by Kan extensions

\cimage[scale=0.5]{C_Extension}

where the left adjoint is precisely the functor $C^\ast$ as noted in \cite{felix}. We will relate $\Apl$ and $C$ in order to obtain a natural quasi isomorphism $A(X) \we C^\ast(X)$ for every $X \in \sSet$. Furthermore this map preserves multiplication on the homology algebras.