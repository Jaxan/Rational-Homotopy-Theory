
Recall that an augmented cdga is a cdga $A$ with an algebra map $A \tot{\counit} \k$ (this implies that $\counit \unit = \id$). This is precisely the dual notion of a pointed space. We will use the general fact that if $\cat{C}$ is a model category, then the over (resp. under) category $\cat{C} / A$ (resp. $A / \cat{C}$) for any object $A$ admit an induced model structure. In particular, the category of augmented cdga's (with augmentation preserving maps) has a model structure with the fibrations, cofibrations and weak equilavences as above.

Although the model structure is completely induced, it might still be fruitful to discuss the right notion of a homotopy for augmented cdga's. Consider the following pullback of cdga's:
\[ \xymatrix{
	\Lambda(t, dt) \overline{\tensor} A \ar[r] \xypb \ar[d] & \Lambda(t, dt) \tensor A \ar[d] \\
	\k \ar[r] & \k \tensor \Lambda(t, dt)
}\]
The pullback is the subspace of elements $x \tensor a$ in $\Lambda(t, dt) \tensor A$ such that $\counit(a) \cdot x \in \k$. Note that this construction is dual to a construction on topological spaces: in order to define a homotopy which is constant on the point $x_0$, we define the homotopy to be a map from a quotient ${X \times I} / {x_0 \times I}$.
\Definition{homotopy-augmented}{
	Two maps $f, g: A \to X$ between augmented cdga's are said to be \emph{homotopic} if there is a map
	$$h : A \to \Lambda(t, dt) \overline{\tensor} X$$
	such that $d_0 h = g$ and $d_1 h = f$.
}

In the next section homotopy groups of augmented cdga's will be defined. In order to define this we first need another tool.
\Definition{indecomposables}{
		Define the \Def{augmentation ideal} of $A$ as $\overline{A} = \ker \counit$. Define the \Def{cochain complex of indecomposables} of $A$ as $QA = \overline{A} / \overline{A} \cdot \overline{A}$.
} 

The first observation one should make is that $Q$ is a functor from algebras to modules (or differential algebras to differential modules) which is particularly nice for free (differential) algebras, as we have that $Q \Lambda V = V$ for any (differential) module $V$.

The second observation is that $Q$ is nicely behaved on tensor products and cokernels.
\Lemma{Q-preserves-copord}{
	Let $A$ and $B$ be two augmented cdga's, then there is a natural isomorphism
	\[ Q(A \tensor B) \iso Q(A) \oplus Q(B). \]
}
\Proof{
	First note that the augmentation ideal is expressed as
	$\overline{A \tensor B} = \overline{A} \tensor B \>+\> A \tensor \overline{B}$
	and the product is
	$\overline{A \tensor B} \cdot \overline{A \tensor B} = \overline{A} \tensor \overline{B} \>+\> \overline{A}\cdot\overline{A} \tensor \k \>+\> \k \tensor \overline{B}$.
	With this we can prove the statement
	\begin{align*}
		Q(A \tensor B)
		&= \frac{\overline{A} \tensor B \>+\> A \tensor \overline{B}}
			{\overline{A} \tensor \overline{B} \>+\> \overline{A}\cdot\overline{A} \tensor \k \>+\> \k \tensor \overline{B}} \\
		&\iso \frac{\overline{A} \tensor \k \>\oplus\> \k \tensor \overline{B}}
			{\overline{A}\cdot\overline{A} \tensor \k \>\oplus\> \k \tensor \overline{B}\cdot\overline{B}}
		= Q(A) \,\oplus\, Q(B).
	\end{align*}
}

\Lemma{Q-preserves-coeq}{
	Let $f : A \to B$ be a map of augmented cdga's, then there is a natural isomorphism
	\[ Q(\coker(f)) \iso \coker(Qf). \]
}
\Proof{
	First note that the cokernel of $f$ in the category of augmented cdga's is $\coker(f) = B / f(\overline{A})$ and that its augmentation ideal is $\overline{B} / f(\overline{A})$. Just as above we make a simple calculation, where $p: \overline{B} \to Q(B)$ is the projection map:
	\begin{align*}
		Q(\coker(f))
		&= \frac{\overline{B} / f(\overline{A})}
			{\overline{B} / f(\overline{A}) \cdot \overline{B} / f(\overline{A})} \\
		&\iso \frac{\overline{B} / \overline{B}\cdot\overline{B}}
			{pf(\overline{A})}
		= \frac{Q(B)}{Qf(Q(A))}.
	\end{align*}
}

\Corollary{Q-preserves-pushouts}{
	Combining the two lemmas above, we see that $Q$ (as functor from augmented cdga's to cochain complexes) preserves pushouts.
}

Furthermore we have the following lemma which is of homotopical interest.

\Lemma{Q-preserves-cofibs}{
	If $f: A \to B$ is a cofibration of augmented cdga's, then $Qf$ is injective in positive degrees.
}
\Proof{
	First we define an augmented cdga $U(n)$ for each positive $n$ as $U(n) = D(n) \oplus \k$ with trivial multiplication and where the term $\k$ is used for the unit and augmentation. Notice that the map $U(n) \to \k$ is a trivial fibration. By the lifting property we see that the induced map
	\[ \Hom_\AugCDGA(Y, U(n)) \tot{f^\ast} \Hom_\AugCDGA(X, U(n)) \]
	is surjective for each positive $n$. Note that maps from $X$ to $U(n)$ will send products to zero and that it is fixed on the augmentation. So there is a natural isomorphism $\Hom_\AugCDGA(X, U(n)) \iso \Hom_\k(Q(X)^n, \k)$. Hence
	\[ \Hom_\k(Q(Y)^n, \k) \tot{(Qf)^\ast} \Hom_\k(Q(X)^n, \k) \]
	is surjective, and so $Qf$ itself is injective in positive $n$.
}
