
\chapter{Rationalizations}
\label{sec:rationalizations}

In this section we will prove the existence of rationalizations $X \to X_\Q$. We will do this in a cellular way. The $n$-spheres play an important role here, so their rationalizations will be discussed first. Again spaces (except for $S^1$) are assumed to be $1$-connected.

\section{Rationalization of \texorpdfstring{$S^n$}{Sn}}
In this section we fix $n>0$. We will construct $S^n_\Q$ in stages $S^n(1), S^n(2), \ldots$, where at each stage we wedge a sphere and then glue a $n+1$-cell to ``invert'' some element in the $n$th homotopy group.

\todo{plaatje}

We start the construction with $S^n(1) = S^n$. Assume we constructed $S^n(r) = \bigvee_{i=1}^{r} S^{n} \cup_{h} \coprod_{i=1}^{r-1} D^{n+1}$, where $h$ is a specific attaching map. Assume furthermore the following two properties. Firstly, the inclusion $i_r : S^n \to S^n(r)$ of the terminal sphere is a weak equivalence. Secondly, the inclusion $i_1 : S^n \to S^n(r)$ of the initial sphere induces the multiplication $\pi_n(S^n) \tot{\times r!} \pi_n(S^n(r))$ under the identification of $\pi_n(S^n) = \pi_n(S^n(r)) = \Z$.

We will construct $S^n(r+1)$ with similar properties as follows. Let $f: S^n \to S^n(r)$ be a representative for $1 \in \Z \iso \pi_n(S^n(r))$ and $g: S^n \to S^n$ be a representative for $r+1 \in \Z \iso \pi_n(S^n)$. These maps combine into $\phi: S^n \to S^n \vee S^n \tot{f \vee g} S^n(r) \vee S^n$. We define $S^n(r+1)$ as the pushout in the following diagram.

\begin{displaymath}
	\xymatrix{
	S^n \ar[r]^{\phi} \arcof[d] & S^n(r) \vee S^n \ar[d] \\
	D^{n+1} \ar[r] & S^n(r+1)
	}
\end{displaymath}

So $S^n(r+1) = \bigvee_{i=1}^{r+1} S^{n} \cup_{h'} \coprod_{i=1}^{r} D^{n+1}$. \todo{Prove the two properties}.

Now to finish the construction we define the \Def{rational sphere} as $S^n_\Q = \colim_r S^n(r)$. Note that the homotopy groups commute with filtered colimits \cite[9.4]{may}, so that we can compute $\pi_n(S^n_\Q)$ as the colimit of the terms $\pi_n(S^n(r)) \iso \Z$ and the induced maps as depicted in the following diagram:

$$ \Z \tot{\times 2} \Z \tot{\times 3} \Z \tot{\times 4} \Z \tot{\times 5} \cdots \Q. $$

Moreover we note that the generator $1 \in \pi_n(S^n)$ is sent to $1 \in \pi_n(S^n_\Q)$ via the inclusion $S^n \to S^n_\Q$ of the initial sphere. However the other homotopy groups are harder to calculate as we have generally no idea how the induced maps will look like. But in the case of $n=1$, the other trivial homotopy groups of $S^1$ are trivial.

\Corollary{rationalization-S1}{
	The inclusion $S^1 \to S^1_\Q$ is a rationalization.
}

For $n>1$ we can resort to homology, which also commutes with filtered colimits \cite[14.6]{may}. By connectedness we have $H_0(S^n_\Q) = \Z$ and for $i \neq 0, n$ we have $H_i(S^n) = 0$, so in these cases the homology of the colimit is also $\Z$ and resp. $0$. For $i = n$ we can use the same sequence as above (or use the Hurewicz theorem) to conclude:

$$ H_i(S^n_\Q) = \begin{cases}
	\Z, &\text{ if } i = 0 \\
	\Q, &\text{ if } i = n \\
	0,  &\text{ otherwise.}
\end{cases} $$

By the Serre-Hurewicz theorem (\TheoremRef{serre-hurewicz}, with $\C$ the class of uniquely divisible groups) we see that $S^n_\Q$ is indeed rational. Then by the Serre-Whitehead theorem (\TheoremRef{serre-whitehead}, with $\C$ the class of torsion groups) the inclusion map $S^n \to S^n_\Q$ is a rationalization.

\Corollary{rationalization-Sn}{
	The inclusion $S^n \to S^n_\Q$ is a rationalization.
}

The \Def{rational disk} is now defined as cone of the rational sphere: $D^{n+1}_\Q = CS^n_\Q$. By the naturality of the cone construction we get the following commutative diagram of inclusions.

\begin{displaymath}
	\xymatrix{
	S^{n} \arcof[r] \arcof[d] & S^{n}_{\Q} \arcof[d] \\
	D^{n+1} \arcof[r] & D^{n+1}_{\Q}
	}
\end{displaymath}

\Lemma{SnQ-extension}{
	Let $X$ be a rational space and $f : S^n \to X$ be a map. Then this map extends to a map $f' : S^n_\Q \to X$ making the following diagram commute.
	\begin{displaymath}
		\xymatrix{
		S^n \ar[r]^i \ar[rd]^f & S^n_\Q \ar@{-->}[d]^{f'} \\
		                       & X
		}
	\end{displaymath}
	Furthermore $f'$ is determined up to homotopy (i.e. any map $f''$ with $f''i = f$ is homotopic to $f'$) and homotopic maps have homotopic extensions (i.e. if $f \simeq g$, then $f' \simeq g'$).
}
\Proof{
	Note that $f$ represents a class $\alpha \in \pi_n(X)$. Since $\pi_n(X)$ is a $\Q$-vector space there exists elements $\frac{1}{2}\alpha, \frac{1}{3}\alpha, \ldots$ with representatives $\frac{1}{2}f, \frac{1}{3}f, \ldots$. Recall that $S^n_\Q$ consists of many copies of $S^n$, we can define $f'$ on the $k$th copy to be $\frac{1}{k!}f$, as shown in the following diagram.

	\cimage[scale=0.6]{SnQ_Extension}

	Since $[\frac{1}{(k-1)!}f] = k[\frac{1}{k}f] \in \pi_n(X)$ we can define $f'$ accordingly on the $n+1$-cells. Since our inclusion $i: S^n \cof S^n_\Q$ is in the first sphere, we get $f = f' \circ i$.

	Let $f''$ be any map such that $f''i = f$. Then \todo{finish proof}
}

\section{Rationalizations of arbitrary spaces}
Having rational cells we wish to replace the cells in a CW complex $X$ by the rational cells to obtain a rationalization.

\Lemma{rationalization-CW}{
	Any simply connected CW complex admits a rationalization.
}
\Proof{
	Let $X$ be a CW complex. We will define $X_\Q$ with induction on the dimension of the cells. Since $X$ is simply connected we can start with $X^0_\Q = X^1_\Q = \ast$. Now assume that the rationalization $X^k \tot{\phi^k} X^k_\Q$ is already defined. Let $A$ be the set of $k+1$-cells and $f_\alpha : S^k \to X^{k+1}$ be the attaching maps. Then by \LemmaRef{SnQ-extension} these extend to $g_\alpha = (\phi^k \circ f_\alpha)' : S^k_\Q \to X^k_\Q$. This defines $X^{k+1}_\Q$ as the pullback in the following diagram.

	\begin{displaymath}
		\xymatrix{
		\coprod_A S^n_\Q \ar[r]^{(g_\alpha)} \arcof[d] \xypo  &  X^k_\Q \ar@{-->}[d] \\
		\coprod_A D^{n+1}_\Q \ar@{-->}[r] & X^{k+1}_\Q
		}
	\end{displaymath}

	Now by the universal property of $X^{k+1}$, we get a map $\phi^{k+1} : X^{k+1} \to X^{k+1}_\Q$ which is compatible with $\phi^k$ and which is a rationalization.
}

\Lemma{rationalization}{
	Any simply connected space admits a rationalization.
}
\Proof{
	Let $Y \tot{f} X$ be a CW approximation and let $Y \tot{\phi} Y_\Q$ be the rationalization of $Y$. Now we define $X_\Q$ as the double mapping cylinder (or homotopy pushout):
	$$ X_\Q = X \cup_f (Y \times I) \cup_{\phi} Y_\Q. $$

	\todo{bewijs afmaken met excision?}
}

\Theorem{}{
	The above construction is in fact a \Def{localization}, i.e. for any map $f : X \to Z$ to a rational space $Z$, there is an extension $f' : X_\Q \to Z$ making the following diagram commute.

	\begin{displaymath}
		\xymatrix{
		X \ar[r]^i \ar[rd]^f & X_\Q \ar@{-->}[d]^{f'} \\
		                       & Z
		}
	\end{displaymath}

	Moreover, $f'$ is determined up to homotopy and homotopic maps have homotopic extensions.
}

The extension property allows us to define a rationalization of maps. Given $f : X \to Y$, we can consider the composite $if : X \to Y \to Y_\Q$. Now this extends to $(if)' : X_\Q \to Y_\Q$. Note that this construction is not functorial, since there are choices of homotopies involved. When passing to the homotopy category, however, this construction \emph{is} functorial and has an universal property.

We already mentioned in the first section that for rational spaces the notions of weak equivalence and rational equivalence coincide. Now that we always have a rationalization we have:

\Corollary{}{
	Let $f: X \to Y$ be a map, then $f$ is a rational equivalence if and only if $f_\Q : X_\Q \to Y_\Q$ is a weak equivalence.
}

\Corollary{}{
	The homotopy category of $1$-connected rational spaces is equivalent to the rational homotopy category of $1$-connected spaces.
}

\section{Other constructions}
There are others ways to obtain a rationalization. One of them relies on the observations that it is easy to rationalize Eilenberg-MacLane spaces. Since we already have a rationalization at hand the details in this section will be skipped and the focus lies on the construction.

\Remark{rationalization-em-space}{
	Let $A$ be an abelian group and $n \geq 1$. Then
	$$ K(A, n) \to K(A \tensor \Q, n) $$
	is a rationalization
}

Any simply connected space can be decomposed into a Postnikov tower $X \to \ldots \fib P_2(X) \fib P_1(X) \fib P_0(X)$ \cite[Chapter 22.4]{may}. Furthermore if $X$ is a simply connected CW complex, $P_{n}(X)$ can be constructed from $P_{n-1}(X)$ as the pushout in
\begin{displaymath}
	\xymatrix{
	P_{n}(X) \ar[r] \arfib[d] \xypb & PK(\pi_n(X), n+1) \arfib[d] \\
	P_{n-1}(X) \ar[r]^-{k_{n-1}} & K(\pi_n(X), n+1),
	}
\end{displaymath}
where the map $k_{n-1}$ is called the $k$-invariant. We will only need its existence for the construction. The rationalization can now be constructed with induction on this Postnikov tower. Start the induction with $X_\Q(2) = K(\pi_2(X) \tensor \Q, 2)$. Now assume we constructed $X_\Q(r-1)$ compatible with the $k$-invariant described above. We are in the following situation:
\begin{displaymath}
	\xymatrix@C=0cm@R=0.5cm{
		P_r(X) \ar[rr] \arfib[dd] & & PK(\pi_r(X), r+1) \ar[rd] \arfib[dd] & \\
		& & & PK(\pi_r(X) \tensor \Q, r+1) \arfib[dd] \\
		P_{r-1}(X) \ar[rr]^{k_{r-1}} \ar[rd]^{\phi_{r-1}} & & K(\pi_r(X), r+1) \ar[rd] \\
		& X_\Q(r-1) \ar[rr] & & K(\pi_r(X) \tensor \Q, r+1)
	}
\end{displaymath}
where the bottom square is our induction hypothesis, the right square is by naturality of the path space fibration and the back face is the pullback described above. We can define $X_\Q(r)$ to be the pullback of the front face, which induces a map $\phi_r : P_r(X) \to X_\Q(r)$. By inspecting the long exact sequence of the fibration $X_\Q(r) \fib X_\Q(r-1)$ we see that $\phi_r$ is indeed a rationalization.

We finish the construction by defining $X_\Q = \lim_r X_\Q(r)$. For more details, one can read \cite{sullivan} or \cite{berglund}.
