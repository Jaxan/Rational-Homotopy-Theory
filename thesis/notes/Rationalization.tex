
\chapter{Rationalizations}
\label{sec:rationalizations}

In this section we will prove the existence of rationalizations $X \to X_\Q$. We will do this in a cellular way. The $n$-spheres play an important role here, so their rationalizations will be discussed first. Again spaces (except for $S^1$) are assumed to be $1$-connected.

\section{Construction of \texorpdfstring{$S^n_\Q$}{SnQ}}
Fix $n>0$ we will construct the rationalization in stages, where at each stage we wedge a sphere and then glue a $n+1$-cell to ``invert'' some element in the $n$th homotopy group. At each stage the space will be homotopy equivalent to $S^n$.

\todo{Put this in a lemma. And make it more readable.}
We start with $S^n(1) = S^n$. Now assume $S^n(k)$ is constructed. Let $f: S^n \to S^n(k)$ be a representative for $1 \in \Z \iso \pi_n(S^n(k))$ and $g: S^n \to S^n$ be a representative for $k+1 \in \Z \iso \pi_n(S^n)$. These maps combine into $\phi: S^n \to S^n \vee S^n \tot{f \vee g} S^n(k) \vee S^n$. We define $S^n(k+1)$ as the pushout in the following diagram.

\begin{displaymath}
	\xymatrix{
	S^n \ar[r]^{\phi} \arcof[d] & S^n(k) \vee S^n \ar[d] \\
	D^{n+1} \ar[r] & S^n(k+1)
	}
\end{displaymath}

Note that $S^n(k+1)$ is homotopy equivalent to $S^n$. More importantly if we identify $\pi_n(S^n(k)) \iso \Z$ and $\pi_n(S^n(k+1)) \iso \Z$, then the inclusions $S^n(k) \subset S^n(k+1)$ induces multiplication by $k+1$ on the homotopy groups.

Now to finish we define the \Def{rational sphere} as $S^n_\Q = \colim_r S^n(r)$. Note that the homotopy groups commute with filtered colimits \cite[9.4]{may}, so that we can compute $\pi_n(S^n_\Q)$ as the colimit of the terms $\pi_n(S^n(r)) \iso \Z$ and the induced maps as depicted in the following diagram:

$$ \Z \tot{\cdots 2} \Z \tot{\cdots 3} \Z \tot{\cdots 4} \Z \tot{\cdots 5} \cdots \Q. $$

Moreover we note that the generator $1 \in \pi_n(S^n)$ is sent to $1 \in \pi_n(S^n_\Q$. However the other homotopy groups are harder to calculate as we have generally no idea how the induced maps will look like. But in the case of $n=1$, the other trivial homotopy groups of $S^1$ are trivial.

\Corollary{rationalization-S1}{
	The inclusion $S^1 \to S^1_\Q$ is a rationalization.
}

For $n>1$ we can resort to homology, which also commutes with filtered colimits \cite[14.6]{may}. By connectedness we have $H_0(S^n_\Q) = \Z$ and for $i \neq 0, n$ we have $H_i(S^n) = 0$, so in these cases the homology of the colimit is also $\Z$ and resp. $0$. For $i = n$ we can use the same sequence as above (or use the Hurewicz theorem) to conclude:

$$ H_i(S^n_\Q) = \begin{cases}
	\Z, &\text{ if } i = 0 \\
	\Q, &\text{ if } i = n \\
	0,  &\text{ otherwise.}
\end{cases} $$

By the Serre-Hurewicz theorem (\TheoremRef{serre-hurewicz}, with $\C$ the class of uniquely divisible groups) we see that $S^n_\Q$ is indeed rational. Then by the Serre-Whitehead theorem (\TheoremRef{serre-whitehead}, with $\C$ the class of torsion groups) the inclusion map $S^n \to S^n_\Q$ is a rationalization.

\Corollary{rationalization-Sn}{
	The inclusion $S^n \to S^n_\Q$ is a rationalization.
}

The \Def{rational disk} is now defined as cone of the rational sphere: $D^{n+1}_\Q = CS^n_\Q$. This gives an inclusions of pairs of spaces $(D^{n+1}, S^n) \subset (D^{n+1}_\Q, S^n_\Q)$.

\Lemma{SnQ-extension}{
	Let $X$ be a rational space and $f : S^n \to X$ be a map. Then this map extends to a map $f' : S^n_\Q \to X$ making the following diagram commute.
	\begin{displaymath}
		\xymatrix{
		S^n \ar[r]^i \ar[rd]^f & S^n_\Q \ar@{-->}[d]^{f'} \\
		                       & X
		}
	\end{displaymath}
}
\Proof{
	Note that $f$ represents a class $\alpha \in \pi_n(X)$. Since $\pi_n(X)$ is a $\Q$-vector space there are elements $\frac{1}{2}\alpha, \frac{1}{3}\alpha, \ldots$ with representatives $\frac{1}{2}f, \frac{1}{3}f, \ldots$. Recall that $S^n_\Q$ consists of many copies of $S^n$, we can define $f'$ on the $k$th copy to be $\frac{1}{k!}f$, as shown in the following diagram.

	\cimage[scale=0.6]{SnQ_Extension}

	Since $[\frac{1}{(k-1)!}f] = k[\frac{1}{k}f] \in \pi_n(X)$ we can extend on the $n+1$-cells. This defines $f'$. Since our inclusion $i: S^n \cof S^n_\Q$ is in the first sphere, we get $f = f' \circ i$.
}
\todo{Add: unique up to homotopy. A homotopy extends to a homotopy.}

\section{Rationalizations of arbitrary spaces}
Having rational cells we wish to replace the cells in a CW complex $X$ by the rational cells to obtain a rationalization.

\Lemma{rationalization-CW}{
	Any CW complex admits a rationalization.
}
\Proof{
	Let $X$ be a CW complex. We will define $X_\Q$ with induction on the dimension of the cells. Since $X$ is simply connected we can start with $X^0_\Q = X^1_\Q = \ast$. Now assume that the rationalization $X^k \tot{\phi^k} X^k_\Q$ is already defined. Let $A$ be the set of $k+1$-cells and $f_\alpha : S^k \to X^{k+1}$ be the attaching maps. Then by \LemmaRef{SnQ-extension} these extend to $g_\alpha = (\phi^k \circ f_\alpha)' : S^k_\Q \to X^k_\Q$. This defines $X^{k+1}_\Q$ as the pushout in the following diagram.

	\begin{displaymath}
		\xymatrix{
		\coprod_A S^n_\Q \ar[r]^{(g_\alpha)} \arcof[d] \xypo  &  X^k_\Q \ar@{-->}[d] \\
		\coprod_A D^{n+1}_\Q \ar@{-->}[r] & X^{k+1}_\Q
		}
	\end{displaymath}

	Now by the universal property of $X^{k+1}$, we get a map $\phi^{k+1} : X^{k+1} \to X^{k+1}_\Q$ which is compatible with $\phi^k$ and which is a rationalization.
}

\todo{For arbitrary spaces}

\Theorem{}{
	The above construction is in fact a \Def{localization}, i.e. for any map $f : X \to Z$ to a rational space $Z$, there is an extension $f' : X_\Q \to Z$ making the following diagram commute.

	\begin{displaymath}
		\xymatrix{
		X \ar[r]^i \ar[rd]^f & X_\Q \ar@{-->}[d]^{f'} \\
		                       & Z
		}
	\end{displaymath}

	Moreover, any $f''$ making the diagram commute is homotopic to $f'$ and if $g : X \to Z$ is homotopic to $f$ then the extension $g'$ is homotopic to $f'$.
}

The extension property allows us to define a rationalization of maps. Given $f : X \to Y$, we can consider the composite $if : X \to Y \to Y_\Q$. Now this extends to $(if)' : X_\Q \to Y_\Q$. Note that this construction is not functorial, since there are choices of homotopies involved. When passing to the homotopy category, however, this construction \emph{is} functorial and has an universal property.

We already mentioned in the first section that for rational spaces the notions of weak equivalence and rational equivalence coincide. Now that we always have a rationalization we have:

\Corollary{}{
	Let $f: X \to Y$ be a map, then $f$ is a rational equivalence if and only if $f_\Q : X_\Q \to Y_\Q$ is a weak equivalence.
}

\Corollary{}{
	The homotopy category of $1$-connected rational spaces is equivalent to the rational homotopy category of $1$-connected spaces.
}

\section{Other constructions}
There are others ways to obtain a rationalization. One of them relies on the observations that it is easy to rationalize Eilenberg-MacLane spaces.

\Lemma{rationalization-em-space}{
	Let $A$ be an abelian group and $n \geq 1$. Then
	$$ K(A, n) \to K(A \tensor \Q, n) $$
	is a rationalization
}

Postnikov
