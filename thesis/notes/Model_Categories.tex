
\section{Model categories}
\label{sec:model_categories}

As this thesis considers different categories, each with its own homotopy theory, it is natural to use Quillen's formalism of model categories. Not only gives this the right definition of the associated homotopy category, it also gives existence of lifts and lifts of homotopies.

\newcommand{\W}{\mathfrak{W}}
\newcommand{\Fib}{\mathfrak{Fib}}
\newcommand{\Cof}{\mathfrak{Cof}}

\begin{definition}
	A \Def{model category} is a category $\cat{C}$ together with three subcategories:
	\begin{itemize}
		\item the class of \Def{weak equivalences} $\W$,
		\item the class of \Def{fibrations} $\Fib$ and
		\item the class of \Def{cofibrations} $\Cof$,
	\end{itemize}
	such that the following five axioms hold:
	\begin{itemize}
		\item[MC1] All finite limits and colimits exist in $\cat{C}$.
		\item[MC2] If $f$, $g$ and $fg$ are maps such that two of them are weak equivalences, then so it the third. This is called the \emph{2-out-of-3} property.
		\item[MC3] All three classes of maps are closed under retracts\todo{Either draw the diagram or define a retract earlier}.
		\item[MC4] In any commuting square as follows where $i \in \Cof$ and $p \in \Fib$,
			\cdiagram{Model_Liftproblem}

			there exist a lift $h: B \to Y$ if either 
			\begin{itemize}
				\item[a)] $i \in \W$ or
				\item[b)] $p \in \W$.
			\end{itemize}
		\item[MC5] Any map $f : A \to B$ can be factored in two ways:
			\begin{itemize}
				\item[a)] as $f = pi$, where $i \in \Cof \cap \W$ and $p \in \Fib$ and
				\item[b)] as $f = pi$, where $i \in \Cof$ and $p \in \Fib \cap \W$.
			\end{itemize}
	\end{itemize}
\end{definition}

\Notation{model-cats-arrows}{
	For brevity
	\begin{itemize}
		\item we write $f: A \fib B$ when $f$ is a fibration,
		\item we write $f: A \cof B$ when $f$ is a cofibration and
		\item we write $f: A \we B$ when $f$ is a weak equivalence.
	\end{itemize}
	Furthermore a map which is a fibration and a weak equivalence is called a \Def{trivial fibration}, similarly we have \Def{trivial cofibration}.
}

\Definition{model-cats-fibrant-cofibrant}{
	An object $A$ in a model category $\cat{C}$ will be called \Def{fibrant} if $A \to \cat{1}$ is a fibration and \Def{cofibrant} if $\cat{0} \to A$ is a cofibration.
}

Note that axiom [MC5a] allows us to replace any object $X$ with a weakly equivalent fibrant object $X^{fib}$ and by [MC5b] by a weakly equivalent cofibrant object $X^{cof}$, as seen in the following diagram:

\cdiagram{Model_Replacements}

The fourth axiom actually characterizes the classes of (trivial) fibrations and (trivial) cofibrations. We will abbreviate left lifting property with LLP and right lifting property with RLP. We will not prove these statements, but only expose them because we use them throughout this thesis. One can find proofs in \cite{dwyer, may}.

\Lemma{model-cats-characterization}{
	Let $\cat{C}$ be a model category.
	\begin{itemize}
		\item The cofibrations in $\cat{C}$ are the maps with a LLP w.r.t. trivial fibrations.
		\item The fibrations in $\cat{C}$ are the maps with a RLP w.r.t. trivial cofibrations.
		\item The trivial cofibrations in $\cat{C}$ are the maps with a LLP w.r.t. fibrations.
		\item The trivial fibrations in $\cat{C}$ are the maps with a RLP w.r.t. cofibrations.
	\end{itemize}
}

This means that once we choose weak equivalences and fibrations for a category $\cat{C}$, the third class is determined, and vice versa. The classes of fibrations behave nice with respect to pullbacks and dually cofibrations behave nice with pushouts:

\Lemma{model-cats-pushouts}{
	Let $\cat{C}$ be a model category. Consider the following two diagrams where $P$ is the pushout and pullback respectively.

	\cdiagram{Model_Cats_Pushouts}

	\begin{itemize}
		\item If $i$ is a (trivial) cofibration, so is $j$.
		\item If $p$ is a (trivial) fibration, so is $q$.
	\end{itemize}
}

\Lemma{model-cats-coproducts}{
	Let $\cat{C}$ be a model category. Let $f: A \cof B$ and $g:A' \cof B'$ be two (trivial) cofibrations, then the induced map of the coproducts $f+g: A+A' \to B+B'$ is also a (trivial) cofibration. Dually: the product of two (trivial) fibrations is a (trivial) fibration.
}

\TODO{Maybe some basic propositions (refer to Dwyer \& Spalinski):
\titem Over/under category (or simply pointed objects)
\titem Cofibrantly generated mod. cats.
\titem Small object argument
}

Of course the most important model category is the one of topological spaces. We will be interested in the standard model structure on topological spaces, which has weak homotopy equivalences as weak equivalences. Equally important is the model category of simplicial sets.

\Example{top-model-structure}{
	The category $\Top$ of topological spaces admits a model structure as follows.
	\begin{itemize}
		\item Weak equivalences: maps inducing isomorphisms on all homotopy groups.
		\item Fibrations: Serre fibrations, i.e. maps with the right lifting property with respect to the inclusions $D^n \cof D^n \times I$.
		\item Cofibrations: maps $S^{n-1} \cof D^n$ and transfinite compositions of pushouts and coproducts thereof.
	\end{itemize}
}

\Example{sset-model-structure}{
	The category $\sSet$ of simplicial sets has the following model structure.
	\begin{itemize}
		\item Weak equivalences: 
		\item Fibrations: Kan fibrations, i.e. maps with the right lifting property with respect to the inclusions $\Lambda_n^k \cof \Delta[n]$.
		\item Cofibrations: all monomorphisms.
	\end{itemize}
}

In this thesis we often restrict to $1$-connected spaces. The full subcategory $\Top_1$ of $1$-connected spaces satisfies MC2-MC5: the 2-out-of-3 property, retract property and lifting properties hold as we take the \emph{full} subcategory, factorizations exist as the middle space is $1$-connected as well. However $\Top_1$ does not have all limits and colimits.

\Lemma{topr-no-colimit}{
	Let $r > 0$ and $\Top_r$ be the full subcategory of $r$-connected spaces. The diagrams

	\cdiagram{Topr_No_Coequalizer}

	have no coequalizer and respectively no equalizer in $\Top_r$.
}

\subsection{Homotopies}
So far we have only seen equivalences between objects of the category. We can, however, also define homotopy relations between maps (as we are used to in $\Top$). There are two such construction, which will coincide on nice objects. We will only state the definitions and important results. One can find proofs of these results in \cite{dwyer}. Throughout this section we silently work with a fixed model category $\cat{C}$.

\newcommand{\cylobj}[1]{Cyl_{#1}}
\newcommand{\pathobj}[1]{Path_{#1}}
\newcommand{\lhtpy}{{\sim^{l}}}
\newcommand{\rhtpy}{{\sim^{r}}}
\newcommand{\lhtpycl}{{\pi^l}}
\newcommand{\rhtpycl}{{\pi^r}}

\Definition{cylinder_object}{
	A \Def{cylinder object} for an object $A$ is an object $\cylobj{A}$ together with maps:
	$$ A \coprod A \tot{i} \cylobj{A} \we^{p} A, $$
	which factors the folding map $\id_A + \id_A: A \coprod A \to A$ (note that we use MC1 here). The cylinder object is called
	\begin{itemize}
		\item \emph{good} if $i$ is a cofibration and
		\item \emph{very good} if in addition $p$ is a fibration.
	\end{itemize}
}
\Notation{cylinder_maps}{
	The map $i$ consists of two factors, which we will denote $i_0$ and $i_1$.
}

Note that we do not require cylinder objects to be functorial. There can also be more than one cylinder object for $A$. Cylinder objects can now be used to define left homotopies.

\Definition{left_homotopy}{
	Two maps $f, g: A \to X$ are \Def{left homotopic} if there exists a cylinder object $\cylobj{A}$ and a map $H: \cylobj{A} \to X$ such that $H \circ i_0 = f$ and $H \circ i_1 = g$.

	We will call $H$ a \Def{left homotopy} for $f$ to $g$ and write $f \lhtpy r$. Moreover, the homotopy is called good (resp. very good) is the cylinder object is good (resp. very good).
}

Note that the relation need not be transitive: consider $f \lhtpy g$ and $g \lhtpy h$, then these homotopies may be defined on different cylinder objects and in general we cannot relate the cylinder objects. However for nice domains $\lhtpy$ will be an equivalence relation.

\Lemma{left_homotopy_eqrel}{
	If $A$ is cofibrant, then $\lhtpy$ is an equivalence relation on $\Hom_\cat{C}(A, X)$.
}
\Definition{left_homotopy_classes}{
	We will denote the set of \Def{left homotopy classes} as
	$$ \lhtpycl(A, X) = \Hom_\cat{C}(A, X) / \lhtpy', $$
	where $\lhtpy'$ is the equivalence relation generated by $\lhtpy$.
}

\Lemma{left_homotopy_properties}{
	We have the following properties
	\begin{itemize}
		\item If $A$ is cofibrant and $p: X \to Y$ a trivial fibration, then
		$$ p_\ast : \lhtpycl(A, X) \tot{\iso} \lhtpycl(A, Y). $$
		\item If $X$ is fibrant, $f \lhtpy g: B \to X$ and we have a map $h: A \to B$, then
		$$ fh \lhtpy gh. $$
	\end{itemize}
}

Of course there is a completely dual definition of right homotopy, in terms of path objects. All of the above also applies (but in a dual way).

\Definition{path_object}{
	A \Def{path object} for an object $X$ is an object $\pathobj{X}$ together with maps:
	$$ X \we^{i} \pathobj{X} \tot{p} X \times X, $$
	which factors the diagonal map $(\id_X, \id_X): X \to X \times X$. The path object is called
	\begin{itemize}
		\item \emph{good} if $p$ is a fibration and
		\item \emph{very good} if in addition $i$ is a cofibration.
	\end{itemize}
}
\Notation{cylinder_maps}{
	The map $p$ consists of two factors, which we will denote $p_0$ and $p_1$.
}

\Definition{right_homotopy}{
	Two maps $f, g: A \to X$ are \Def{right homotopic} if there exists a path object $\pathobj{X}$ and a map $H: A \to \pathobj{X}$ such that $p_0 \circ H = f$ and $p_1 \circ H = g$.

	We will call $H$ a \Def{right homotopy} for $f$ to $g$ and write $f \rhtpy r$. Moreover, the homotopy is called good (resp. very good) is the path object is good (resp. very good).
}

\Lemma{right_homotopy_eqrel}{
	If $X$ is fibrant, then $\rhtpy$ is an equivalence relation on $\Hom_\cat{C}(A, X)$.
}
\Definition{right_homotopy_classes}{
	We will denote the set of \Def{left homotopy classes} as
	$$ \rhtpycl(A, X) = \Hom_\cat{C}(A, X) / \rhtpy', $$
	where $\rhtpy'$ is the equivalence relation generated by $\rhtpy$.
}

\Lemma{right_homotopy_properties}{
	We have the following properties
	\begin{itemize}
		\item If $X$ is fibrant and $i: A \to B$ a trivial cofibration, then
		$$ i^\ast : \rhtpycl(B, X) \tot{\iso} \rhtpycl(A, X). $$
		\item If $A$ is cofibrant, $f \rhtpy g: A \to X$ and we have a map $h: X \to Y$, then
		$$ hf \rhtpy hg. $$
	\end{itemize}
}

The two notions (left resp. right homotopy) agree on nice objects. Hence in this case we can speak of homotopic maps.

\Lemma{homotopy}{
	Let $f, g: A \to X$ be two maps and $A$ cofibrant and $X$ fibrant, then
	$$ f \lhtpy g \iff f \rhtpy g. $$
}

\Definition{homotopy}{
	In the above case we say that $f$ and $g$ are \Def{homotopic}, this is denoted by $f \sim g$. Furthermore we can define the set of homotopy classes as:
	$$ [A, X] = \Hom_\cat{C}(A, X) / \sim. $$

	A map $f: A \to X$ between cofibrant-fibrant objects is said to have a \Def{homotopy inverse} if there exists $g: X \to A$ such that $fg \sim \id$ and $gf \sim \id$. We will also call $f$ a \Def{strong homotopy equivalence}.
}

\Lemma{weak_strong_homotopy}{
	Let $f: A \to B$ be a map between cofibrant-fibrant objects, then:
	$$ f \text{ is a weak equivalence } \iff f \text{ is a strong equivalence }. $$
}


\subsection{The Homotopy Category \texorpdfstring{$\Ho(\cat{C})$}{Ho(C)}}
A model category induces a homotopy category $\Ho(\cat{C})$, in which weak equivalences are isomorphisms and homotopic maps are equal. This category only depens on the category $\cat{C}$ and the class of weak equivalences.
\todo{Definition etc}

\subsection{Quillen pairs}
In order to relate model categories and their associated homotopy categories we need a notion of maps between them. We want the maps such that they induce maps on the homotopy categories.
\todo{Definition etc}
