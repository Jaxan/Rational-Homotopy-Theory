
\section{Model categories}
\label{sec:model_categories}

\newcommand{\W}{\mathfrak{W}}
\newcommand{\Fib}{\mathfrak{Fib}}
\newcommand{\Cof}{\mathfrak{Cof}}

\begin{definition}
	A \Def{model category} is a category $\cat{C}$ together with three subcategories:
	\begin{itemize}
		\item the class of \Def{weak equivalences} $\W$,
		\item the class of \Def{fibrations} $\Fib$ and
		\item the class of \Def{cofibrations} $\Cof$,
	\end{itemize}
	such that the following five axioms hold:
	\begin{itemize}
		\item[MC1] All finite limits and colimits exist in $\cat{C}$.
		\item[MC2] If $f$, $g$ and $fg$ are maps such that two of them are weak equivalences, then so it the third. This is called the \emph{2-out-of-3} property.
		\item[MC3] All three classes of maps are closed under retracts\todo{Either draw the diagram or define a retract earlier}.
		\item[MC4] In any commuting square as follows where $i \in \Cof$ and $p \in \Fib$,
			\begin{center}
			\begin{tikzpicture}
			\matrix (m) [matrix of math nodes]{
				A & X \\
				B & Y \\
			};

			\path[->] (m-1-1) edge (m-1-2);
			\path[->] (m-2-1) edge (m-2-2);
			\path[->] (m-1-1) edge node[auto] {$i$} (m-2-1);
			\path[->] (m-1-2) edge node[auto] {$p$} (m-2-2);

			\end{tikzpicture}
			\end{center}

			there exist a lift $h: B \to Y$ if either 
			\begin{itemize}
				\item[a)] $i \in \W$ or
				\item[b)] $p \in \W$.
			\end{itemize}
		\item[MC5] Any map $f : A \to B$ can be factored in two ways:
			\begin{itemize}
				\item[a)] as $f = pi$, where $i \in \Cof \cap \W$ and $p \in \Fib$ and
				\item[b)] as $f = pi$, where $i \in \Cof$ and $p \in \Fib \cap \W$.
			\end{itemize}
	\end{itemize}
\end{definition}

\Notation{model-cats-arrows}{
	For brevity
	\begin{itemize}
		\item we write $f: A \fib B$ when $f$ is a fibration,
		\item we write $f: A \cof B$ when $f$ is a cofibration and
		\item we write $f: A \we B$ when $f$ is a weak equivalence.
	\end{itemize}
	Furthermore a map which is a fibration and a weak equivalence is called a \Def{trivial fibration}, similarly we have \Def{trivial cofibration}.
}

\Definition{model-cats-fibrant-cofibrant}{
	An object $A$ in a model category $\cat{C}$ will be called \Def{fibrant} if $A \to \cat{1}$ is a fibration and \Def{cofibrant} if $\cat{0} \to A$ is a cofibration.
}

Note that axiom [MC5a] allows us to replace any object $X$ with a weakly equivalent fibrant object $X^{fib}$ and by [MC5b] by a weakly equivalent cofibrant object $X^{cof}$, as seen in the following diagram:

\begin{center}
\begin{tikzpicture}
\matrix (m) [matrix of math nodes]{
	\cat{0} &         & X \\
	        & X^{cof} &   \\
};

\path[->] (m-1-1) edge (m-1-3);
\path[right hook->] (m-1-1) edge (m-2-2);
\path[->>] (m-2-2) edge node[auto] {$ \simeq $} (m-1-3);

\end{tikzpicture}\quad
\begin{tikzpicture}
\matrix (m) [matrix of math nodes]{
	X &         & \cat{1} \\
	  & X^{fib} &         \\
};

\path[->] (m-1-1) edge (m-1-3);
\path[right hook->] (m-1-1) edge node[auto] {$ \simeq $} (m-2-2);
\path[->>] (m-2-2) edge (m-1-3);

\end{tikzpicture}
\end{center}

The fourth axiom actually characterizes the classes of (trivial) fibrations and (trivial) cofibrations. We will abbreviate left lifting property with LLP and right lifting property with RLP. We will not prove these statements, but only expose them because we use them throughout this thesis. One can find proofs in \cite{dwyer, may}.

\Lemma{model-cats-characterization}{
	Let $\cat{C}$ be a model category.
	\begin{itemize}
		\item The cofibrations in $\cat{C}$ are the maps with a LLP w.r.t. trivial fibrations.
		\item The fibrations in $\cat{C}$ are the maps with a RLP w.r.t. trivial cofibrations.
		\item The trivial cofibrations in $\cat{C}$ are the maps with a LLP w.r.t. fibrations.
		\item The trivial fibrations in $\cat{C}$ are the maps with a RLP w.r.t. cofibrations.
	\end{itemize}
}

This means that once we choose weak equivalences and fibrations for a category $\cat{C}$, the third class is determined, and vice versa. The classes of fibrations behave nice with respect to pullbacks and dually cofibrations behave nice with pushouts:

\Lemma{model-cats-pushouts}{
	Let $\cat{C}$ be a model category. Consider the following two diagrams where $P$ is the pushout and pullback respectively.

	\cimage{Model_Cats_Pushouts}

	\begin{itemize}
		\item If $i$ is a (trivial) cofibrations, so is $j$.
		\item If $p$ is a (trivial) fibrations, so is $q$.
	\end{itemize}
}

\Lemma{model-cats-coproducts}{
	Let $\cat{C}$ be a model category. Let $f: A \cof B$ and $g:A' \cof B'$ be two (trivial) cofibrations, then the induced map of the coproducts $f+g: A+A' \to B+B'$ is also a (trivial) cofibration. Dually: the product of two (trivial) fibrations is a (trivial) fibration.
}

\TODO{Maybe some basic propositions (refer to Dwyer \& Spalinski):
\titem Over/under category (or simply pointed objects)
\titem Cofibrantly generated mod. cats.
\titem Small object argument
}

\Example{top-model-structure}{
	The category $\Top$ of topological spaces admits a model structure as follows.
	\begin{itemize}
		\item Weak equivalences: maps inducing isomorphisms on all homotopy groups.
		\item Fibrations: Serre fibrations, i.e. maps with the right lifting property with respect to the inclusions $D^n \cof D^n \times I$.
		\item Cofibrations: maps $S^{n-1} \cof D^n$ and transfinite compositions of pushouts and coproducts thereof.
	\end{itemize}
}

\Example{sset-model-structure}{
	The category $\sSet$ of simplicial sets has the following model structure.
	\begin{itemize}
		\item Weak equivalences: 
		\item Fibrations: Kan fibrations, i.e. maps with the right lifting property with respect to the inclusions $\Lambda_n^k \cof \Delta[n]$.
		\item Cofibrations: all monomorphisms.
	\end{itemize}
}

In this thesis we often restrict to $1$-connected spaces. The full subcategory $\Top_1$ of $1$-connected spaces satisfies MC2-MC5: the 2-out-of-3 property, retract property and lifting properties hold as we take the \emph{full} subcategory, factorizations exist as the middle space is $1$-connected as well. However $\Top_1$ does not have all limits and colimits.

\Lemma{topr-no-colimit}{
	Let $r > 0$ and $\Top_r$ be the full subcategory of $r$-connected spaces. The diagrams

	\cimage[scale=0.5]{Topr_No_Coequalizer}
	\cimage[scale=0.5]{Topr_No_Equalizer}

	have no coequalizer and respectively no equalizer in $\Top_r$.
}

\todo{Define homotopy category}

\subsection{Quillen pairs}
In order to relate model categories and their associated homotopy categories we need a notion of maps between them. We want the maps such that they induce maps on the homotopy categories.
\todo{Definition etc}
