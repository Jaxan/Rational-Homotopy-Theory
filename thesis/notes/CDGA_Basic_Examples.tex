
We will first define some basic cochain complexes which model the $n$-disk and $n$-sphere. $D(n)$ is the cochain complex generated by one element $b \in D(n)^n$ and its differential $c = d(b) \in D(n)^{n+1}$. On the other hand we define $S(n)$ to be the cochain complex generated by one element $a \in S(n)^n$ with trivial differential (i.e. $d a = 0$). In other words:
$$ D(n) = ... \to 0 \to \k \to \k \to 0 \to ... $$
$$ S(n) = ... \to 0 \to \k \to 0 \to 0 \to ... $$

Note that $D(n)$ is acyclic for all $n$, or put in different words: $j_n : 0 \to D(n)$ induces an isomorphism in cohomology. The sphere $S(n)$ has exactly one non-trivial cohomology group $H^n(S(n)) = \k \cdot [a]$. There is an injective function $i_n : S(n+1) \to D(n)$, sending $a$ to $c$. The maps $j_n$ and $i_n$ play the following important role in the model structure of cochain complexes, where weak equivalences are quasi isomorphisms, fibrations are degreewise surjective and cofibrations are degreewise injective for positive degrees \cite[Example 1.6]{goerss2}.

\begin{claim}
	The set $I = \{i_n : S(n+1) \to D(n) \I n \in \N\}$ generates all cofibrations and the set $J = \{j_n : 0 \to D(n) \I n \in \N\}$ generates all trivial cofibrations.
\end{claim}

As we do not directly need this claim, we omit the proof. However, in the next section we will prove a similar result for cdga's in detail.

$S(n)$ plays a another special role: maps from $S(n)$ to some cochain complex $X$ correspond directly to elements in the kernel of $\restr{d}{X^n}$. Any such map is null-homotopic precisely when the corresponding elements in the kernel is a coboundary. So there is a natural isomorphism: $\Hom(S(n), X) / {\simeq} \iso H^n(X)$.

By using the free cdga functor we can turn these cochain complexes into cdga's $\Lambda D(n)$ and $\Lambda S(n) $. So $\Lambda D(n)$ consists of linear combinations of $b^k$ and $c b^k$ when $n$ is even, and it consists of linear combinations of $c^k b$ and $c^k$ when $n$ is odd. In both cases we can compute the differentials using the Leibniz rule:
$$ d(b^k) = k \cdot c b^{k-1} $$
$$ d(c b^k) = 0 $$

$$ d(c^k b) = c^{k+1} $$
$$ d(c^k) = 0 $$

Those cocycles are in fact coboundaries (using that $\k$ is a field of characteristic $0$):
$$ c b^k = \frac{1}{k} d(b^{k+1}) $$
$$ c^k = d(b c^{k-1}) $$

There are no additional cocycles in $\Lambda D(n)$ besides the constants and $c$. So we conclude that $\Lambda D(n)$ is acyclic as an augmented algebra. In other words $\Lambda(j_n): \k \to \Lambda D(n)$ is a quasi isomorphism.

The situation for $\Lambda S(n)$ is easier as it has only one generator (as algebra). For even $n$ this means it is given by polynomials in $a$. For odd $n$ it is an exterior algebra, meaning $a^2 = 0$. Again the sets $\Lambda(I) = \{ \Lambda(i_n) : \Lambda S(n+1) \to \Lambda D(n) \I n \in \N\}$ and $\Lambda(J) = \{ \Lambda(j_n) : \k \to \Lambda D(n) \I n \in \N\}$ play an important role.

\begin{theorem}
	The sets $\Lambda(I)$ and $\Lambda(J)$ generate a model structure on $\CDGA_\k$ where:
	\begin{itemize}
		\item weak equivalences are quasi isomorphisms,
		\item fibrations are (degree wise) surjective maps and
		\item cofibrations are maps with the left lifting property against trivial fibrations.
	\end{itemize}
\end{theorem}

We will prove this theorem in the next section. Note that the functors $\Lambda$ and $U$ thus form a Quillen pair with this model structure.

\subsection{Why we need $\Char{\k} = 0$ for algebras}
The above Quillen pair $(\Lambda, U)$ fails to be a Quillen pair if $\Char{\k} = p \neq 0$. We will show this by proving that the maps $\Lambda(j_n)$ are not weak equivalences for even $n$. Consider $b^p \in \Lambda D(n)$, then by the Leibniz rule:
$$ d(b^p) = p \cdot c b^{p-1} = 0. $$
So $b^p$ is a cocycle. Now assume $b^p = d x$ for some $x$ of degree $p n - 1$, then $x$ contains a factor $c$ for degree reasons. By the calculations above we see that any element containing $c$ has a trivial differential or has a factor $c$ in its differential, contradicting $b^p =  d x$. So this cocycle is not a coboundary and $\Lambda D(n)$ is not acyclic.
