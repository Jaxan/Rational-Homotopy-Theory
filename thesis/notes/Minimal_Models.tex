
\Chapter{Minimal models}{MinimalModels}
\label{sec:minimal-models}

In this section we will discuss the so called minimal models. These cdga's enjoy the property that we can easily prove properties inductively. Moreover it will turn out that weakly equivalent minimal models are actually isomorphic.

\Definition{minimal-algebra}{
	A cdga $(A, d)$ is a \Def{Sullivan algebra} if
	\begin{itemize}
		\item $A = \Lambda V$ is free as a commutative graded algebra, and
		\item $V$ has a filtration
		$$ 0 = V(-1) \subset V(0) \subset V(1) \subset \cdots \subset \bigcup_{k \in \N} V(k) = V, $$
		such that $d(V(k)) \subset \Lambda V(k-1)$.
	\end{itemize}

	A cdga $(A, d)$ is a \Def{minimal Sullivan algebra} if in addition
	\begin{itemize}
		\item $d$ is decomposable, i.e. $\im(d) \subset \Lambda^{\geq 2}V$.
	\end{itemize}
}

\begin{definition}
	Let $(A, d)$ be any cdga. A \Def{(minimal) Sullivan model} is a (minimal) Sullivan algebra $(M, d)$ with a weak equivalence:
	$$ (M, d) \we (A, d). $$
\end{definition}

We will often say \Def{minimal model} or \Def{minimal algebra} to mean minimal Sullivan model or minimal Sullivan algebra. Note that a minimal algebra is naturally augmented as it is free as an algebra. This will be used implicitly. In many cases we can take the degree of the elements in $V$ to induce the filtration, as seen in the following lemma.

\Lemma{1-reduced-minimal-model}{
	Let $(A, d)$ be a cdga which is $1$-reduced, such that $A = \Lambda V$ is free as cga. Then the differential $d$ is decomposable if and only if $(A, d)$ is a Sullivan algebra filtered by degree.
}
\Proof{
	Let $V$ be filtered by degree: $V(k) = V^{\leq k}$. Now $d(v) \in \Lambda V^{< k}$ for any $v \in V^k$. For degree reasons $d(v)$ is a product, so $d$ is decomposable.

	For the converse take $V(n) = V^{\leq n}$ (note that $V^0 = V^1 = 0$). Since $d$ is decomposable we see that for $v \in V^n$: $d(v) = x \cdot y$ for some $x, y \in A$. Assuming $dv$ to be non-zero we can compute the degrees:
	$$ \deg{x} + \deg{y} = \deg{xy} = \deg{dv} = \deg{v} + 1 = n + 1. $$
	As $A$ is $1$-reduced we have $\deg{x}, \deg{y} \geq 2$ and so by the above $\deg{x}, \deg{y} \leq n-1$. Conclude that $d(V(k)) \subset \Lambda(V(n-1))$.
}

Minimal models admit very nice homotopy groups. Note that for a minimal algebra $\Lambda V$ there is a natural augmentation and the the differential is decomposable. Hence $Q \Lambda V$ is naturally isomorphic to $(V, 0)$. In particular the homotopy groups are simply given by $\pi^n(\Lambda V) = V^n$.

\DefinitionRef{minimal-algebra} is the same as in \cite{felix} without assuming connectivity. We find some different definitions of (minimal) Sullivan algebras in the literature. For example we find a definition using well orderings in \cite{hess}. The decomposability of $d$ also admits a different characterization (at least in the connected case). The equivalence of the definitions is expressed in the following two lemmas. The first can be easily proven by choosing subspaces with bases $V'_k = \langle v_j \rangle_{j \in J_k}$ such that $V(k) = V(k-1) \oplus V'_k$ for each degree. Then choose some well order on $J_k$ to define a well order on $J = \bigcup_k J_k$. The second lemma is a more refined version of \LemmaRef{1-reduced-minimal-model}. Since we will not need these equivalent definitions, the details are left out.

\Lemma{sullivan-hess}{
	A cdga $(\Lambda V, d)$ is a Sullivan algebra if and only if there exists a well order $J$ such that $V$ is generated by $v_j$ for $j \in J$ and $d v_j \in \Lambda V_{<j}$.
}

\Lemma{minimal-hess}{
	Let $(\Lambda V, d)$ be a Sullivan algebra with $V^0 = 0$, then $d$ is decomposable if and only if there is a well order $J$ as above such that $i < j$ implies $\deg{v_i} \leq \deg{v_j}$.
}

It is clear that induction will be an important technique when proving things about (minimal) Sullivan algebras. We will first prove that minimal models always exist for $1$-connected cdga's and afterwards prove uniqueness.

\section{Existence}

\begin{theorem}
	Let $(A, d)$ be a $1$-connected cdga, then it has a minimal model $(\Lambda V, d)$.
\end{theorem}
\begin{proof}
	We construct the model and by induction on the degree. The resulting filtration will be on degree, so that the minimality follows from \LemmaRef{1-reduced-minimal-model}. We start by setting $V^0 = V^1 = 0$ and $V^2 = H^2(A)$. At this stage the differential is trivial, i.e. $d(V^2) = 0$. Sending the cohomology classes to their representatives extends to a map of cdga's $m_2 : \Lambda V^{\leq 2} \to A$.

	Suppose $m_k : \Lambda V^{\leq k} \to A$ is constructed. We will add elements in degree $k+1$ and extend $m_k$ to $m_{k+1}$ to assert surjectivity and injectivity of $H(m_{k+1})$.\, Let $\{ [a_\alpha] \}_{\alpha \in I}$ be a basis for the cokernel of $H(m_k) : H^{k+1}(\Lambda V^{\leq k}) \to H^{k+1}(A)$ and $b_\alpha \in A^{k+1}$ be a representing cycle for $a_\alpha$.\, Let $\{ [z_\beta] \}_{\beta \in J}$ be a basis for the kernel of $H(m_k) : H^{k+2}(\Lambda V^{\leq k}) \to H^{k+2}(A)$, note that $m_k(b_\beta)$ is a boundary, so that there are elements $c_\beta$ such that $m_k(b_\beta) = d c_\beta$.

	Define $V^{k+1} = \bigoplus_{\alpha \in I} \k \cdot v_\alpha \oplus \bigoplus_{\beta \in J} \k \cdot v'_\beta$ and extend $d$ and $m_{k+1}$ by defining
	\[ d(v_\alpha) = 0 \qquad d(v'_\beta) = z_\beta \]
	\[ m_{k+1}(v_\alpha) = b_\alpha \qquad m_{k+1}(v'_\beta) = c_\beta \]
	Now clearly $d^2=0$ on the generators, so this extends to a derivation on $\Lambda V^{\leq k+1}$, similarly $m_{k+1}$ commutes with $d$ on the generators and hence extends to a chain map.

	This finished the construction of $V$ and $m : \Lambda V \to A$. Now we will prove that $H(m)$ is an isomorphism. We will do so by proving surjectivity and injectivity by induction on $k$.

	Start by noting that $H^i(m_2)$ is surjective for $i \leq 2$. Now assume by induction that $H^i(m_k)$ is surjective for $i \leq k$. Since $\im H(m_k) \subset \im H(m_{k+1})$ we see that $H^i(m_{k+1})$ is surjective for $i < k+1$. By construction it is also surjective in degree $k+1$. So $H^i(m_k)$ is surjective for all $i \leq k$ for all $k$.

	For injectivity we note that $H^i(m_2)$ is injective for $i \leq 3$, since $\Lambda V^{\leq 2}$ has no elements of degree $3$. Assume $H^i(m_k)$ is injective for $i \leq k+1$ and let $[z] \in \ker H^i(m_{k+1})$. Now if $\deg{z} \leq k$ we get $[z] = 0$ by induction and if $\deg{z} = k+2$ we get $[z] = 0$ by construction. Finally if $\deg{z} = k+1$, then we write $z = \sum \lambda_\alpha v_\alpha + \sum \lambda'_\beta v'_\beta + w$ where $v_\alpha, v'_\beta$ are the generators as above and $w \in \Lambda V^{\leq k}$. Now $d z = 0$ and so $\sum \lambda'_\beta v'_\beta + dw = 0$, so that $\sum \lambda'_\beta [z_\beta] = 0$. Since $\{ [z_\beta] \}$ was a basis, we see that $\lambda'_\beta = 0$ for all $\beta$. Now by applying $m_k$ we get $\sum \lambda_\alpha [b_\alpha] = H(m_k)[w]$, so that $\sum \lambda_\alpha [a_\alpha] = 0$ in the cokernel, recall that $\{ [a_\alpha] \}$ formed a basis and hence $\lambda_\alpha = 0$ for all $\alpha$. Now $z = w$ and the statement follows by induction. Conclude that $H^i(m_{k+1})$ is injective for $i \leq k+2$.

	This concludes that $H(m)$ is indeed an isomorphism. So we constructed a weak equivalence $m: \Lambda V \to A$, where $\Lambda V$ is minimal by \LemmaRef{1-reduced-minimal-model}.
\end{proof}

\Remark{finited-dim-minimal-model}{
	The previous construction will construct an $r$-reduced minimal model for an $r$-connected cdga $A$.

	Moreover if $H(A)$ is finite dimensional in each degree, then so is the minimal model $\Lambda V$. This follows inductively. First notice that $V^2$ is clearly finite dimensional. Now assume that $\Lambda V^{<k}$ is finite dimensional in each degree, then both the cokernel and kernel are, so we adjoin only finitely many elements in $V^k$.
}

\section{Uniqueness}

Before we state the uniqueness theorem we need some more properties of minimal models. In fact we will prove that Sullivan algebras are cofibrant. This allows us to use some general facts about model categories.

\begin{lemma}
	Sullivan algebras are cofibrant and the inclusions induced by the filtration are cofibrations.
\end{lemma}
\begin{proof}
	Consider the following lifting problem, where $\Lambda V$ is a Sullivan algebra.
	\begin{displaymath}
		\xymatrix {
		\k \ar[r]^\unit \ar[d]^\unit & X \artfib[d]^p \\
		\Lambda V \ar[r]^g & Y
		}
	\end{displaymath}

	By the left adjointness of $\Lambda$ we only have to specify a map $\phi: V \to X$ which commutes with the differential such that $p \circ \phi = g$. We will do this by induction. Note that the induction step proves precisely that $(\Lambda V(k), d) \to (\Lambda V(k+1), d)$ is a cofibration.
	\begin{itemize}
		\item Suppose $\{v_\alpha\}$ is a basis for $V(0)$. Define $V(0) \to X$ by choosing preimages $x_\alpha$ such that $p(x_\alpha) = g(v_\alpha)$ ($p$ is surjective). Define $\phi(v_\alpha) = x_\alpha$.
		\item Suppose $\phi$ has been defined on $V(n)$. Write $V(n+1) = V(n) \oplus V'$ and let $\{v_\alpha\}$ be a basis for $V'$. Then $dv_\alpha \in \Lambda V(n)$, hence $\phi(dv_\alpha)$ is defined and
		$$ d \phi d v_\alpha = \phi d^2 v_\alpha = 0 $$
		$$ p \phi d v_\alpha = g d v_\alpha = d g v_\alpha. $$
		Now $\phi d v_\alpha$ is a cycle and $p \phi d v_\alpha$ is a boundary of $g v_\alpha$. By the following lemma there is a $x_\alpha \in X$ such that $d x_\alpha = \phi d v_\alpha$ and $p x_\alpha = g v_\alpha$. The former property proves that $\phi$ is a chain map, the latter proves the needed commutativity $p \circ \phi = g$.
	\end{itemize}
\end{proof}

\begin{lemma}
	Let $p: X \to Y$ be a trivial fibration, $x \in X$ a cycle, $p(x) \in Y$ a boundary of $y' \in Y$. Then there is a $x' \in X$ such that
	$$ dx' = x \quad\text{ and }\quad px' = y'. $$
\end{lemma}
\begin{proof}
	We have $p^\ast [x] = [px] = 0$, since $p^\ast$ is injective we have $x = d \overline{x}$ for some $\overline{x} \in X$. Now $p \overline{x} = y' + db$ for some $b \in Y$. Choose $a \in X$ with $p a = b$, then define $x' = \overline{x} - da$. Now check the requirements: $p x' = p \overline{x} - p a = y'$ and $d x' = d \overline{x} - d d a = d \overline{x} = x$.
\end{proof}

As minimal models are cofibrant \RemarkRef{cdga-weak-eq-bijection} immediately implies the following.

\Corollary{minimal-model-bijection}{
	Let $f: X \we Y$ be a weak equivalence between cdga's and $M$ a minimal algebra. Then $f$ induces an bijection:
	$$ f_\ast: [M, X] \tot{\iso} [M, Y]. $$
}

\begin{lemma}
	Let $\phi: (M, d) \we (M', d')$ be a weak equivalence between minimal algebras. Then $\phi$ is an isomorphism.
\end{lemma}
\begin{proof}
	Since both $M$ and $M'$ are minimal, they are cofibrant and so the weak equivalence is a strong homotopy equivalence (\CorollaryRef{cdga_homotopy_properties}). And so the induced map $\pi^n(\phi) : \pi^n(M) \to \pi^n(M')$ is an isomorphism (\LemmaRef{cdga-homotopic-maps-equal-pin}).

	Since $M$ (resp. $M'$) is free as a cga's, it is generated by some graded vector space $V$ (resp. $V'$). By an earlier remark the homotopy groups were easy to calculate and we conclude that $\phi$ induces an isomorphism from $V$ to $V'$:
	\[ \pi^\ast(\phi) : V \tot{\iso} V'. \]

	By induction on the degree one can prove that $\phi$ is surjective and hence it is a fibration. By the lifting property we can find a right inverse $\psi$, which is then injective and a weak equivalence. Now the above argument also applies to $\psi$ and so $\psi$ is surjective. Conclude that $\psi$ is an isomorphism and $\phi$, being its right inverse, is an isomorphism as well.
\end{proof}

\Theorem{unique-minimal-model}{
	Let $m: (M, d) \we (A, d)$ and $m': (M', d') \we (A, d)$ be two minimal models for $A$. Then there is an isomorphism $\phi (M, d) \tot{\iso} (M', d')$ such that $m' \circ \phi \eq m$.
}
\begin{proof}
	By \CorollaryRef{minimal-model-bijection} we have $[M', M] \iso [M', A]$. By going from right to left we get a map $\phi: M' \to M$ such that $m' \circ \phi \eq m$. On homology we get $H(m') \circ H(\phi) = H(m)$, proving that (2-out-of-3) $\phi$ is a weak equivalence. The previous lemma states that $\phi$ is then an isomorphism.
\end{proof}

The assignment to $X$ of its minimal model $M_X = (\Lambda V, d)$ can be extended to morphisms. Let $X$ and $Y$ be two cdga's and $f: X \to Y$ be a map. By considering their minimal models we get the following diagram.
\begin{displaymath}
	\xymatrix @C=1.5cm{
	X \ar[r]^f & Y \\
	M_X \arwe[u]^{m_X} \ar[ur]^{f m_X} & M_Y \arwe[u]^{m_Y}
	}
\end{displaymath}
Now by \LemmaRef{minimal-model-bijection} we get a bijection ${m_Y}_\ast^{-1} : [M_X, Y] \iso [M_X, M_Y]$. This gives a map $M(f) = {m_Y}_\ast^{-1} (f m_X)$ from $M_X$ to $M_Y$. Of course this does not define a functor of cdga's as it is only well defined on homotopy classes. However it is clear that it does define a functor on the homotopy category of cdga's.

\Corollary{minimal-model-equivalence}{
	The assignment $X \mapsto M_X$ defines a functor $M: \Ho(\CDGA_{\k,1}) \to \Ho(\CDGA_{\k,1})$. Moreover, since the minimal model is weakly equivalent, $M$ gives an equivalence of categories:
	$$ M: \Ho(\CDGA_{\k,1}) \iso \Ho(\text{Minimal algebras}_1) $$
}


\section{The minimal model of the sphere}
We know from singular cohomology that the cohomology ring of a $n$-sphere is $\Z[X] / (X^2)$, i.e. the cga with 1 generator $X$ in degree $n$ such that $X^2 = 0$. This allows us to construct a minimal model for $S^n$.
\Definition{minimal-model-sphere}{
	Define $A(n)$ to be the cdga defined as
	$$ A(n) = \begin{cases}
		\Lambda(e) \quad \deg{e} = n \quad de = 0 \qquad &\text{ if $n$ is odd } \\
		\Lambda(e, f) \quad \deg{e} = n, \deg{f} = 2n-1 \quad df = e^2 \qquad &\text{ if $n$ is even }
	\end{cases}. $$
}

To prove that this indeed defines minimal models, we first note that $A(n)$ indeed has the same cohomology groups. All we need to prove is that there is an actual  weak equivalence $A(n) \to A(S^n)$.

For the odd case, we can choose a representative $y \in A(S^n)$ for the generator $X$. Sending $e$ to $y$ defines a map $\phi: A(n) \to A(S^n)$. Note that since $\deg{y}$ is odd we have $y^2 = 0$ by commutativity of $A(S^n)$, so indeed $\phi$ is a map of algebras. Both $e$ and $y$ are cocycles, so $\phi$ is a chain map. Finally we see that $H(\phi)$ sends $[e]$ to $X$, hence this is an isomorphism.

For the even case, we need to choose two elements in $A(S^n)$. Again let $y \in A(S^n)$ be a representative for $X$. Now since $X^2 = 0$ there is an element $c \in A(S^n)$ such that $y^2 = d c$. Sending $e$ to $y$ and $f$ to $c$ defines a map of cdga's $\phi : A(n) \to A(S^n)$. And $H(\phi)$ sends the class $[e]$ to $X$.
