
\chapter{Minimal models}
\label{sec:minimal-models}

In this section we will discuss the so called minimal models. These are cdga's with the property that a quasi isomorphism between them is an actual isomorphism.

\begin{definition}
	A cdga $(A, d)$ is a \emph{Sullivan algebra} if
	\begin{itemize}
		\item $A = \Lambda V$ is free as a commutative graded algebra, and
		\item $V$ has a filtration
		$$ 0 = V(-1) \subset V(0) \subset V(1) \subset \cdots \subset \bigcup_{k \in \N} V(k) = V, $$
		such that $d(V(k)) \subset \Lambda V(k-1)$.
	\end{itemize}

	An cdga $(A, d)$ is a \emph{minimal (Sullivan) algebra} if in addition
	\begin{itemize}
		\item $d$ is decomposable, i.e. $\im(d) \subset \Lambda^{\geq 2}V$.
	\end{itemize}
\end{definition}

\begin{definition}
	Let $(A, d)$ be any cdga. A \emph{(minimal) Sullivan model} is a (minimal) Sullivan algebra $(M, d)$ with a weak equivalence:
	$$ (M, d) \we (A, d). $$
\end{definition}

The requirement that there exists a filtration can be replaced by a stronger statement.

\begin{lemma}
	Let $(A, d)$ be a cdga which is $1$-reduced, quasi-free and with a decomposable differential. Then $(A, d)$ is a minimal algebra.
\end{lemma}
\begin{proof}
	Let $V$ generate $A$. Take $V(n) = \bigoplus_{k=0}^n V^k$ (note that $V^0 = V^1 = 0$). Since $d$ is decomposable we see that for $v \in V^n$: $d(v) = x \cdot y$ for some $x, y \in A$. Assuming $dv$ to be non-zero we can compute the degrees:
	$$ \deg{x} + \deg{y} = \deg{xy} = \deg{dv} = \deg{v} + 1 = n + 1. $$
	As $A$ is $1$-reduced we have $\deg{x}, \deg{y} \geq 2$ and so by the above $\deg{x}, \deg{y} \leq n-1$. Conclude that $d(V(k)) \subset \Lambda(V(n-1))$.
\end{proof}


\section{Existence}

\begin{theorem}
	Let $(A, d)$ be an $0$-connected cdga, then it has a Sullivan model $(\Lambda V, d)$. Furthermore if $(A, d)$ is $r$-connected with $r \geq 1$ then $V^i = 0$ for all $i \leq r$ and in particular $(\Lambda V, d)$ is minimal.
\end{theorem}
\begin{proof}
	Start by setting $V(0) = H^{\geq 1}(A)$ and $d = 0$. This extends to a morphism $m_0 : (\Lambda V(0), 0) \to (A, d)$.
	Note that the freeness introduces products such that the map $H(m_0) : H(\Lambda V(0)) \to H(A)$ is \emph{not} an isomorphism. We will ``kill'' these defects inductively.

	Suppose $V(k)$ and $m_k$ are constructed. Consider the defect $\ker H(m_k)$ and let $\{[z_\alpha]\}_{\alpha \in A}$ be a basis for it. Define $V_{k+1} = \bigoplus_{\alpha \in A} \k \cdot v_\alpha$ with the degrees $\deg{v_\alpha} = \deg{z_\alpha}-1$.
	Now extend the differential by defining $d(v_\alpha) = z_\alpha$. This step kills the defect, but also introduces new defects which will be killed later. Notice that $z_\alpha$ is a cocycle and hence $d^2 v_\alpha = 0$, so $d$ is still a differential.
	Since $[z_\alpha]$ is in the kernel of $H(m_k)$ we see that $m_k z_\alpha = d a_\alpha$ for some $a_\alpha$. Extend $m_k$ to $m_{k+1}$ by defining $m_{k+1}(v_\alpha) = a_\alpha$. Notice that $m_{k+1} d v_\alpha = m_{k+1} z_\alpha = d a_\alpha = d m_{k+1} v_\alpha$, so $m_{k+1}$ is a cochain map.
	Now take $V(k+1) = V(k) \oplus V_{k+1}$.

	Complete the construction by taking the union: $V = \bigcup_k V(k)$. Clearly $H(m)$ is surjective, this was established in the first step. Now if $H(m)[z] = 0$, then we know $z \in \Lambda V(k)$ for some stage $k$ and hence by construction is was killed, i.e. $[z] = 0$. So we see that $m$ is a quasi isomorphism and by construction $(\Lambda V, d)$ is a Sullivan algebra.

	Now assume that $(A, d)$ is $r$-connected ($r \geq 1$), this means that $H^i(A) = 0$ for all $1 \leq i \leq r$, and so $V(0)^i = 0$ for all $i \leq r$. Now $H(m_0)$ is injective on $\Lambda^{\leq 1} V(0)$, and so the defects are in $\Lambda^{\geq 2} V(0)$ and have at least degree $2(r+1)$. This means two things in the first inductive step of the construction. First, the newly added elements have decomposable differential. Secondly, these elements are at least of degree $2(r+1) - 1$. After adding these elements, the new defects are in $\Lambda^{\geq 2} V(1)$ and have at least degree $2(2(r+1) - 1)$. We see that as the construction continues, the degrees of adjoined elements go up. Hence $V^i = 0$ for all $i \leq r$ and by the previous lemma $(\Lambda V, d)$ is minimal.
\end{proof}


\section{Uniqueness}

Before we state the uniqueness theorem we need some more properties of minimal models.

\begin{lemma}
	Sullivan algebras are cofibrant and the inclusions $(\Lambda V(k), d) \to (\Lambda V(k+1), d)$ are cofibrations.
\end{lemma}
\begin{proof}
	Consider the following lifting problem, where $\Lambda V$ is a Sullivan algebra.

	\cimage[scale=0.5]{Sullivan_Lifting}

	By the left adjointness of $\Lambda$ we only have to specify a map $\phi: V \to X$ such that $p \circ \phi = g$. We will do this by induction. Note that the induction step proves precisely that $(\Lambda V(k), d) \to (\Lambda V(k+1), d)$ is a cofibrations.
	\begin{itemize}
		\item Suppose $\{v_\alpha\}$ is a basis for $V(0)$. Define $V(0) \to X$ by choosing preimages $x_\alpha$ such that $p(x_\alpha) = g(v_\alpha)$ ($p$ is surjective). Define $\phi(v_\alpha) = x_\alpha$.
		\item Suppose $\phi$ has been defined on $V(n)$. Write $V(n+1) = V(n) \oplus V'$ and let $\{v_\alpha\}$ be a basis for $V'$. Then $dv_\alpha \in \Lambda V(n)$, hence $\phi(dv_\alpha)$ is defined and
		$$ d \phi d v_\alpha = \phi d^2 v_\alpha = 0 $$
		$$ p \phi d v_\alpha = g d v_\alpha = d g v_\alpha. $$
		Now $\phi d v_\alpha$ is a cycle and $p \phi d v_\alpha$ is a boundary of $g v_\alpha$. By the following lemma there is a $x_\alpha \in X$ such that $d x_\alpha = \phi d v_\alpha$ and $p x_\alpha = g v_\alpha$. The former property proves that $\phi$ is a chain map, the latter proves the needed commutativity $p \circ \phi = g$.
	\end{itemize}
\end{proof}

\begin{lemma}
	Let $p: X \to Y$ be a trivial fibration, $x \in X$ a cycle, $p(x) \in Y$ a boundary of $y' \in Y$. Then there is a $x' \in X$ such that
	$$ dx' = x \quad\text{ and }\quad px' = y'. $$
\end{lemma}
\begin{proof}
	We have $p^\ast [x] = [px] = 0$, since $p^\ast$ is injective we have $x = d \overline{x}$ for some $\overline{x} \in X$. Now $p \overline{x} = y' + db$ for some $b \in Y$. Choose $a \in X$ with $p a = b$, then define $x' = \overline{x} - da$. Now check the requirements: $p x' = p \overline{x} - p a = y'$ and $d x' = d \overline{x} - d d a = d \overline{x} = x$.
\end{proof}

\Lemma{minimal-model-bijection}{
	Let $f: X \we Y$ be a weak equivalence between cdga's and $M$ a minimal algebra. Then $f$ induces an bijection:
	$$ f_\ast: [M, X] \tot{\iso} [M, Y]. $$
}
\begin{proof}
	If $f$ is surjective this follows from the fact that $M$ is cofibrant and $f$ being a trivial fibration, see \CorollaryRef{cdga_homotopy_properties}.

	\todo{Put this surjectivity trick in a lemma} In general we will reduce to the surjective case. Let $C$ be any cochain complex and define $\delta C^k = C^{k-1}$. Now $(C \oplus \delta C, \delta)$ is again a cochain complex and there is a surjective map to $C$. Define $E(Y) = \Lambda(Y \oplus \delta Y, \delta)$ (we consider $Y$ as a cochain complex). We obtain:
	$$ f: X \tot{x \mapsto x \tensor 1} X \tensor E(Y) \tot{x \tensor y \mapsto f(x) \cdot y} Y. $$
	Now the first map has a left inverse. We have two trivial fibrations $E(Y) \to X$ and $E(Y) \to Y$. This induces
	$$ [M, X] \toti{\iso} [M, E(Y)] \tot{\iso} [M, Y], $$
	compatible with $f_\ast$.
\end{proof}

\begin{lemma}
	Let $\phi: (M, d) \we (M', d')$ be a weak equivalence between minimal algebras. Then $\phi$ is an isomorphism.
\end{lemma}
\begin{proof}
	Let $M$ and $M'$ be generated by $V$ and $V'$. Then $\phi$ induces a weak equivalence on the linear part $\phi_0: V \we V'$ \cite[Theorem 1.5.2]{loday}. Since the differentials are decomposable, their linear part vanishes. So we see that $\phi_0: (V, 0) \tot{\iso} (V', 0)$ is an isomorphism.
	Conclude that $\phi = \Lambda \phi_0$ is an isomorphism.
\end{proof}

\Theorem{unique-minimal-model}{
	Let $m: (M, d) \we (A, d)$ and $m': (M', d') \we (A, d)$ be two minimal models. Then there is an isomorphism $\phi (M, d) \tot{\iso} (M', d')$ such that $m' \circ \phi \eq m$.
}
\begin{proof}
	By the previous lemmas we have $[M', M] \iso [M', A]$. By going from right to left we get a map $\phi: M' \to M$ such that $m' \circ \phi \eq m$. On homology we get $H(m') \circ H(\phi) = H(m)$, proving that (2-out-of-3) $\phi$ is a weak equivalence. The previous lemma states that $\phi$ is then an isomorphism.
\end{proof}

The assignment of $X$ to its minimal model $M_X = (\Lambda V, d)$ can be extended to morphisms. Let $X$ and $Y$ be two cdga's and $f: X \to Y$ be a map. By considering their minimal models we get the following diagram.
\begin{displaymath}
	\xymatrix{
	X \ar[r]^f & Y \\
	M_X \arwe[u]^{m_X} \ar[ur]^{f m_X} & M_Y \arwe[u]^{m_Y}
	}
\end{displaymath}
Now by \LemmaRef{minimal-model-bijection} we get a bijection ${m_Y}_\ast^{-1} : [M_X, Y] \iso [M_X, M_Y]$. This gives a map $M(f) = {m_Y}_\ast^{-1} (f m_X)$ from $M_X$ to $M_Y$. Of course this does not define a functor of cdga's as it is only well defined on homotopy classes. However it is clear that it does define a functor on the homotopy category of cdga's.

\Corollary{}{
	The assignment $X \mapsto M_X$ defines a functor $M: \Ho(\CDGA^1_\Q) \to \Ho(\CDGA^1_\Q)$. Moreover, since the minimal model is weakly equivalent, $M$ gives an equivalence of categories:
	$$ M: \Ho(\CDGA^1_\Q) \iso \Ho(\text{Minimal algebras}^1), $$
	where weakly equivalent cdga's are sent to \emph{isomorphic} minimal models.
}

\section{The minimal model of the sphere}
We know from singular cohomology that the cohomology ring of a $n$-sphere is $\Z[X] / (X^2)$. This allows us to construct a minimal model for $S^n$.
\Definition{minimal-model-sphere}{
	Define $A(n)$ to be the cdga defined as
	$$ A(n) = \begin{cases}
		\Lambda(e) \quad \deg{e} = n \quad de = 0 \qquad &\text{ if $n$ is odd } \\
		\Lambda(e, f) \quad \deg{e} = n, \deg{f} = 2n-1 \quad df = e^2 \qquad &\text{ if $n$ is even }
	\end{cases}. $$
}

\todo{prove $HA(n) \tot{\iso} HA(S^n)$}
