
\chapter{Differential Graded Algebra}
\label{sec:algebra}

In this section $\k$ will be any commutative ring. We will recap some of the basic definitions of commutative algebra in a graded setting. By \emph{linear}, \emph{module}, \emph{tensor product}, etc\dots we always mean $\k$-linear, $\k$-module, tensor product over $\k$, etc\dots.

\section{Graded algebra}

\Definition{graded-module}{
	A module $M$ is said to be \Def{graded} if it is equipped with a decomposition
	\[ M = \bigoplus_{n\in\Z} M_n. \]
	An element $x \in M_n$ is called a \Def{homogeneous element} and said to be of \Def{degree} $\deg{x} = n$.
}

If $M$ is just any module, it always has the trivial grading given by $M_0 = M$ and $M_i = 0$ for $i \neq 0$, i.e. $M$ is \Def{concentrated in degree 0}. In particular $\k$ itself is a graded module concentrated in degree $0$.

\begin{definition}
	A linear map $f: M \to N$ between graded modules is \emph{graded of degree $p$} if it respects the grading and raises the degree by $p$, i.e.
	$$ \restr{f}{M_n} : M_n \to N_{n+p}. $$
\end{definition}

\begin{definition}
	The graded maps $f: M \to N$ between graded modules can be arranged in a graded module by defining:
	$$ \Hom_{gr}(M, N)_n = \{ f: M \to N \I f \text{ is graded of degree } n \}. $$
\end{definition}

Note that not all linear maps can be decomposed into a sum of graded maps, so that $\Hom_{gr}(M, N) \subset \Hom(M, N)$ may be proper for some $M$ and $N$.

Recall that the tensor product of modules distributes over direct sums. This defines a natural grading on the ordinary tensor product.

\begin{definition}
	The graded tensor product is defined as:
	$$ (M \tensor N)_n = \bigoplus_{i + j = n} M_i \tensor N_j. $$
	The tensor product extends to graded maps. Let $f: A \to B$ and $g:X \to Y$ be two graded maps, then their tensor product $f \tensor g: A \tensor B \to X \tensor Y$ is defined as:
	$$ (f \tensor g)(a \tensor x) = (-1)^{\deg{a}\deg{g}} \cdot f(a) \tensor g(x). $$
\end{definition}
\todo{graded tor}

The sign is due to \emph{Koszul's sign convention}: whenever two elements next to each other are swapped (in this case $g$ and $a$) a minus sign appears if both elements are of odd degree. More formally we can define a swap map
$$ \tau : A \tensor B \to B \tensor A : a \tensor b \mapsto (-1)^{\deg{a}\deg{b}} b \tensor a. $$

The graded modules together with graded maps of degree $0$ form the category $\grMod{\k}$ of graded modules. From now on we will simply refer to maps instead of graded maps. Together with the tensor product and the ground ring, $(\grMod{\k}, \tensor, \k)$ is a symmetric monoidal category (with the symmetry given by $\tau$). This now dictates the definition of a graded algebra.

\begin{definition}
	A \emph{graded algebra} consists of a graded module $A$ together with two maps of degree $0$:
	$$ \mu: A \tensor A \to A \quad\text{ and }\quad \eta: k \to A $$
	such that $\mu$ is associative and $\eta$ is a unit for $\mu$.

	A map between two graded algebra will be called a \emph{graded algebra map} if the map is compatible with the multiplication and unit. Such a map is necessarily of degree $0$.
\end{definition}

Again these objects and maps form a category, denoted as $\grAlg{\k}$. We will denote multiplication by a dot or juxtaposition, instead of explicitly mentioning $\mu$.

\begin{definition}
	A graded algebra $A$ is \emph{commutative} if for all $x, y \in A$
	$$ x y = (-1)^{\deg{x}\deg{y}} y x. $$
\end{definition}


\section{Differential graded algebra}

\begin{definition}
	A \emph{differential graded module} $(M, d)$ is a graded module $M$ together with a map $d: M \to M$ of degree $-1$, called a \emph{differential}, such that $dd = 0$. A map $f: M \to N$ is a \emph{chain map} if it is compatible with the differential, i.e. $d_N f = f d_M$.
\end{definition}

A differential graded module $(M, d)$ with $M_i = 0$ for all $i < 0$ is a \emph{chain complex}. A differential graded module $(M, d)$ with $M_i = 0$ for all $i > 0$ is a \emph{cochain complex}. It will be convenient to define $M^i = M_{-i}$ in the latter case, so that $M = \bigoplus_{n \in \N} M^i$ and $d$ is a map of \emph{upper degree} $+1$.

\begin{definition}
	Let $(M, d_M)$ and $(N, d_N)$ be two differential graded modules, their tensor product $M \tensor N$ is a differential graded module with the differential given by:
	$$ d_{M \tensor N} = d_M \tensor \id_N + \id_M \tensor d_N. $$
\end{definition}

Finally we come to the definition of a differential graded algebra. This will be a graded algebra with a differential. Of course we want this to be compatible with the algebra structure, or stated differently: we want $\mu$ and $\eta$ to be chain maps.

\begin{definition}
	A \emph{differential graded algebra (dga)} is a graded algebra $A$ together with an differential $d$ such that in addition the \emph{Leibniz rule} holds:
	$$ d(x y) = d(x) y + (-1)^{\deg{x}} x d(y) \quad\text{ for all } x, y \in A. $$
\end{definition}

In general, a map which satisfies the above Leibniz rule is called a \Def{derivation}. It is not hard to see that the definition of a dga precisely defines the monoidal objects in the category of differential graded modules.

In this thesis we will restrict our atention to dga's $M$ with $M^i = 0 $ for all $i < 0$, i.e. non-negatively (cohomologically) graded dga's. We denote the category of these dga's by $\DGA_\k$, the category of commutative dga's (cdga's) will be denoted by $\CDGA_\k$. If no confusion can arise, the ground ring $\k$ will be suppressed in this notation. These objects are also refered to as \emph{(co)chain algebras}.

\Definition{augmented-cdga}{
	An \Def{augmented dga} is a dga $A$ with an map $\counit : A \to \k$. Note that this necessarily means that $\counit \unit = \id$.
}

The above notion is dual to the notion of a pointed objects.


\Remark{orthogonal-definition}{
	Note that all the above definitions (i.e. the definitions of graded objects, algebras, differentials, augmentations) are orthogonal, meaning that any combination makes sense. However, keep in mind that we require the structures to be compatible. For example, an algebra with differential should satisfy the Leibniz rule (i.e. the differential should be a map of algebras).
}


\section{Homology}

Whenever we have a differential module we have $d \circ d = 0$, or put in other words: the image of $d$ is a submodule of the kernel of $d$. The quotient of the two graded modules will be of interest. Note that the following definition depends on the differential $d$, however it is often left out from the notation.

\Definition{homology}{
	Given a differential module $(M, d)$ we define the \Def{homology} of $M$ as:
	$$ H(M) = \ker(d) / \im(d). $$
}

If the module has more structure as discussed above, homology will preserve this.
\Remark{homology-preserves-structure}{
	Let $M$ be a differential module. Then homology preserves the following.
	\begin{itemize}
		\item If $M$ is graded, so is $H(M)$, where the grading is given by
		\[ H(M)_i = \ker(\restr{d}{M_i}) / d(M_{i+1}) \]
		\item If $M$ has an algebra structure, then so does $H(M)$, given by
		\[ [z_1] \cdot [z_2] = [z_1 \cdot z_2] \]
		\item If $M$ is a commutative algebra, so is $H(M)$.
		\item If $M$ is augmented, so is $H(M)$.
	\end{itemize}
}
Of course the converses need not be true. For example the singular cochain complex associated to a space is a graded differential algebra which is \emph{not} commutative. However, by taking homology one gets a commutative algebra.

Note that taking homology of a differential graded module (or algebra) is functorial. Whenever a map $f: M \to N$ of differential graded modules (or algebras) induces an isomorphism on homology, we say that $f$ is a \emph{quasi isomorphism}.

\begin{definition}
	Let $M$ be a graded module. We say that $M$ is $n$-reduced if $M_i = 0$ for all $i \leq n$. Similarly we say that a graded augmented algebra $A$ is $n$-reduced if $A_i = 0$ for all $1 \leq i \leq n$ and $\eta: \k \tot{\iso} A_0$.

	Let $(M, d)$ be a chain complex (or algebra). We say that $M$ is $n$-connected if $H(M)$ is $n$-reduced as graded module (resp. augmented algebra). Similarly for cochain complexes (or algebras).
\end{definition}


\section{Classical results}

We will give some classical known results of algebraic topology or homological algebra. Proofs of these theorems can be found in many places such as \cite{rotman, weibel}.

\begin{theorem}
	(Universal coefficient theorem) Let $C$ be a chain complex and $A$ an abelian group, then there are natural short exact sequences for each $n$:
	$$ 0 \to H_n(C) \tensor A \to H_n(C \tensor A) \to Tor(H_{n-1}(C), A) \to 0 $$
	$$ 0 \to Ext(H_{n-1}(C), A) \to H^n(\Hom(C, A)) \to \Hom(H_n(C), A) \to 0 $$
\end{theorem}

The first statement generalizes to a theorem where $A$ is a chain complex itself. When choosing to work over a field the torsion will vanish and the exactness will induce an isomorphism. This is (one formulation of) the Künneth theorem.

\begin{theorem}
	(Künneth) Assume that $\k$ is a field and let $C$ and $D$ be (co)chain complexes, then there is a natural isomorphism (a linear graded map of degree $0$):
	$$ H(C) \tensor H(D) \tot{\iso} H(C \tensor D), $$
	where we understand both tensors as graded. If $C$ and $D$ are algebras, this isomorphism is an isomorphism of algebras.
\end{theorem}
