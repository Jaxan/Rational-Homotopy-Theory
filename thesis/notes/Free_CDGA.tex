
\subsection{The free cdga}

Just as in ordinary linear algebra we can form an algebra from any graded module. Furthermore we will see that a differential induces a derivation.

\begin{definition}
	The \emph{tensor algebra} of a graded module $M$ is defined as
	$$ T(M) = \bigoplus_{n\in\N} M^{\tensor n}, $$
	where $M^{\tensor 0} = \k$. An element $m = m_1 \tensor \ldots \tensor m_n$ has a \emph{word length} of $n$ and its degree is $\deg{m} = \sum_{i=i}^n \deg{m_i}$. The multiplication is given by the tensor product (note that the bilinearity follows immediately).
\end{definition}

Note that this construction is functorial and it is free in the following sense.

\begin{lemma}
	Let $M$ be a graded module and $A$ a graded algebra.
	\begin{itemize}
		\item A graded map $f: M \to A$ of degree $0$ extends uniquely to an algebra map $\overline{f} : TM \to A$.
		\item A differential $d: M \to M$ extends uniquely to a derivation $d: TM \to TM$.
	\end{itemize}
\end{lemma}

\begin{corollary}
	Let $U$ be the forgetful functor from graded algebras to graded modules, then $T$ and $U$ form an adjoint pair:
	$$ T: \grMod{\k} \leftadj \grAlg{\k} $$
	Moreover it extends and restricts to
	$$ T: \dgMod{\k} \leftadj \dgAlg{\k} $$
	$$ T: \CoCh{\k} \leftadj \DGA{\k} $$
\end{corollary}

As with the symmetric algebra and exterior algebra of a vector space, we can turn this graded tensor algebra in a commutative graded algebra.

\begin{definition}
	Let $A$ be a graded algebra and define
	$$ I = < ab - (-1)^{\deg{a}\deg{b}}ba \I a,b \in A >. $$
	Then $A / I$ is a commutative graded algebra.

	For a graded module $M$ we define the \emph{free commutative graded algebra} as
	$$ \Lambda(M) = TM / I $$
\end{definition}

Again this extends to differential graded modules (i.e. the ideal is preserved by the derivative) and restricts to cochain complexes.

\begin{lemma}
	We have the following adjunctions:
	$$ \Lambda: \grMod{\k} \leftadj \grAlg{\k}^{comm} $$
	$$ \Lambda: \dgMod{\k} \leftadj \dgAlg{\k}^{comm} $$
	$$ \Lambda: \CoCh{\k} \leftadj \CDGA_\k $$
\end{lemma}

We can now easily construct cdga's by specifying generators and their differentials.
