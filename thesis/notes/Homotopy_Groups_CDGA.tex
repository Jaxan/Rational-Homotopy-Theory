
As the eventual goal is to compare the homotopy theory of spaces with the homotopy theory of cdga's, it is natural to investigate an analogue of homotopy groups in the category of cdga's. In topology we can only define homotopy groups on pointed spaces, dually we will consider augmented cdga's in this section.

\Definition{cdga-homotopy-groups}{
	The \Def{homotopy groups of an augmented cdga} $A$ are
	$$ \pi^i(A) = H^i(QA). $$
}

This construction is functorial (since both $Q$ and $H$ are) and, as the following lemma shows, homotopy invariant.

\Lemma{cdga-homotopic-maps-equal-pin}{
	Let $f: A \to X$ and $g: A \to X$ be a maps of augmented cdga's. If $f$ and $g$ are homotopic, then the induced maps are equal:
	$$ f_\ast = g_\ast : \pi_\ast(A) \to \pi_\ast(X). $$
}
\Proof{
	Let $h: A \to \Lambda(t, dt) \tensor X$ be a homotopy. We will, just as in \LemmaRef{cdga-homotopy-homology}, prove that the maps $HQ(d_0)$ and $HQ(d_1)$ are equal, then it follows that $HQ(f) = HQ(d_1 h) = HQ(d_0 h) = HQ(g)$.

	Using \LemmaRef{Q-preserves-copord} we can identify the induced maps $Q(d_i) : Q(\Lambda(t, dt) \tensor X) \to Q(X)$ with maps
	\[ Q(d_i) : Q(\Lambda(t, dt)) \oplus Q(A) \to Q(A). \]
	Now $Q(\Lambda(t, dt)) = D(0)$ and hence it is acyclic, so when passing to homology, this term vanishes. In other words both maps ${d_i}_\ast : H(D(0)) \oplus H(Q(A)) \to H(Q(A))$ are the identity maps on $H(Q(A))$.
}

Consider the augmented cdga $V(n) = S(n) \oplus \k$, with trivial multiplication and where the term $\k$ is used for the unit and augmentation. This augmented cdga can be thought of as a specific model of the sphere. In particular the homotopy groups can be expressed as follows.

\Lemma{cdga-dual-homotopy-groups}{
	There is a natural bijection for any augmented cdga $A$
	$$ [A, V(n)] \tot{\iso} \Hom_\k(\pi^n(A), \k). $$
}
\Proof{
	Note that $Q(V(n))$ in degree $n$ is just $\k$ and $0$ in the other degrees, so its homotopy groups consists of a single $\k$ in degree $n$. This establishes the map:
	$$ \Phi: \Hom_\CDGA(A, V(n)) \to \Hom_\k(\pi^n(A), \k). $$

	Now by \LemmaRef{cdga-homotopic-maps-equal-pin} we get a map from the set of homotopy classes $[A, V(n)]$ instead of just maps. \todo{injective, surjective}
}

From now on the dual of a vector space will be denoted as $V^\ast = \Hom_\k(V, \k)$. So the above lemma states that there is a bijection $[A, V(n)] \iso \pi^n(A)^\ast$.

In topology we know that a fibration induces a long exact sequence of homotopy groups. In the case of cdga's a similar long exact sequence for a cofibration will exist.

\Lemma{long-exact-cdga-homotopy}{
	Given a pushout square of augmented cdga's
	\[ \xymatrix{
		A \ar[d]^-f \arcof[r]^-g \xypo & C \ar[d]^-i \\
		B \ar[r]^-j & P
	} \]
	where $g$ is a cofibration. There is a natural long exact sequence
	\[ \pi^o(V) \tot{(f_\ast, g_\ast)} \pi^0(B) \oplus \pi^0(C) \tot{j_\ast - i_\ast} \pi^0(P) \tot{\del} \pi^1(A) \to \cdots \]
}
\Proof{
	First note that $j$ is also a cofibration. By \LemmaRef{Q-preserves-cofibs} the maps $Qg$ and $Qj$ are injective in positive degrees. By applying $Q$ we get two exact sequence (in positive degrees) as shown in the following diagram. By the fact that $Q$ preserves pushouts (\LemmaRef{Q-preserves-pushouts}) the cokernels coincide.
	\[ \xymatrix {
		0 \ar[r] & Q(A) \ar[r] \ar[d] \xypo & Q(C) \ar[r] \ar[d] & \coker(f_\ast) \ar[r] \ar[d] & 0 \\
		0 \ar[r] & Q(B) \ar[r] & Q(P) \ar[r] & \coker(f_\ast) \ar[r] & 0
	} \]
	Now the well known Mayer-Vietoris exact sequence can be constructed. This proves the statement.
}

\Corollary{long-exact-cdga-homotopy}{
	When we take $B = \k$ in the above situation, we get a long exact sequence
	\[ \pi^0(A) \tot{g_\ast} \pi^0(C) \to \pi^0(\coker(g)) \to \pi^1(A) \to \cdots \]
}
