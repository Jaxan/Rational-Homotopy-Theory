
\section{Model categories}
\label{sec:model_cats}

\newcommand{\W}{\mathfrak{W}}
\newcommand{\Fib}{\mathfrak{Fib}}
\newcommand{\Cof}{\mathfrak{Cof}}

\begin{definition}
	A \emph{(closed) model category} is a category $\cat{C}$ together with three subcategories:
	\begin{itemize}
		\item the class of weak equivalences $\W$,
		\item the class of fibrations $\Fib$ and
		\item the class of cofibrations $\Cof$,
	\end{itemize}
	such that the following five axioms hold:
	\begin{itemize}
		\item[MC1] All finite limits and colimits exist in $\cat{C}$.
		\item[MC2] If $f$, $g$ and $fg$ are maps such that two of them are weak equivalences, then so it the third. This is called the \emph{2-out-of-3} property.
		\item[MC3] All three classes of maps are closed under retracts\todo{Either draw the diagram or define a retract earlier}.
		\item[MC4] In any commuting square as follows where $i \in \Cof$ and $p \in \Fib$,
		\begin{center}
		\begin{tikzpicture}
		\matrix (m) [matrix of math nodes]{
			A & X \\
			B & Y \\
		};

		\path[->] (m-1-1) edge (m-1-2);
		\path[->] (m-2-1) edge (m-2-2);
		\path[->] (m-1-1) edge node[auto] {$i$} (m-2-1);
		\path[->] (m-1-2) edge node[auto] {$p$} (m-2-2);

		\end{tikzpicture}
		\end{center}

		 there exist a lift $h: B \to Y$ if either 
		\begin{itemize}
			\item[a)] $i \in \W$ or
			\item[b)] $p \in \W$.
		\end{itemize}
		\item[MC5] Any map $f : A \to B$ can be factored in two ways:
		\begin{itemize}
			\item[a)] as $f = pi$, where $i \in \Cof \cap \W$ and $p \in \Fib$ and
			\item[b)] as $f = pi$, where $i \in \Cof$ and $p \in \Fib \cap \W$.
		\end{itemize}
	\end{itemize}
\end{definition}

\begin{notation} For brevity
	\begin{itemize}
		\item we write $f: A \fib B$ when $f$ is a fibration,
		\item we write $f: A \cof B$ when $f$ is a cofibration and
		\item we write $f: A \we B$ when $f$ is a weak equivalence.
	\end{itemize}
\end{notation}

\begin{definition}
	An object $A$ in a model category $\cat{C}$ will be called \emph{fibrant} if $A \to \cat{1}$ is a fibration and \emph{cofibrant} if $\cat{0} \to A$ is a cofibration.
\end{definition}

Note that axiom [MC5a] allows us to replace any object $X$ with a weakly equivalent fibrant object $X^{fib}$ and by [MC5b] by a weakly equivalent cofibrant object $X^{cof}$, as seen in the following diagram:

\begin{center}
\begin{tikzpicture}
\matrix (m) [matrix of math nodes]{
	\cat{0} &         & X \\
	        & X^{cof} &   \\
};

\path[->] (m-1-1) edge (m-1-3);
\path[right hook->] (m-1-1) edge (m-2-2);
\path[->>] (m-2-2) edge node[auto] {$ \simeq $} (m-1-3);

\end{tikzpicture}\quad
\begin{tikzpicture}
\matrix (m) [matrix of math nodes]{
	X &         & \cat{1} \\
	  & X^{fib} &         \\
};

\path[->] (m-1-1) edge (m-1-3);
\path[right hook->] (m-1-1) edge node[auto] {$ \simeq $} (m-2-2);
\path[->>] (m-2-2) edge (m-1-3);

\end{tikzpicture}
\end{center}

\TODO{Maybe some basic propositions (refer to Dwyer \& Spalinski):
\titem Over/under category (or simply pointed objects)
\titem If a map has LLP/RLP wrt fib/cof, it is a cof/fib
\titem Fibs are preserved under pullbacks/limits
\titem Cofibrantly generated mod. cats.
\titem Small object argument
}

\todo{Define homotopy category}

\subsection{Quillen pairs}
In order to relate model categories and their associated homotopy categories we need a notion of maps between them. We want the maps such that they induce maps on the homotopy categories.
\todo{Definition etc}
