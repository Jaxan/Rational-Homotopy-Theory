\documentclass[a4paper, 12pt]{amsart}


% lesser margins
% \usepackage{geometry}
% \geometry{a4paper}
% \geometry{twoside=false}

% no indent, but vertical spacing
% \usepackage[parfill]{parskip}
% \setlength{\marginparwidth}{2cm}


% normally included with amsart
% \usepackage{amsmath, amsthm}

% font with unicode support, does not work with classicthesis
% \usepackage{fontspec}

% clickable tocs
\usepackage{hyperref}

% floating figures
\usepackage{float}

% for multiple cites
\usepackage{cite}

% fancy diagrams
% \usepackage{tikz}
% \usetikzlibrary{matrix, arrows, decorations}
% \tikzset{node distance=2.5em, row sep=2.2em, column sep=2.7em, auto}

% simple diagrams
\usepackage[all,cmtip]{xy}

\usepackage{graphicx}
\graphicspath{ {./images/} }
\usepackage{caption}
\usepackage{subcaption}

% for the fib arrow
\usepackage{amssymb}

% Some basic objects
\newcommand{\N}{\mathbb{N}}				% natural numbers
\newcommand{\Np}{{\mathbb{N}^{>0}}}		% positive numbers
\newcommand{\Z}{\mathbb{Z}}				% integers
\newcommand{\R}{\mathbb{R}}				% reals
\DeclareRobustCommand{\Q}{\mathbb{Q}}				% rationals
\renewcommand{\k}{\mathrm{I\!k}}		% default ground ring

% Basic category stuff
\newcommand{\cat}[1]{\mathbf{#1}}		% the category of ...
\newcommand{\opCat}[1]{{#1}^{\text{op}}}% opposite category
\newcommand{\Hom}{\mathbf{Hom}}
\newcommand{\id}{\mathbf{id}}
\newcommand{\Ho}{\cat{Ho}}

% Categories
\newcommand{\Set}{\cat{Set}}			% sets
\newcommand{\Top}{\cat{Top}}			% topological spaces
\newcommand{\Grp}{\cat{Grp}}			% groups
\newcommand{\Ab}{\cat{Ab}}				% abelian groups
\newcommand{\DELTA}{\boldsymbol{\Delta}}% the simplicial cat
\newcommand{\simplicial}[1]{\cat{s{#1}}}% simplicial objects
\newcommand{\sSet}{\simplicial{\Set}}	% simplicial sets
\newcommand{\Mod}[1]{\cat{{#1}Mod}}		% modules over a ring
\newcommand{\Alg}[1]{\cat{{#1}Alg}}		% algebras over a ring
\newcommand{\grMod}[1]{\cat{gr\mbox{-}{#1}Mod}}	% graded modules over a ring
\newcommand{\grAlg}[1]{\cat{gr\mbox{-}{#1}Alg}}	% graded algebras over a ring
\newcommand{\dgMod}[1]{\cat{dg\mbox{-}{#1}Mod}}	% differential graded modules over a ring
\newcommand{\dgAlg}[1]{\cat{dg\mbox{-}{#1}Alg}}	% differential graded algebras over a ring
\newcommand{\Ch}[1]{\cat{Ch_{n\geq0}({#1})}}	% chain complexes
\newcommand{\CoCh}[1]{\cat{Ch^{n\geq0}({#1})}}	% cochain complexes
\DeclareRobustCommand{\DGA}{\cat{DGA}}			% cochain algebras
\DeclareRobustCommand{\CDGA}{\cat{CDGA}}		% commutative cochain algebras
\DeclareRobustCommand{\AugCDGA}{\cat{CDGA^\ast}}% augmentedcommutative cochain algebras

\newcommand{\cof}{\hookrightarrow}		% cofibration
\newcommand{\fib}{\twoheadrightarrow}	% fibration
\newcommand{\we}{\tot{\simeq}}			% weak equivalence

% for use in xy diagrams
\newcommand{\arcof}{\ar@{^{(}->}}
\newcommand{\artcof}{\ar@{^{(}->}|\simeq}
\newcommand{\arfib}{\ar@{->>}}
\newcommand{\artfib}{\ar@{->>}|\simeq}
\newcommand{\arwe}{\ar|-\simeq}
\newcommand{\ariso}{\ar|-\iso}

% adjunction symbol for xymatrices
\newcommand{\xyadj}{\raisebox{0.2\height}{\scalebox{0.5}{$\perp$}}}

% pushout and pullback for xymatrices (makes empty arrow with text)
\newcommand{\xypo}{\ar@{}[dr]|(.75){\scalebox{1.2}{$\ulcorner$}}}
\newcommand{\xypb}{\ar@{}[dr]|(.25){\scalebox{1.2}{$\lrcorner$}}}

%\newcommand{\leftadj}{\ooalign{\hss\rightleftarrows\hss\cr\bot}}
\newcommand{\leftadj}{\rightleftarrows}

% Notation and operators
\newcommand{\I}{\,\mid\,}				% seperator in set notation
\newcommand{\del}{\partial}				% boundary
\newcommand{\iso}{\cong}				% isomorphic
\newcommand{\eq}{\sim}					% homotopic
\newcommand{\ison}[1]{\stackrel{(#1)}{\iso}} % isos to refer to
\newcommand{\refison}[1]{{\small $(#1)$}}    % ref
\newcommand{\tot}[1]{\xrightarrow{\,\,{#1}\,\,}}	% arrow with name
\newcommand{\toti}[1]{\xleftarrow{\,\,{#1}\,\,}}	% left arrow with name
\newcommand{\mapstot}[1]{\xmapsto{\,\,{#1}\,\,}}	% mapsto with name
\newcommand{\unit}{\eta}
\newcommand{\counit}{\epsilon}
\DeclareMathOperator*{\im}{im}
\DeclareMathOperator*{\coker}{coker}
\DeclareMathOperator*{\colim}{colim}
\DeclareMathOperator*{\Tor}{Tor}
\DeclareMathOperator*{\Ext}{Ext}
\DeclareMathOperator*{\tensor}{\otimes}
\DeclareMathOperator*{\bigtensor}{\bigotimes}
\renewcommand{\deg}[1]{{|{#1}|}}
\newcommand{\Char}[1]{char({#1})}
\newcommand{\RH}{\widetilde{H}}			% reduced homology
\DeclareRobustCommand{\C}{\mathcal{C}}			% Serre mod C class
\newcommand{\Apl}[0]{{A_{PL}}}			% Apl simplicial set of polynomials

\newcommand{\titleCDGA}{\texorpdfstring{$\CDGA$}{CDGA}}

% restriction of a function
\newcommand\restr[2]{{% we make the whole thing an ordinary symbol
  \left.\kern-\nulldelimiterspace % automatically resize the bar with \right
  #1 % the function
  \vphantom{\big|} % pretend it's a little taller at normal size
  \right|_{#2} % this is the delimiter
  }}


\theoremstyle{plain}
\newtheorem{theorem}{Theorem}[section]
\newtheorem{proposition}[theorem]{Proposition}
\newtheorem{lemma}[theorem]{Lemma}
\newtheorem{corollary}[theorem]{Corollary}
\newtheorem{claim}[theorem]{Claim}
\newtheorem{remark}[theorem]{Remark}

\theoremstyle{definition}
\newtheorem{definition}[theorem]{Definition}
\newtheorem{notation}[theorem]{Notation}
\newtheorem{example}[theorem]{Example}

\newcommand{\EnvTemp}[4]{
	\begin{#1}\label{#2:#3}
		{#4}
	\end{#1}
}

\newcommand{\RefTemp}[3]{{#1}~\ref{#2:#3}}

\newcommand{\Theorem}{\EnvTemp{theorem}{thm}}
\newcommand{\Proposition}{\EnvTemp{proposition}{prop}}
\newcommand{\Lemma}{\EnvTemp{lemma}{lem}}
\newcommand{\Corollary}{\EnvTemp{corollary}{cor}}
\newcommand{\Claim}{\EnvTemp{claim}{clm}}
\newcommand{\Remark}{\EnvTemp{remark}{rmk}}
\newcommand{\Proof}[1]{\begin{proof}{#1}\end{proof}}

\newcommand{\Def}{\emph}
\newcommand{\Definition}{\EnvTemp{definition}{def}}
\newcommand{\Notation}{\EnvTemp{notation}{not}}
\newcommand{\Example}{\EnvTemp{example}{eg}}

\newcommand{\TheoremRef}{\RefTemp{Theorem}{thm}}
\newcommand{\LemmaRef}{\RefTemp{Lemma}{lem}}
\newcommand{\CorollaryRef}{\RefTemp{Corollary}{cor}}
\newcommand{\RemarkRef}{\RefTemp{Remark}{rmk}}

\newcommand{\DefinitionRef}{\RefTemp{Definition}{def}}

\newcommand{\Chapter}[2]{\chapter{#1}\label{chp:#2}}
\newcommand{\ChapterRef}{\RefTemp{Chapter}{chp}}

% headings for a table
\newcommand*{\thead}[1]{\multicolumn{1}{c}{\bfseries #1}}

% simple way to center an image
\newcommand{\cimage}[2][]{
	\begin{center}
	\includegraphics[#1]{#2}
	\end{center}
}

% simple way to center a diagram
\newcommand{\cdiagrambase}[1]{
	\begin{displaymath}
	\input{#1}
	\end{displaymath}
}
\newcommand{\cdiagram}[1]{
	\cdiagrambase{diagrams/#1}
}


\title{Rational Homotopy Theory}
\author{Joshua Moerman}

\begin{document}

\maketitle
{\bf \today}

\section*{Contents}
\tableofcontents

\section*{Preliminaries}
We assume the reader is familiar with category theory, basics from algebraic topology and the basics of simplicial sets. Some knowledge about differential graded algebra (or homological algebra) and model categories is assumed, but the reader may review this in the appendices.

Some notation:
\begin{itemize}
	\item $\k$ will denote an arbitrary commutative ring (or field, if indicated at the start of a section).
	\item $\cat{C}$ will denote an arbitrary category.
	\item $\cat{0}$ (resp. $\cat{1}$) will denote the initial (resp. final) objects in a category $\cat{C}$.
	\item $\Hom_\cat{C}(A, B)$ will denote the set of maps from $A$ to $B$ in the category $\cat{C}$. We may leave out the subscript $\cat{C}$.
\end{itemize}

Some categories:
\begin{itemize}
	\item $\Top$: category of topological spaces and continuos maps.
	\item $\Ab$: category of abelian groups and group homomorphisms.
	\item $\DELTA$: category of simplices (i.e. finite, non-empty ordinals) and order preserving maps.
	\item $\sSet$: category of simplicial sets and simplicial maps (more generally we have the category of simplicial objects, $\cat{sC}$, for any category $\cat{C}$).
	\item $\Ch{\k}, \CoCh{\k}$: category of non-negatively graded chain (resp. cochain) complexes and chain maps.
	\item $\DGA_\k$: category of non-negatively differential graded algebras over $\k$ (these are cochain complexes with a multiplication) and graded algebra  maps. As a shorthand we will refer to such an object as \emph{dga}.
	\item $\CDGA_\k$: the full subcategory of $\DGA_\k$ of commutative dga's (cdga's).
\end{itemize}

\newcommand{\myinput}[1]{\include{#1}}

\addtocontents{toc}{\protect\setcounter{tocdepth}{2}}
\myinput{notes/Basics}
\myinput{notes/Serre}
\myinput{chapters/Homotopy_Theory_CDGA}
\myinput{chapters/Polynomial_Forms}
\myinput{notes/Minimal_Models}
\myinput{notes/A_K_Quillen_Pair}

\addtocontents{toc}{\protect\setcounter{tocdepth}{1}}
\begin{appendices}
	
\chapter{Differential Graded Algebra}
\label{sec:algebra}

In this section $\k$ will be any commutative ring. We will recap some of the basic definitions of commutative algebra in a graded setting. By \emph{linear}, \emph{module}, \emph{tensor product}, etc\dots we always mean $\k$-linear, $\k$-module, tensor product over $\k$, etc\dots.

\section{Graded algebra}

\Definition{graded-module}{
	A module $M$ is said to be \Def{graded} if it is equipped with a decomposition
	\[ M = \bigoplus_{n\in\Z} M_n. \]
	An element $x \in M_n$ is called a \Def{homogeneous element} and said to be of \Def{degree} $\deg{x} = n$.
}

If $M$ is just any module, it always has the trivial grading given by $M_0 = M$ and $M_i = 0$ for $i \neq 0$, i.e. $M$ is \Def{concentrated in degree 0}. In particular $\k$ itself is a graded module concentrated in degree $0$.

\begin{definition}
	A linear map $f: M \to N$ between graded modules is \emph{graded of degree $p$} if it respects the grading and raises the degree by $p$, i.e.
	$$ \restr{f}{M_n} : M_n \to N_{n+p}. $$
\end{definition}

\begin{definition}
	The graded maps $f: M \to N$ between graded modules can be arranged in a graded module by defining:
	$$ \Hom_{gr}(M, N)_n = \{ f: M \to N \I f \text{ is graded of degree } n \}. $$
\end{definition}

Note that not all linear maps can be decomposed into a sum of graded maps, so that $\Hom_{gr}(M, N) \subset \Hom(M, N)$ may be proper for some $M$ and $N$.

Recall that the tensor product of modules distributes over direct sums. This defines a natural grading on the ordinary tensor product.

\begin{definition}
	The graded tensor product is defined as:
	$$ (M \tensor N)_n = \bigoplus_{i + j = n} M_i \tensor N_j. $$
	The tensor product extends to graded maps. Let $f: A \to B$ and $g:X \to Y$ be two graded maps, then their tensor product $f \tensor g: A \tensor B \to X \tensor Y$ is defined as:
	$$ (f \tensor g)(a \tensor x) = (-1)^{\deg{a}\deg{g}} \cdot f(a) \tensor g(x). $$
\end{definition}
\todo{graded tor}

The sign is due to \emph{Koszul's sign convention}: whenever two elements next to each other are swapped (in this case $g$ and $a$) a minus sign appears if both elements are of odd degree. More formally we can define a swap map
$$ \tau : A \tensor B \to B \tensor A : a \tensor b \mapsto (-1)^{\deg{a}\deg{b}} b \tensor a. $$

The graded modules together with graded maps of degree $0$ form the category $\grMod{\k}$ of graded modules. From now on we will simply refer to maps instead of graded maps. Together with the tensor product and the ground ring, $(\grMod{\k}, \tensor, \k)$ is a symmetric monoidal category (with the symmetry given by $\tau$). This now dictates the definition of a graded algebra.

\begin{definition}
	A \emph{graded algebra} consists of a graded module $A$ together with two maps of degree $0$:
	$$ \mu: A \tensor A \to A \quad\text{ and }\quad \eta: k \to A $$
	such that $\mu$ is associative and $\eta$ is a unit for $\mu$.

	A map between two graded algebra will be called a \emph{graded algebra map} if the map is compatible with the multiplication and unit. Such a map is necessarily of degree $0$.
\end{definition}

Again these objects and maps form a category, denoted as $\grAlg{\k}$. We will denote multiplication by a dot or juxtaposition, instead of explicitly mentioning $\mu$.

\begin{definition}
	A graded algebra $A$ is \emph{commutative} if for all $x, y \in A$
	$$ x y = (-1)^{\deg{x}\deg{y}} y x. $$
\end{definition}


\section{Differential graded algebra}

\begin{definition}
	A \emph{differential graded module} $(M, d)$ is a graded module $M$ together with a map $d: M \to M$ of degree $-1$, called a \emph{differential}, such that $dd = 0$. A map $f: M \to N$ is a \emph{chain map} if it is compatible with the differential, i.e. $d_N f = f d_M$.
\end{definition}

A differential graded module $(M, d)$ with $M_i = 0$ for all $i < 0$ is a \emph{chain complex}. A differential graded module $(M, d)$ with $M_i = 0$ for all $i > 0$ is a \emph{cochain complex}. It will be convenient to define $M^i = M_{-i}$ in the latter case, so that $M = \bigoplus_{n \in \N} M^i$ and $d$ is a map of \emph{upper degree} $+1$.

\begin{definition}
	Let $(M, d_M)$ and $(N, d_N)$ be two differential graded modules, their tensor product $M \tensor N$ is a differential graded module with the differential given by:
	$$ d_{M \tensor N} = d_M \tensor \id_N + \id_M \tensor d_N. $$
\end{definition}

Finally we come to the definition of a differential graded algebra. This will be a graded algebra with a differential. Of course we want this to be compatible with the algebra structure, or stated differently: we want $\mu$ and $\eta$ to be chain maps.

\begin{definition}
	A \emph{differential graded algebra (dga)} is a graded algebra $A$ together with an differential $d$ such that in addition the \emph{Leibniz rule} holds:
	$$ d(x y) = d(x) y + (-1)^{\deg{x}} x d(y) \quad\text{ for all } x, y \in A. $$
\end{definition}

In general, a map which satisfies the above Leibniz rule is called a \Def{derivation}.

It is not hard to see that the definition of a dga precisely defines the monoidal objects in the category of differential graded modules. The category of dga's will be denoted by $\DGA_\k$, the category of commutative dga's (cdga's) will be denoted by $\CDGA_\k$. If no confusion can arise, the ground ring $\k$ will be suppressed in this notation.

Let $M$ be a DGA, just as before $M$ is called a \emph{chain algebras} if $M_i = 0$ for $i < 0$. Similarly if $M^i = 0$ for all $i < 0$, then $M$ is a \emph{cochain algebra}.

\todo{The notation $\CDGA$ seem to refer to cochain algebras in literature and not arbitrary cdga's.}
\todo{Augmentations}


\Remark{orthogonal-definition}{
	Note that all the above definitions (i.e. the definitions of graded objects, algebras, differentials) are orthogonal, meaning that any combination makes sense. However, keep in mind that we require the structures to be compatible. For example, an algebra with differential should satisfy the Leibniz rule (i.e. the differential should be a map of algebras).
}


\section{Homology}

Whenever we have a differential module we have $d \circ d = 0$, or put in other words: the image of $d$ is a submodule of the kernel of $d$. The quotient of the two graded modules will be of interest. Note that the following definition depends on the differential $d$, however it is often left out from the notation.

\Definition{homology}{
	Given a differential module $(M, d)$ we define the \Def{homology} of $M$ as:
	$$ H(M) = \ker(d) / \im(d). $$
}

If the module has more structure as discussed above, homology will preserve this.
\Remark{homology-preserves-structure}{
	Let $M$ be a differential module. Then homology preserves the following.
	\begin{itemize}
		\item If $M$ is graded, so is $H(M)$, where the grading is given by
		\[ H(M)_i = \ker(\restr{d}{M_i}) / d(M_{i+1}) \]
		\item If $M$ has an algebra structure, then so does $H(M)$, given by
		\[ [z_1] \cdot [z_2] = [z_1 \cdot z_2] \]
		\item If $M$ is a commutative algebra, so is $H(M)$.
		\todo{augmented}
	\end{itemize}
}
Of course the converses need not be true. For example the singular cochain complex associated to a space is a graded differential algebra which is \emph{not} commutative. However, by taking homology one gets a commutative algebra.

Note that taking homology of a differential graded module (or algebra) is functorial. Whenever a map $f: M \to N$ of differential graded modules (or algebras) induces an isomorphism on homology, we say that $f$ is a \emph{quasi isomorphism}.

\begin{definition}
	Let $M$ be a graded module. We say that $M$ is $n$-reduced if $M_i = 0$ for all $i \leq n$. Similarly we say that a graded \todo{augmented} algebra $A$ is $n$-reduced if $A_i = 0$ for all $1 \leq i \leq n$ and $\eta: \k \tot{\iso} A_0$.

	Let $(M, d)$ be a chain complex (or algebra). We say that $M$ is $n$-connected if $H(M)$ is $n$-reduced as graded module (resp. \todo{augmented} algebra). Similarly for cochain complexes.
\end{definition}


\section{Classical results}

We will give some classical known results of algebraic topology or homological algebra. Proofs of these theorems can be found in many places such as \cite{rotman, weibel}.

\begin{theorem}
	(Universal coefficient theorem) Let $C$ be a chain complex and $A$ an abelian group, then there are natural short exact sequences for each $n$:
	$$ 0 \to H_n(C) \tensor A \to H_n(C \tensor A) \to Tor(H_{n-1}(C), A) \to 0 $$
	$$ 0 \to Ext(H_{n-1}(C), A) \to H^n(\Hom(C, A)) \to \Hom(H_n(C), A) \to 0 $$
\end{theorem}

The first statement generalizes to a theorem where $A$ is a chain complex itself. When choosing to work over a field the torsion will vanish and the exactness will induce an isomorphism. This is (one formulation of) the Künneth theorem.

\begin{theorem}
	(Künneth) Assume that $\k$ is a field and let $C$ and $D$ be (co)chain complexes, then there is a natural isomorphism (a linear graded map of degree $0$):
	$$ H(C) \tensor H(D) \tot{\iso} H(C \tensor D), $$
	where we understand both tensors as graded. If $C$ and $D$ are algebras, this isomorphism is an isomorphism of algebras.
\end{theorem}

	
\section{The free cdga}
\label{sec:free-cdga}

Just as in ordinary linear algebra we can form an algebra from any graded module. Furthermore we will see that a differential induces a derivation.

\begin{definition}
	The \emph{tensor algebra} of a graded module $M$ is defined as
	$$ T(M) = \bigoplus_{n\in\N} M^{\tensor n}, $$
	where $M^{\tensor 0} = \k$. An element $m = m_1 \tensor \ldots \tensor m_n$ has a \emph{word length} of $n$ and its degree is $\deg{m} = \sum_{i=i}^n \deg{m_i}$. The multiplication is given by the tensor product (note that the bilinearity follows immediately).
\end{definition}

Note that this construction is functorial and it is free in the following sense.

\begin{lemma}
	Let $M$ be a graded module and $A$ a graded algebra.
	\begin{itemize}
		\item A graded map $f: M \to A$ of degree $0$ extends uniquely to an algebra map $\overline{f} : TM \to A$.
		\item A differential $d: M \to M$ extends uniquely to a derivation $d: TM \to TM$.
	\end{itemize}
\end{lemma}

\begin{corollary}
	Let $U$ be the forgetful functor from graded algebras to graded modules, then $T$ and $U$ form an adjoint pair:
	$$ T: \grMod{\k} \leftadj \grAlg{\k} :U $$
	Moreover it extends and restricts to
	$$ T: \dgMod{\k} \leftadj \dgAlg{\k} :U $$
	$$ T: \CoCh{\k} \leftadj \DGA{\k} :U $$
\end{corollary}

As with the symmetric algebra and exterior algebra of a vector space, we can turn this graded tensor algebra in a commutative graded algebra.

\begin{definition}
	Let $A$ be a graded algebra and define
	$$ I = \langle ab - (-1)^{\deg{a}\deg{b}}b a \I a,b \in A \rangle $$
	Then $A / I$ is a commutative graded algebra.

	For a graded module $M$ we define the \emph{free commutative graded algebra} as
	$$ \Lambda(M) = TM / I $$
\end{definition}

Again this extends to differential graded modules (i.e. the ideal is preserved by the derivative) and restricts to cochain complexes.

\begin{lemma}
	We have the following adjunctions:
	$$ \Lambda: \grMod{\k} \leftadj \grAlg{\k}^{comm} :U $$
	$$ \Lambda: \dgMod{\k} \leftadj \dgAlg{\k}^{comm} :U $$
	$$ \Lambda: \CoCh{\k} \leftadj \CDGA_\k :U $$
\end{lemma}

We can now easily construct cdga's by specifying generators and their differentials. Note that a free algebra has a natural augmentation, defined as $\counit(v) = 0$ for every generator $v$ and $\counit(1) = 1$.

	
\section{Model categories}
\label{sec:model_categories}

\newcommand{\W}{\mathfrak{W}}
\newcommand{\Fib}{\mathfrak{Fib}}
\newcommand{\Cof}{\mathfrak{Cof}}

\begin{definition}
	A \emph{(closed) model category} is a category $\cat{C}$ together with three subcategories:
	\begin{itemize}
		\item the class of weak equivalences $\W$,
		\item the class of fibrations $\Fib$ and
		\item the class of cofibrations $\Cof$,
	\end{itemize}
	such that the following five axioms hold:
	\begin{itemize}
		\item[MC1] All finite limits and colimits exist in $\cat{C}$.
		\item[MC2] If $f$, $g$ and $fg$ are maps such that two of them are weak equivalences, then so it the third. This is called the \emph{2-out-of-3} property.
		\item[MC3] All three classes of maps are closed under retracts\todo{Either draw the diagram or define a retract earlier}.
		\item[MC4] In any commuting square as follows where $i \in \Cof$ and $p \in \Fib$,
		\begin{center}
		\begin{tikzpicture}
		\matrix (m) [matrix of math nodes]{
			A & X \\
			B & Y \\
		};

		\path[->] (m-1-1) edge (m-1-2);
		\path[->] (m-2-1) edge (m-2-2);
		\path[->] (m-1-1) edge node[auto] {$i$} (m-2-1);
		\path[->] (m-1-2) edge node[auto] {$p$} (m-2-2);

		\end{tikzpicture}
		\end{center}

		 there exist a lift $h: B \to Y$ if either 
		\begin{itemize}
			\item[a)] $i \in \W$ or
			\item[b)] $p \in \W$.
		\end{itemize}
		\item[MC5] Any map $f : A \to B$ can be factored in two ways:
		\begin{itemize}
			\item[a)] as $f = pi$, where $i \in \Cof \cap \W$ and $p \in \Fib$ and
			\item[b)] as $f = pi$, where $i \in \Cof$ and $p \in \Fib \cap \W$.
		\end{itemize}
	\end{itemize}
\end{definition}

\begin{notation} For brevity
	\begin{itemize}
		\item we write $f: A \fib B$ when $f$ is a fibration,
		\item we write $f: A \cof B$ when $f$ is a cofibration and
		\item we write $f: A \we B$ when $f$ is a weak equivalence.
	\end{itemize}
\end{notation}

\begin{definition}
	An object $A$ in a model category $\cat{C}$ will be called \emph{fibrant} if $A \to \cat{1}$ is a fibration and \emph{cofibrant} if $\cat{0} \to A$ is a cofibration.
\end{definition}

Note that axiom [MC5a] allows us to replace any object $X$ with a weakly equivalent fibrant object $X^{fib}$ and by [MC5b] by a weakly equivalent cofibrant object $X^{cof}$, as seen in the following diagram:

\begin{center}
\begin{tikzpicture}
\matrix (m) [matrix of math nodes]{
	\cat{0} &         & X \\
	        & X^{cof} &   \\
};

\path[->] (m-1-1) edge (m-1-3);
\path[right hook->] (m-1-1) edge (m-2-2);
\path[->>] (m-2-2) edge node[auto] {$ \simeq $} (m-1-3);

\end{tikzpicture}\quad
\begin{tikzpicture}
\matrix (m) [matrix of math nodes]{
	X &         & \cat{1} \\
	  & X^{fib} &         \\
};

\path[->] (m-1-1) edge (m-1-3);
\path[right hook->] (m-1-1) edge node[auto] {$ \simeq $} (m-2-2);
\path[->>] (m-2-2) edge (m-1-3);

\end{tikzpicture}
\end{center}

\TODO{Maybe some basic propositions (refer to Dwyer \& Spalinski):
\titem Over/under category (or simply pointed objects)
\titem If a map has LLP/RLP w.r.t. fib/cof, it is a cof/fib
\titem Fibs are preserved under pullbacks/limits
\titem Cofibrantly generated mod. cats.
\titem Small object argument
}

\Example{top-model-structure}{
	The category $\Top$ of topological spaces admits a model structure as follows.
	\begin{itemize}
		\item Weak equivalences: maps inducing isomorphisms on all homotopy groups.
		\item Fibrations: Serre fibrations, i.e. maps with the right lifting property with respect to the inclusions $D^n \cof D^n \times I$.
		\item Cofibrations: maps $S^{n-1} \cof D^n$ and transfinite compositions of pushouts and coproducts thereof.
	\end{itemize}
}

\Example{sset-model-structure}{
	The category $\sSet$ of simplicial sets has the following model structure.
	\begin{itemize}
		\item Weak equivalences: 
		\item Fibrations: Kan fibrations, i.e. maps with the right lifting property with respect to the inclusions $\Lambda_n^k \cof \Delta[n]$.
		\item Cofibrations: all monomorphisms.
	\end{itemize}
}

In this thesis we often restrict to $1$-connected spaces. The full subcategory $\Top_1$ of $1$-connected spaces satisfies MC2-MC5: the 2-out-of-3 property, retract property and lifting properties hold as we take the \emph{full} subcategory, factorizations exist as the middle space is $1$-connected as well. However $\Top_1$ does not have all limits and colimits.

\Lemma{topr-no-colimit}{
	Let $r > 0$ and $\Top_r$ be the full subcategory of $r$-connected spaces. The diagrams

	\cimage[scale=0.5]{Topr_No_Coequalizer}
	\cimage[scale=0.5]{Topr_No_Equalizer}

	have no coequalizer and respectively no equalizer in $\Top_r$.
}

\todo{Define homotopy category}

\subsection{Quillen pairs}
In order to relate model categories and their associated homotopy categories we need a notion of maps between them. We want the maps such that they induce maps on the homotopy categories.
\todo{Definition etc}

\end{appendices}

% \listoftodos

\bibliographystyle{alpha}
\bibliography{references}

\end{document}
